\chapter*{Appendix}

%%%%%%%%%%%%%%%%%%%%%%%%%%%%%%%%%%%%%
\section*{Python code for determining the set $\Delta(G,q)$}

For a DAG $G$, the following code determines which nodes are in the se $\Delta(G,q)$ for a query $q$. It uses two functions:
\begin{enumerate}
\item \texttt{find\_case}, a helper function which determines whether nodes are in the special case or the general case according to the logic from ~\cref{thm:querycost}.
\item \texttt{create\_delta\_set}, the main function which populates the set $\Delta(G,q)$ depending on the result of \texttt{find\_case}.
\end{enumerate}


\begin{python}

# Helper function for create_delta_set(); checks for paths between targets and parents of conditions.
# Inputs: a DAG $G$, a target vertex $X$, a single conditional vertex $Y \in \{Y_{0},Y_{1},...Y_{n}\}$ for the query $q = (X|Y_{0},Y_{1},....Y_{n})$.
#Returns: the case of individual vertex $Y$ in the query $q$, either special or general.

def find_case(G,X,Y,cases):
    skeleton = G.to_undirected()
    y_ancestors = list(nx.ancestors(G,Y))
    paths = list(nx.all_simple_paths(skeleton, X, Y, cutoff=None)) #use skeleton so they are undirected (routes not paths).

    if len(y_ancestors) == 0:
        cases[Y] = "special" #Yj in trivial special case
        return(cases)

    # The logic of the following is to find paths from X to Y: if any of them pass through an ancestor of Y then there is a "back route" to Y through its ancestors, meaning there is a path connecting X to a(Y)
    # which does not pass through Y. 
    for path in paths:
        for y_ancestor in y_ancestors:
            if y_ancestor in path: #There exists a path from X to a(Y) which does not pass through Y
                cases[Y] = "general" #Yj in general case 
                return(cases)
            else:
                continue
            
    #Only reaches this block of code if all paths to a(Y) passed through Y; in nontrivial special case.
    cases[Y] = "special" #Yj in the special case
    return(cases)	
    
\end{python}
    
    
\begin{python}
  
#Inputs: A DAG G, a target vertex X, and conditional vertices Ys = {Y_{0},Y_{1},...Y_{n}})$corresponding to a query q = p(X|Y_{0}, Y_{1},...Y_{n}). 
#Returns: the set Delta(G,q). 

def create_delta_set(G,X,Ys): #Graph G, one target X, set of multiple conditions Ys = {y0, y1, ...}.
    skeleton = G.to_undirected()
    
    dependent = []
    cases = {} #Dictionary with (key, value) pairs: (Y,case).
    if not nx.is_connected(skeleton): #Only consider connected graphs.
        return(0)
    if len(Ys) == 0:
        return(list(nx.ancestors(G,X)))
    
    for Y in Ys:
        cases = check_paths(G,X,Y,cases)
    
    shortest_path = nx.shortest_path(skeleton,X,Y) #use skeleton so we have shortest route (undirected)
    x_ancestors = list(nx.ancestors(G,X))
    y_ancestors = list(nx.ancestors(G,Y))
    
    dependent.append(X)
    for x_ancestor in x_ancestors:
        dependent.append(x_ancestor)

    for Y in Ys:
        if cases[Y] == "general":
            for y_ancestor in y_ancestors:
                dependent.append(y_ancestor)
                
        if cases[Y] == "special":
            for node in shortest_path:
                dependent.append(node)
            dependent = list(set(dependent))
            dependent.remove(Y)
            for y_ancestor in y_ancestors:
                if y_ancestor in dependent:
                    dependent.remove(y_ancestor)
    
    return(list(set(dependent)))
    
\end{python}


%%%%%%%%%%%%%%%%%%%%%%%%%%%%%%%%%%%%%
\section*{Python code for identifying advantageous reversals}

\null \quad \quad The following pieces of code determine which edge reversal is most advantageous at a single given time given a Dag $G$ and a query $q = (X|Y_{0},Y_{1},...,Y_{n})$. The first piece, \texttt{find\_covered\_edges} searches for covered edges since they are equivalent to immediately reversible edges. It does so by comparing the parents of each of the two vertices corresponding to an edge in $G$. 

\begin{python}
# Inputs: A DAG G
# Return the set of covered edges in a graph. Independent of query. 

def find_covered_edges(G):
    covered_edges = []
    for edge in G.edges: #for edge (X,Y)
        source_parents = list(G.predecessors(edge[0])) #parents of X
        target_parents = list(G.predecessors(edge[1])) #parents of Y
        source_parents.append(edge[0]) #parents of X union X
        if set(source_parents) == set(target_parents):
            covered_edges.append(edge)
    return(covered_edges)

\end{python}

\null \quad \quad Next, \texttt{is\_advantageous} takes in a DAG $G$, a covered edge $e$, and the parameters of the query $q = p(X|Y_{0},Y_{1},...,Y_{n})$. It creates a graph $Grev$, the graph which results if $e$ is reversed, and calculates $\Delta(Grev,q)$. Then, the difference between $\Delta(G,q)$ and $\Delta(Grev,q)$ is returned, indicating how advantageous the reversal is. 

\begin{python}
# Inputs: A DAG G, a covered edge, the parameters of the query q = p(X|Ys).
# Returns: The number of vertices Delta(G,q) is reduced by if the edge is reversed, that is, how advantageous the reversal is. 

def is_advantageous(G,edge, X, Ys):
    vertex_cost_difference = 0
    cost_before = len(find_case(G, X, Ys))
    
    x = edge[0]
    y = edge[1]
    Grev = G.copy()
    Grev = structure_reverse_edge(Grev,x,y)
    cost_after = len(find_case(Grev,X,Ys))
    if cost_after < cost_before:
        vertex_cost_difference = cost_before - cost_after
    return(vertex_cost_difference)

\end{python}

\null \quad \quad Finally, \texttt{best\_reversal} iterates through the set of covered edges in $G$ to determine which covered edge is most advantageous when reversed, and returns this edge. 

\begin{python}
# Input: A DAG G, target vertex X, the set of conditional vertices Ys which correspond to the query q = p(X|Ys). 
# Returns: the most advantageous edge reversal possible at a given time. 
def best_reversal(G, X, Ys):
    covered_edges = find_covered_edges(G) 
    edge_advantages = {}
    for edge in covered_edges:
         edge_advantages[edge] = is_advantageous(G,edge,X,Ys)
    max_adv_edge = max(edge_advantages, key=edge_advantages.get)
    return(max_adv_edge)

\end{python}
\newpage

%%%%%%%%%%%%%%%%%%%%%%%%%%%%%%%%%%%%%
\section*{Python code for encoding conditional node data}

\null \quad \quad The following code assume binary variables, that is, for each vertex $X$ in a DAG $G$ corresponding to a random variable, $X = x \in \{0,1\}.$ Encoding even binary variables is expensive and nontrivial, but the same arguments can be extended to categoricals for those patient enough to implement it. \newline
\null \quad \quad Each vertex in $G$ is specified by a set of conditionals corresponding to the values its parents can take. Therefore, the number of conditionals corresponds to the number of parents it has. For example, if $X$ is a parentless vertex, it will be specified by $p(X=1)$. Since we are in the binary case, this is sufficient, since $p(X=0) = 1-p(X=1)$.\newline
\null \quad \quad If the vertex $Y$ has a single parent $X$, it will be specified by $p(Y=1|X=0)$ and $p(Y=1|X=1)$ from which one can again determine $p(Y=0|X=0)$ and $p(Y=0|X=1)$. A node $Z$ with parents $X$ and $Y$ will be specified by $p(Z=1|X=0,Y=0)$, $p(Z=1|X=0,Y=1)$, $p(Z=1|X=1,Y=0)$ and $p(Z=1|X=1,Y=1)$. Generally, a vertex with $P$ parents will be specified by $2^{P}$ conditionals. Therefore, one must construct a dynamic data structure such as the following, which accounts for the variety in number of parents (and therefore conditionals) a vertex can have. 

\begin{python}
import itertools
class NodeData:
    '''
    Attributes:
        parents: a list of strings that contains the names of the parent nodes
        conditionals: a list of size 2^p (where p is the number of parents) 
            that contains the conditional probabilities of the RV of this node given all possible values for the parents
    '''
    def __init__(self, name, parents=None, conditionals=None):
        self.name = name
        self.parents = parents

        # initializing the data structures,
        # mostly stuff that ensures that everything also works for nodes without parents
        if parents == None:
            self.num_parents = 0
        else:
            self.num_parents = len(self.parents)

        if conditionals == None:
            self.conditionals = [0 for _ in range(2**self.num_parents)]
        else:
            self.conditionals = conditionals

            
    def set_conditional(self, p, parent_assignment=None):
        '''
        parent_assignment: a dict that maps values to names of parents, e.g.
        {'x1' : 0, 'x2' : 1, 'x3' : 0, ...}
        this function computes the index of self.conditionals where the required value is stored and sets the conditional probability
        '''
        # handle special case of node without parents:
        if self.parents == None:
            self.conditionals[0] = p
            return
        # general case:
        index = 0
        for p_i in range(len(self.parents)):
            index += parent_assignment[self.parents[p_i]] * 2**p_i

        self.conditionals[index] = p

        
    def get_conditional(self, parent_assignment):
        '''
        parent_assignment: a dict that maps values to names of parents, e.g.
        {'x1' : 0, 'x2' : 1, 'x3' : 0, ...}
        this function computes the index of self.conditionals where the required value is stored and returns the conditional
        '''
        
        ##input of get conditionals is parent_assignment, dict which grows linearly with number of parents. 
        ##So we can calculate this in terms of numparsx. 
        
        # handle special case of node without parents:
        if self.parents == None:
            return self.conditionals[0]
        # general case:
        index = 0
        for p_i in range(len(self.parents)):
            index += parent_assignment[self.parents[p_i]] * 2**p_i

        return self.conditionals[index]

    
    def print_conditionals(self):
        '''
        prints the full conditional distribution
        iterates through all possible assignments and prints the conditional distribution for each assignment in a seperate line
        '''
        all_possible_parent_assignments = [x for x in itertools.product([0,1], repeat=self.num_parents)]

        for a_i in range(2**self.num_parents):
            assignment = {self.parents[i] : all_possible_parent_assignments[a_i][i] for i in range(self.num_parents)}
            # some list-magic that produces a sufficiently nice string:
            assignment_string = "P("+self.name+" = 1 | " + ', '.join([str(x) 
            		+ " = " + str(assignment[x]) for x in assignment]) + " ) 
            		= " + str(self.get_conditional(assignment))
            print (assignment_string)
\end{python}

%%%%%%%%%%%%%%%%%%%%%%%%%%%%%%%%%%%%%
\section*{Python code for calculating edge reversals in a binary setting}

Likewise to the data structures above, the following code assumes binary variables. It must also dynamically access the conditional probabilities specifying each node, since the length of specifiers of a vertex with $P$ parents is $2^{P}$. 

\begin{python}
def switch_edge(x, y):
    '''
    switches the edge (x,y) to (y,x)
    x, y: NodeData
    '''
    # preparation:
    # use itertools.product to get all possible assignments of values for the parent nodes:
    
    all_possible_parent_assignments = list(itertools.product([0,1], repeat=x.num_parents))
    num_parents_assigments = len(all_possible_parent_assignments)

    # get conditional joint distribution p(x,y | parents):
    # initialization:
    pxe0ye0 = [0 for _ in range(2**x.num_parents)]  
    pxe0ye1 = [0 for _ in range(2**x.num_parents)]
    pxe1ye0 = [0 for _ in range(2**x.num_parents)]
    pxe1ye1 = [0 for _ in range(2**x.num_parents)]
    
    
    # iterate through all possible assignments for the common parents of x and y:
    for a_i in range(num_parents_assigments):
        
        assignemnt_dict_x = {x.parents[i] : all_possible_parent_assignments[a_i][i] for i in range(x.num_parents)}
        assignemnt_dict_y = {x.parents[i] : all_possible_parent_assignments[a_i][i] for i in range(x.num_parents)}
        assignemnt_dict_y[x.name] = 0
        
        # case x=0, y=1:
        pxe0ye1[a_i] = (1-x.get_conditional(assignemnt_dict_x)) 
        		* y.get_conditional(assignemnt_dict_y)
        # case x=0, y=0:
        pxe0ye0[a_i] = (1-x.get_conditional(assignemnt_dict_x)) 
        		* (1-y.get_conditional(assignemnt_dict_y))

        assignemnt_dict_y[x.name] = 1
        # case x=1, y=1:
        pxe1ye1[a_i] = x.get_conditional(assignemnt_dict_x) 
        		* y.get_conditional(assignemnt_dict_y)
        # case x=1, y=0:
        pxe1ye0[a_i] = x.get_conditional(assignemnt_dict_x) 
        		* (1-y.get_conditional(assignemnt_dict_y))

    
    # marginalize x from the conditional distribution of y:
    pye0 = [pxe0ye0[i] + pxe1ye0[i] for i in range(num_parents_assigments)]
    pye1 = [pxe0ye1[i] + pxe1ye1[i] for i in range(num_parents_assigments)]

    # condition x on the possible values of y:
    pxe0gye0 = [pxe0ye0[i]/pye0[i] for i in range(num_parents_assigments)]
    pxe0gye1 = [pxe0ye1[i]/pye1[i] for i in range(num_parents_assigments)]
    pxe1gye0 = [pxe1ye0[i]/pye0[i] for i in range(num_parents_assigments)]
    pxe1gye1 = [pxe1ye1[i]/pye1[i] for i in range(num_parents_assigments)]

    # construct new nodedata objects:
    x_new_parents = [y.name]
    if not x.parents == None:
        x_new_parents += x.parents
    x_new = NodeData(x.name, parents=x_new_parents)

    for a_i in range(num_parents_assigments):
        
        # construct the assignment-dictionary:
        assignment_dict = {x.parents[i] : all_possible_parent_assignments[a_i][i] for i in range(x.num_parents)}

        # case y=0:
        # add y to assignment_dict:
        assignment_dict[y.name] = 0
        # set conditional for x_new:
        x_new.set_conditional(p=pxe1gye0[a_i], parent_assignment=assignment_dict)

        # case y=1
        # add y to assignment_dict:
        assignment_dict[y.name] = 1
        # set conditional for x_new:
        x_new.set_conditional(p=pxe1gye1[a_i], parent_assignment=assignment_dict)

    y_new_parents = []
    
    if not x.parents == None:
        y_new_parents += x.parents
        
    y_new = NodeData(y.name, parents=y_new_parents)
    
    for a_i in range(num_parents_assigments):
        # construct the assignment-dictionary:
        assignment_dict = {x.parents[i] : all_possible_parent_assignments[a_i][i] for i in range(x.num_parents)}
        y_new.set_conditional(p=pye1[a_i], parent_assignment=assignment_dict)

    return x_new, y_new
\end{python}