{
\null\vspace{\fill}
{\center{\section*{Abstract}}}
\addcontentsline{toc}{section}{Abstract}
As observational data collection becomes more accessible, the importance of modeling multivariate probability distributions and answering inference queries efficiently increases. Probabilistic dependencies are often encoded in directed acyclic graphs (DAGs) and associated probability distributions, which together comprise a Bayesian network. Bayesian networks have the property of non-identifiability, meaning that several structurally distinguishable networks can encode the same probability distribution. Such Bayesian networks are called \textit{Markov equivalent}. Resultingly, the same inference queries asked over different Markov-equivalent Bayesian networks will always produce statistically indistinguishable results. This thesis explores whether it can be computationally beneficial to exploit this non-identifiability property for faster inference; that is, whether one can achieve speed-up in answering a sequence of inference queries by changing the representation of a Bayesian network on the fly to serve the queries. This exploration yields a short survey of related topics, a theorem for identifying the set of vertices a query depends on, an algorithm for making beneficial network transformations, and a discussion of their efficacy.  \newline 

\textbf{Keywords}: Bayesian Networks, causal models, inference queries, optimization, non-identifiability, Markov Equivalence.

\vspace{\fill}
}

{
\newpage
\null\vspace{\fill}
{\center{\section*{Zusammenfassung}}}

\begin{otherlanguage}{german}

Mit stetig besser werdendem Zugang zur Erfassung von beobachteten Daten wächst die Wichtigkeit effizienter Modellierung multivariater Wahrscheinlichkeitsverteilungen und der Auswertung von Inferenzanfragen. Stochastische Abhängigkeiten werden häufig abgebildet in gerichteten azyklischen Graphen (DAG; engl. directed acyclic graph) und von ihnen abhängingen Wahrscheinlichkeitsverteilungen, die zusammen ein Bayessches Netz bilden. Bayessche Netze zeichnen sich durch Nicht-Eindeutigkeit aus, was bedeutet, dass meherere strukturell verschiedene Netze dieselbe Wahrscheinlichkeitsverteilung beschreiben können. Diese Eigenschaft heißt \textit{Markov-Äquivalenz}. Eine Konsequenz daraus ist, dass die gleichen Inferenzanfragen an verschiedene Markov-äquivalente Bayessche Netze immer die gleichen Ergebnisse liefern werden. Diese Arbeit untersucht, ob es mit Bezug auf den Rechenaufwand vorteilhaft sein kann, diese Nicht-Eindeutigkeit auszunutzen, um eine schnellere Inferenz zu erreichen; das heißt, ob eine schnellere Beantwortung von Inferenzanfragen möglich ist, indem spontan eine andere Representation des Bayesschen Netzes gewählt wird. Das Ergebnis dieser Untersuchung ist eine kurze Übersicht verwandter Themen, ein Theorem für die Identifikation der Teilmenge an Knoten, von denen die Anfrage abhängt, ein Algorithmus für vorteilhafte Transformationen eines Netzes und eine Diskussion ihrer Wirksamkeit.\newline

\end{otherlanguage}

\vspace{\fill}
\newpage
}