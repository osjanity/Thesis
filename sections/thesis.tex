%%%%%%%%%%%%%%%%%%%%%%%%%%%%%
% structure of the document %
%%%%%%%%%%%%%%%%%%%%%%%%%%%%%


\documentclass[12pt, a4paper, twoside]{report}


\usepackage[paper=a4paper, left=20mm, right=20mm, bottom=10mm, top=10mm]{geometry}
\usepackage[T1]{fontenc} % Use 8-bit encoding that has 256 glyphs
\usepackage{fourier} % Use the Adobe Utopia font for the document - comment this line to return to the LaTeX default
\usepackage[german,english]{babel} % English language/hyphenation
\usepackage{amsmath,amsfonts,amsthm} % Math packages
\usepackage{listings}
\usepackage{color}
\usepackage{url}
\usepackage[hidelinks]{hyperref}
\usepackage{cleveref}
\usepackage{algpseudocode}
\usepackage{framed}

\usepackage{subcaption,graphicx}
\usepackage{tikz-network}
\usepackage{xcolor}
\usepackage[utf8]{inputenc}
\usepackage{cleveref}
\usepackage[ruled,linesnumbered,dotocloa]{algorithm2e}
\usepackage{mathtools}
\usepackage{multirow}
\usepackage{blindtext}
\usepackage{caption}
\captionsetup{font=footnotesize}
\usepackage{setspace}
\usetikzlibrary{decorations.pathreplacing}
\usepackage{booktabs}
\usepackage{tabularx}
\usepackage{makecell}
\usepackage{tabu, booktabs}

%for formatting python snippets
\usepackage{pythonhighlight} 
\usepackage{pythontex}

\renewcommand\theadfont{\bfseries}
\renewcommand\theadgape{\Gape[4pt]}
\renewcommand\cellgape{\Gape[4pt]}

\DeclarePairedDelimiter\abs{\lvert}{\rvert}%
\DeclarePairedDelimiter\norm{\lVert}{\rVert}%

\DeclareMathOperator*{\argmax}{arg\,max}
\DeclareMathOperator*{\argmin}{arg\,min}

% configure image path
\graphicspath{ {./images/} }

% nicer font
\usepackage{mathpazo}

% nomenclature
\usepackage[intoc, english]{nomencl}
\makenomenclature

\usepackage{setspace}
\onehalfspacing
\raggedbottom

% meta data
\newcommand{\ttitle}{Exploiting Markov Equivalence \\ for Fast Inference}
\newcommand{\tauthor}{Jenette Sellin}
\newcommand{\tdate}{\today}

% additional macros
\newcommand{\code}[1]{\texttt{#1}}

% Theorems, lemmas
\newtheoremstyle{spacethm}% name of the style to be used
  {10pt}% measure of space to leave above the theorem. E.g.: 3pt
  {20pt}% measure of space to leave below the theorem. E.g.: 3pt
  {\itshape}% name of font to use in the body of the theorem
  {0pt}% measure of space to indent
  {\bfseries}% name of head font
  {.}% punctuation between head and body
  { }% space after theorem head; " " = normal interword space
  {\thmname{#1}\thmnumber{ #2}\thmnote{ (#3)}}
  
  \newtheoremstyle{spaceex}% name of the style to be used
  {10pt}% measure of space to leave above the theorem. E.g.: 3pt
  {20pt}% measure of space to leave below the theorem. E.g.: 3pt
  {}% name of font to use in the body of the theorem
  {0pt}% measure of space to indent
  {\bfseries}% name of head font
  {.}% punctuation between head and body
  { }% space after theorem head; " " = normal interword space
  {\thmname{#1}\thmnumber{ #2}\thmnote{ (#3)}}
  

% some custom 'theorem'-like environments that each get their own counter  

\theoremstyle{spacethm}% default
\newtheorem{theorem}{Theorem}[section]
\newtheorem{lemma}[theorem]{Lemma}
\newtheorem{proposition}[theorem]{Proposition}
\newtheorem{corollary}{Corollary}
\newtheorem{definition}{Definition}[section]

\theoremstyle{spaceex}
\newtheorem{conjecture}{Conjecture}[section]
\newtheorem{example}{Example}[section]
\newtheorem*{remark}{Remark}
\newtheorem*{note}{Note}
\newtheorem{case}{Case}
\newtheorem*{notation}{Notation}


\crefname{lemma}{Lemma}{Lemmas}
\crefname{theorem}{Theorem}{Theorems}
\crefname{definition}{Definition}{Definitions}
\crefname{algorithm}{Algorithm}{Algorithms}

%---------------

\lstset{frame=tb,
  language=Java,
  aboveskip=3mm,
  belowskip=3mm,
  showstringspaces=false,
  columns=flexible,
  basicstyle={\small\ttfamily},
  numbers=none,
  numberstyle=\tiny\color{gray},
  keywordstyle=\color{blue},
  commentstyle=\color{dkgreen},
  stringstyle=\color{mauve},
  breaklines=true,
  breakatwhitespace=true,
  tabsize=3
}
%---------------
\numberwithin{equation}{section} % Number equations within sections (i.e. 1.1, 1.2, 2.1, 2.2 instead of 1, 2, 3, 4)
\numberwithin{figure}{section} % Number figures within sections (i.e. 1.1, 1.2, 2.1, 2.2 instead of 1, 2, 3, 4)
\numberwithin{table}{section} % Number tables within sections (i.e. 1.1, 1.2, 2.1, 2.2 instead of 1, 2, 3, 4)

\setlength\parindent{0pt} % Removes all indentation from paragraphs

\newenvironment{myindentpar}[1]
 {\begin{list}{}
         {\setlength{\leftmargin}{#1}}
         \item[]
 }
 {\end{list}}
 
 \newcommand{\ME}{\sim_{M}}

%---------------------bibstuff
 \bibliographystyle{ieeetr}
% \bibliographystyle{apa} 
%---------------------/bibstuff

%---------------

% start here
\begin{document}

% environment up until table of contents
{
% start new sections on an odd page
%\newcommand*\stdsection{}
%\let\stdsection\section
%\renewcommand*\section{%
%\clearpage\ifodd\value{page}\else{\mbox{}\thispagestyle{empty}}\clearpage\fi
%\stdsection}

% no page numbers on first page
\pagenumbering{gobble}
\clearpage
\thispagestyle{empty}

% title page
\begin{titlepage}

\newcommand{\HRule}{\rule{\linewidth}{0.5mm}}

% center everything on the page
\center

% logo
\includegraphics[width=400px]{logo.jpg}\\[0.5cm]

% title
\HRule \\[0.4cm]
{ \huge \bfseries \ttitle}\\[0.4cm]
\HRule \\[1.0cm]
 
% headings
\textsc{\LARGE Master Thesis}\\[0.5cm]
{\Large Submitted for the degree of\\[0.2cm]
\textsc{Master of Science (M.Sc.)}}\\[1.0cm]

\textsc{\LARGE Friedrich Schiller University Jena}\\[0.5cm]
\textsc{\Large Faculty of Mathematics and Computer Science}\\[0.2cm]
\textsc{\large Mathematics}\\[1.0cm]
 
% author
\begin{minipage}[t]{0.4\textwidth}
\begin{flushleft} \large
\emph{Author}\\
Jenette \textsc{Sellin} \\[0.5cm]
Born December 22, 1995 \\
in \textsc{California}, \textsc{USA}
\end{flushleft}
\end{minipage}
~
\begin{minipage}[t]{0.4\textwidth}
\begin{flushright} \large
\emph{Advisor} \\
Andreas \textsc{Goral} \\ [0.5cm]
\emph{Supervisor} \\
Prof. Dr. Joachim \textsc{Giesen} 
\end{flushright}
\end{minipage}

% date
\vspace{8em}
{\large Jena, \today}\\[2cm]
 
\vfill % Fill the rest of the page with whitespace

\end{titlepage}

\newgeometry{
    left=40mm,
    right=20mm,
    top=30mm,
    bottom=30mm
}

% blank page after title page
\shipout\null
\newpage

\pagenumbering{Roman}

\newgeometry{
    left=40mm,
    right=20mm,
    top=30mm,
    bottom=30mm
}

% abstract
\newpage
{
\null\vspace{\fill}
{\center{\section*{Abstract}}}
\addcontentsline{toc}{section}{Abstract}
As observational data collection becomes more accessible, the importance of modeling multivariate probability distributions and answering inference queries efficiently increases. Probabilistic dependencies are often encoded in directed acyclic graphs (DAGs) and associated probability distributions, which together comprise a Bayesian network. Bayesian networks have the property of non-identifiability, meaning that several structurally distinguishable networks can encode the same probability distribution. Such Bayesian networks are called \textit{Markov equivalent}. Resultingly, the same inference queries asked over different Markov-equivalent Bayesian networks will always produce statistically indistinguishable results. This thesis explores whether it can be computationally beneficial to exploit this non-identifiability property for faster inference; that is, whether one can achieve speed-up in answering a sequence of inference queries by changing the representation of a Bayesian network on the fly to serve the queries. This exploration yields a short survey of related topics, a theorem for identifying the set of vertices a query depends on, an algorithm for making beneficial network transformations, and a discussion of their efficacy.  \newline 

\textbf{Keywords}: Bayesian Networks, causal models, inference queries, optimization, non-identifiability, Markov Equivalence.

\vspace{\fill}
}

{
\newpage
\null\vspace{\fill}
{\center{\section*{Zusammenfassung}}}

\begin{otherlanguage}{german}

Mit stetig besser werdendem Zugang zur Erfassung von beobachteten Daten wächst die Wichtigkeit effizienter Modellierung multivariater Wahrscheinlichkeitsverteilungen und der Auswertung von Inferenzanfragen. Stochastische Abhängigkeiten werden häufig abgebildet in gerichteten azyklischen Graphen (DAG; engl. directed acyclic graph) und von ihnen abhängingen Wahrscheinlichkeitsverteilungen, die zusammen ein Bayessches Netz bilden. Bayessche Netze zeichnen sich durch Nicht-Eindeutigkeit aus, was bedeutet, dass meherere strukturell verschiedene Netze dieselbe Wahrscheinlichkeitsverteilung beschreiben können. Diese Eigenschaft heißt \textit{Markov-Äquivalenz}. Eine Konsequenz daraus ist, dass die gleichen Inferenzanfragen an verschiedene Markov-äquivalente Bayessche Netze immer die gleichen Ergebnisse liefern werden. Diese Arbeit untersucht, ob es mit Bezug auf den Rechenaufwand vorteilhaft sein kann, diese Nicht-Eindeutigkeit auszunutzen, um eine schnellere Inferenz zu erreichen; das heißt, ob eine schnellere Beantwortung von Inferenzanfragen möglich ist, indem spontan eine andere Representation des Bayesschen Netzes gewählt wird. Das Ergebnis dieser Untersuchung ist eine kurze Übersicht verwandter Themen, ein Theorem für die Identifikation der Teilmenge an Knoten, von denen die Anfrage abhängt, ein Algorithmus für vorteilhafte Transformationen eines Netzes und eine Diskussion ihrer Wirksamkeit.\newline

\end{otherlanguage}

\vspace{\fill}
\newpage
}
\newpage


%acknowledgements
\newpage
\section*{Acknowledgements}

\null \quad \quad Many thanks to Professor Dr. Joachim Giesen for the consistent guidance and inspiration throughout the Master's program, numerous excellent lectures, and for helping me arrive at such an engaging thesis topic. Thanks to Dr. S\"{o}ren Laue and Prof. Dr. Joachim Giesen for serving as the reviewers of the thesis. \newline
\null \quad \quad This work would not have been possible without the thoughtful and incredibly thorough feedback across intercontinental time zones from my advisor Andreas Goral.\newline 
\null \quad \quad Thanks to Tristan Kreuziger for his detailed responses to my many questions on German translations, and to Lisa and Evin for flipping coins. \newline
\null \quad \quad I would also like to briefly note that I did not particularly expect to pass my $9$th grade algebra course, and had once written off the idea of being able to learn basic calculus. Endless thanks to my many mentors, supporters, and friends along the way for the ongoing encouragement to pursue math, even when it felt against the odds of success. \newline 

\newpage

%notation
\newpage
\section*{Notation and Abbreviations}

\begin{table}[h!]
  \begin{center}
      \begin{spacing}{1.4}
    \begin{tabular}{  l  l }

      \hline
	
	 DAG & Directed Acyclic Graph. \\
	 iff & if and only if. \\
	 $G = (V,E)$ & a graph $G$ with vertices $V$ and edges $E$. \\
	 $G, G', G_{1}, G_{2}, ..., G_{n}$ & directed graphs. \\ 
	$H, H', H_{1}, H_{2}, ..., H_{n}$ & undirected and mixed graphs. \\ 
	$B, B', B_{1}, B_{2}, ..., B_{n}$ &  Bayesian networks. \\
	Capital letters e.g. $X,Y,Z$ & random variables.\\
	Lowercase letters e.g. $x,y,z$ & values attained by random variables. \\
	$p(X=x)$ & the probability that the random variable $X$ takes on \\ 
		& the value $x$. \\ 
	$p(X=x,Y=y)$ & the probability of $X=x$ and $Y=y$ simultaneously. \\
	$p(X=x|Y=y)$ & the probability that $X=x$ given the condition $Y=y$. \\
	$p(X)$ & the joint probability function of possible values \\ 
		&attained by $X$. \\
	$\alpha(X)$ & the set of ancestors of a vertex $X$ in a graph $G$.\\
	$\prod\nolimits_{X}^{G}$ & the set of parents of a vertex $X$ in a graph $G$.\\
	$q = p(X_{0}, ...X_{m}|Y_{0}, ..., Y_{n})$ & a query with targets $X_{i}$ and conditions $Y_{j}$. \\
	$\Delta(G,q)$ & the set of vertices involved in a query $q$ on a \\ 
		& graph $G$.\\
	$\Delta*$ & $\argmin_{G' \in [G]} (|\Delta(G',q)|)$. \\
	$[G]$ & the Markov equivalence class of $G$. \\ 
	MEC &  Markov Equivalence Class. \\ 
	$G \approx_{M} G'$ & $G$ is Markov equivalent to $G'$.\\
	$G_{*}$ & essential graph of $G$. \\ 
	$Q=\{q_{1}, q_{2}, ..., q_{n}\}$ & a sequence of queries. \\
	$d_{q}^{G}(q_{1},q_{2})$ & the query distance of $q_{1}$ and $q_{2}$ with respect to $G$. \\
	$G_{Q}$ & a Query Distance graph determined over \\ 
		& the MEC $[G]$. \\

 \end{tabular}
 \end{spacing}
 \end{center}
 \end{table}
\newpage


\newgeometry{
    left=40mm,
    right=20mm,
    top=30mm,
    bottom=30mm
}

% table of contents
\newpage
\tableofcontents

\clearpage
\pagenumbering{arabic}

%%%%%%%%%%%%%%% CHAPTER 1 INTRODUCTION
% introduction
\newpage
\chapter{Introduction}

%%%%%%%%%%%%%%%%%%
\section{Motivation}
%%%%%%%%%%%%%%%%%%
\null \quad \quad Many fields rely on graphical models as a framework for drawing inference on observational data, as they allow one to encode multivariate probability distributions. Inference on such models typically involves answering inference queries over a Bayesian network, that is, a directed acyclic graph (DAG) paired with an associated joint probability distribution. Computing inference queries is the process of reducing one probability distribution (typically via \textit{conditioning} and \textit{marginalizing}) into another simplified probability distribution from which we can draw samples, yielding a most probable configuration of the random variables involved. \newline
\null \quad \quad For instance, in the medical field, we may have observable data about a large number of patient symptoms (such as headache, cough, fever) and unobservable diseases (called \textit{latent} variables). We may wish to predict whether the patient has a specific disease or not given their observable symptoms. In this scenario, one would represent the data as a Bayesian network and query the model for the likelihood that a patient has the disease given the observable data. This amounts to evaluating the Bayesian network having observed the patient's symptoms, and sampling the resulting reduced probability distribution. This serves as the inference process, and allows for informed decision making. \newline 
\null \quad \quad The models discussed in this thesis have significant applications and adaptions in medicine~\cite{greenland1, greenland2, miller, pauker, shwe}, behavioral and social sciences~\cite{morgan, blalock}, and statistics~\cite{cox, lauritzen}. Furthermore, significant theoretical research has been done on graphical models themselves~\cite{verma, chickering, andersson}. An extensive survey of their genesis, development, and applications can be found in~\cite{pearlsurvey}. \newline
\null \quad \quad Naturally, especially as the use of statistical inference becomes increasingly widespread, it is in our best interest to explore methods for modeling and querying which allow us to do inference as quickly and efficiently as possible. In general, inference is a very expensive process; while there are state-of-the-art algorithms which approximate inference (such as MCMC-methods or variational inference~\cite{NUTS, varinf}), the speed of both these approximations and actual inference are limited. This thesis therefore take advantage of the Bayesian networks' property of non-identifiability, meaning that there can be several networks which encode the same probability distribution. Such networks are called \textit{Markov equivalent}. Markov equivalence informs us that a single inference query evaluated over structurally distinguishable Markov equivalent networks will always yield in statistically indistinguishable results. \newline 
\null \quad \quad We therefore explore the following question: given that the same probability distribution can be encoded in different Bayesian networks, \textbf{can it be computationally beneficial to change the representation of the probability distribution in order to answer a query more efficiently?} Additionally, can we use the same reasoning to answer a sequence of queries (for example, checking for multiple diseases at once) more efficiently?  \newline
\null \quad \quad Furthermore, in certain settings such as sampling, one does not evaluate an inference query over a probability distribution only once, but may instead wish to evaluate hundreds if not thousands of queries. Repeated evaluation of queries is a significant motivation for this work, as further justifies the overhead costs of transforming the networks.  \newline


%%%%%%%%%%%%%%%%%%
\section{Background}
%%%%%%%%%%%%%%%%%%
\null \quad \quad The family of models explored in this thesis are Bayesian networks over discrete probability distributions. The purpose of Bayesian networks is to encode information about vectors of random variables, namely the interdependencies satisfied by the vector's joint probability function. For example, consider a vector of three random binary variables $(X,Y,Z)$. Suppose that the values of $X$ and $Y$ depend on $Z$, but $X$ and $Y$ are independent from one another given $Z$. These dependencies can be encoded in the graph in Figure~\ref{fig:simpleencoding}, which is specified by its joint probability function.\newline
\begin{figure}[h!]
\begin{center}
\begin{tikzpicture}
\Vertex[x=-4.5, size=.5, color=white, label=$X$]{x}
\Vertex[x = -2.5, size=.5, color=white, label=$Y$]{y}
\Vertex[y = 1.5, x=-3.5 ,size=.5,color=white, label=$Z$]{z}
\Edge[Direct, distance=0.5](z)(x)
\Edge[Direct, distance=0.5](z)(y)
\Text[x=-3.5, y=-1]{$p(X,Y|Z) = p(X|Z) \cdot p(Y|Z)$}
\end{tikzpicture}
\end{center}
\caption{}
\label{fig:simpleencoding}
\end{figure}
\null \quad \quad In the case of the directed models we explore in this thesis, there is not necessarily a unique graph which satisfies the dependency constraints of the joint probability function. A random vector and its constraints may be represented by a variety of different graph structures which ultimately describe the same underlying probability distribution. To understand how the complexity of answering a query depends on graph structure, consider the two models in Figure~\ref{fig:introcomparison}.\newline
\begin{figure}[h!]
\begin{center}
\begin{tikzpicture}
\Vertex[x=-3.5, y=3.5, color=white, label=$W$]{w}
\Vertex[x=-4.5, size=.5, color=white, label=$X$]{x}
\Vertex[x = -2.5, size=.5, color=white, label=$Y$]{y}
\Vertex[y = 1.5, x=-3.5 ,size=.5,color=white, label=$Z$]{z}
\Edge[Direct, distance=0.5](z)(x)
\Edge[Direct, distance=0.5](z)(y)
\Edge[Direct, distance=0.5](w)(z)
\Text[x=-3.5, y=-1]{$G_{1}$}
\node [below] at (-3.5, -2) {$\begin{array}{c} p(X,Y,Z,W) = \\  p(X|Z) \cdot p(Y|Z) \cdot p(Z|W) \cdot p(W)  \end{array}$};

\Vertex[x=4.5, y=3.5, color=white, label=$W$]{w}
\Vertex[x=3.5, size=.5, color=white, label=$X$]{x}
\Vertex[x = 5.5, size=.5, color=white, label=$Y$]{y}
\Vertex[y = 1.5, x=4.5 ,size=.5,color=white, label=$Z$]{z}
\Edge[Direct, distance=0.5](z)(x)
\Edge[Direct, distance=0.5](z)(y)
\Edge[Direct, distance=0.5](z)(w)
\Text[x=4.5, y= -1]{$G_{2}$}
\node [below] at (4.5, -2) {$\begin{array}{c} p(X,Y,Z,W) = \\  p(X|Z) \cdot p(Y|Z) \cdot p(W|Z) \cdot p(Z) \end{array}$};
\end{tikzpicture}
\end{center}
\caption{}
\label{fig:introcomparison}
\end{figure}
\null \quad \quad If we wish to answer a query about the random variable $Y$ in $G_{1}$, for instance the probability that $Y$ attains the value $y$, notated $p(Y=y)$, we must also compute the probabilities of each vertex it depends on either directly or indirectly: in this case, all of its \textbf{ancestors} (namely $Z$ and $W$) in addition to $Y$ itself. Alternatively, if we wish to answer a query $p(Y=y)$ on the graph $G_{2}$, we need only to compute $Z$ and $Y$. Then the number of variables which must be computed is reduced when we consider $G_{2}$ instead of $G_{1}$. The set of vertices a query depends on can somewhat vary depending on the query, but the motivation for transforming a network remains the same. \newline
\newpage
\null \quad \quad In a context where answering queries about a single vertex may require us to compute arbitrarily many other vertices in the graph, we seek an equivalent graphical representation of our probability distribution which minimizes the number of vertices that must be computed when answering a query. Further, we wish to explore the optimal sequence of transformations allowing us to answer a sequence of queries most efficiently. 

%%%%%%%%%%%%%%%%%%
\section{Goal}
%%%%%%%%%%%%%%%%%%
\null \quad \quad The central concept of this thesis is the exploitation of non-identifiability of Bayesian networks (Markov equivalence~\cite{verma}) for faster inference. Given that multiple Bayesian networks may encode the same probability distribution, the goals of our research are as follows: \newline
\null \quad \quad Given an inference query $q$ and a Bayesian network $B$ (which represents a single factorization of the probability distribution we wish to query), find a Markov equivalent Bayesian network $B'$ which minimizes the number of variables that must be evaluated to answer the query. The purpose of the minimization is to reduce the cost of evaluating $q$ in order to achieve faster inference, as well as evaluate the gained speedup versus the cost of transforming the representation. \newline
\null \quad \quad To achieve this, we first define the set of nodes which must be evaluated to answer $q$ over a given network $B$, notated $\Delta(B,q)$. Then, we search for a Markov equivalent Bayesian network $B'$ such that the size of $\Delta(B',q)$ is minimized. Finally, we compare the costs of computing the query over $B$ to the costs of computing the query over the minimized graph $B'$ plus the costs of identifying and transforming to $B'$. In short, the minimization problem is to determine the network $\Delta^{*}$ defined as
	$$\Delta^{*} = argmin_{B' \in [B]}|\Delta(B',q)|.$$
\null \quad \quad Subsequently, we consider how to identify optimal transformations for a known sequence of queries $\{q_{0}, q_{1}, ... q_{n}\}$ such that the overhead of transformation is most beneficial compared to the gained speedups by transforming.
%%%%%%%%%%%%%%%%%

\newpage
\section{Motivating Example}
%%%%%%%%%%%%%%%%%
\null \quad \quad The following example demonstrates how answering a query on one graph can be sped up by asking the same query over a Markov equivalent graph with a different structure. Consider the DAG in Figure~\ref{fig:motivatingdag}

\begin{figure}[h!]
\begin{center}
\begin{tikzpicture}
\Vertex[x=0, y=0, size=1,color=white, label=$X$]{x}
\Vertex[x=3.5,y=0, size=1, color=white, label=$Y$]{y}
\Vertex[x = 7, y=0, size=1, color=white, label=$Z$]{z}
\Edge[Direct, distance=0.5](x)(y)
\Edge[Direct, distance=0.5](y)(z)
\Text[x=3.5, y=-1.3]{$G_{1}$}
\end{tikzpicture}
\end{center}
\caption{}
\label{fig:motivatingdag}
\end{figure}

with conditional probabilities given by
\begin{align*}
	& p(X=1) = 0.2 			& \\
	& p(Y = 1 | X = 0) = 0.3		& p(Y = 1 | X = 1) = 0.4 \\ 
	& p(Z = 1 | Y = 0) = 0.1		& p(Z = 1 | Y = 1) = 0.9 
\end{align*}
and the joint probability
$$ p(X,Y,Z) = p(X) \cdot  p(Y|X) \cdot  p(Z|Y)$$
\null \quad \quad Here, if we wish to answer the query $p(Z|Y)$, our computation relies on all three random variables, since $Z$ relies on $Y$ and $Y$ relies on $X$. However, we can find a Markov equivalent graph in which we can answer the same query while relying on fewer random variables. Consider the DAG in Figure~\ref{fig:introdag}. 

\begin{figure}[h!]
\begin{center}
\begin{tikzpicture}
\Vertex[x=0, y=0, size=1,color=white, label=$X$]{x}
\Vertex[x=3.5,y=0, size=1, color=white, label=$Y$]{y}
\Vertex[x = 7, y=0, size=1, color=white, label=$Z$]{z}
\Edge[Direct, distance=0.5](y)(x)
\Edge[Direct, distance=0.5](y)(z)
\Text[x=3.5, y=-1.3]{$G_{2}$}
\end{tikzpicture}
\end{center}
\caption{}
\label{fig:introdag}
\end{figure}
\null \quad \quad We compute the conditional probabilities of $G_{2}$ and claim that the model can equivalently describe our joint probability distribution from $G_{1}$, and therefore be used to answer the query $p(Z|Y)$. We then observe how the structure of the new model affects our computation of the query. By applying Bayes' Theorem, which states that $p(A|B) = \frac{p(B|A)p(A)}{p(B)}$, we can compute
$$p(X|Y) = \frac{p(Y|X) \cdot p(X)}{p(Y)}  =  \frac{ p(Y|X) \cdot p(X) }{ \sum_{X} p(Y|X) \cdot p(X) }.$$

Using our pre-defined conditional probabilities,
\begin{align*}
	p(Y = 0) 	& \ \ = \ \ \sum_{X} p(Y = 0|X) \cdot p(X) \\ 
			& \ \ = \ \ 0.7 \cdot 0.8 + 0.6 \cdot 0.2 \\
			& \ \ = \ \ 0.56 + 0.12 \\
			& \ \ = \ \ 0.68
\end{align*}
and 
\begin{align*}
	p(Y = 1) 	& \ \ = \ \ \sum_{X} p(Y = 1|X) \cdot p(X) \\ 
			& \ \ = \ \ 0.3 \cdot 0.8 + 0.4 \cdot 0.2 \\
			& \ \ = \ \ 0.24 + 0.08 \\
			& \ \ = \ \ 0.32.
\end{align*}

which allow us to compute 
$$
	p(X = 1 | Y = 0) \ \ = \ \ \frac{p(Y = 0 | X=1) \cdot p(X=1)}{p(Y=0)} \ \ = \ \ \frac{0.6 \cdot 0.2}{0.68} \ \ \approx 0.176
$$

and
$$
	p(X = 1 | Y = 1) \ \  = \ \  \frac{p(Y=1 | X=1) \cdot p(X=1)}{p(Y=1)} \ \ = \ \ \frac{0.4 \cdot 0.2}{0.32} \ \ = 0.25.
$$

\null \quad \quad Then, since $X$ and $Y$ do not depend on $Z$, the conditional probability of $Z$ remains unchanged. Therefore, model $G_{2}$ is an equivalent model to $G_{1}$. If we answer the query $p(Z|Y)$ on model $G_{2}$, we do not need to consider $X$, since $Y$ (and consequently $Z$) no longer depends on $X$. \\
\null \quad \quad Through this example, we see that it may be possible to find an equivalent graph to our original model, and to use the new graph structure to answer certain queries more efficiently. It is important to note that the efficiency of the new model for answering a query depends on the content of our query (as opposed to an arbitrary query). 


%%%%%%%%%%%%%%%%%%
\section{Related work}
%%%%%%%%%%%%%%%%%%
\null \quad \quad Significant work has been done on individual aspects of the subject, though this thesis pioneers in using them together for fast inference. Verma and Pearl\cite{verma} give a framework for understanding how the same probability distribution can be encoded into two distinct graphs and present a simplified criterion for when two graphs describe indistinguishable distributions. Lucas and Flesch\cite{flesch} present a survey-style overview of Markov Equivalent graph structures, reduction of graph structures to simplified representations of equivalent distributions, as well as providing a strong background of relevant graph theory and probability preliminaries. Chickering\cite{chickering} explores parameters of equivalent networks, presents a procedure for altering graphs without changing the described probability distribution, and quantifies the complexity of several such manipulations. Andersson \textit{et. al.}\cite{andersson} make similar headway by exploring reductions of graphs to representative form, as well as describing procedures by which one can alter the structure. \newline 
\null \quad \quad Adjacently, Schachter\cite{schachter} confronts the same goal of faster inference of queries using a different method in a modified setting. His task focuses on eliminating unnecessary nodes from the computations directly, rather than by reversing edges. 

%%%%%%%%%%%%%%%%%%
\section{Outline}
%%%%%%%%%%%%%%%%%%
\null \quad \quad In Chapter 2, we introduce preliminary concepts from graph theory and Bayesian statistics to robustly define our models and to aid our understanding of graph manipulations and the process of querying over a graph. Chapter 3 details the problem and proposed querying method, then aims to quantify the computational costs of answering an arbitrary query over a graph by this method. Chapter 4 builds context for understanding Markov equivalence between graphs. In Chapter 5, we examine the circumstances under which Markov equivalence can be utilized to answer a query more efficiently, and then quantify the computational costs of finding a suitable Markov equivalent graph for faster inference. Finally, in Chapter 6, we compare the quantities explored in Chapters 3 and 5 to determine whether our proposed algorithm indeed increases inference speed, and if so, where it is effective. Additionally, we explore the usefulness of graph transformation in the context of sequences of queries rather than an individual query. Finally, in Chapter 7, we present the conclusions, implications, and possible future work. 

%%%%%%%%%%%%%%% CHAPTER 2 PRELIMINARIES
% graph theory basics
\newpage
\chapter{Preliminaries}

\null \quad \quad In this chapter we introduce preliminary concepts from graph theory and Bayesian statistics. These concepts will allow us to robustly define our models and problem statements, give us critical background for evaluating queries and exploring Markov Equivalence, and serve as a basis for the content of the rest of the thesis.

\section{Foundations of Graph Theory}

%-------------------------graph defn
\begin{definition}[Graph (directed, undirected, mixed)]
A \textbf{graph} is defined as a pair $G = (V,E)$ in which $V$ is a finite set of vertices, and $E \subset V \times V$ is a finite set of edges. Vertices (also sometimes called \textbf{nodes}) will generally be denoted by capital letters, \textit{i.e.} $X$, $Y$, $Z$. $G$ is called \textbf{undirected} if every edge in $G$ is undirected, meaning that for each edge $(X,Y) \in E$ (denoting an edge from vertex $X$ to vertex $Y$), the edge $(Y,X)$ is also in $E$, for $X \neq Y$. Otherwise, $G$ is called \textbf{directed}. We denote such undirected edges by $\{X,Y\}$. A \textbf{mixed graph} (also called a hybrid graph) is a graph $G$ which contains both directed and undirected edges.
\end{definition}

\begin{figure}[h!]
\begin{center}
\begin{tikzpicture}
\Vertex[x = .5, y=0, size=.5, color=white]{x2}
\Vertex[x=.5, y=2 ,size=.5,color=white]{x3} 
\Vertex[x = -1.2, y=1, size=.5, color=white]{x4}
\Edge[Direct, distance=0.5](x4)(x2)
\Edge[Direct, distance=0.5](x4)(x3)
\Edge[Direct, distance=0.5](x2)(x3)
\Text[y=-1, x=-.3]{directed} 

\Vertex[x = 4.5, y=0, size=.5, color=white]{x2}
\Vertex[x=4.5, y=2 ,size=.5,color=white]{x3} 
\Vertex[x = 2.8, y=1, size=.5, color=white]{x4}
\Edge[Direct, distance=0.5](x4)(x2)
\Edge[distance=0.5](x4)(x3)
\Edge[distance=0.5](x2)(x3)
\Text[y=-1, x=3.7]{mixed} 

\Vertex[x = 8.5, y=0, size=.5, color=white]{x2}
\Vertex[x=8.5, y=2 ,size=.5,color=white]{x3} 
\Vertex[x = 6.8, y=1, size=.5, color=white]{x4}
\Edge[distance=0.5](x4)(x2)
\Edge[distance=0.5](x4)(x3)
\Edge[distance=0.5](x2)(x3)
\Text[y=-1, x=7.7]{undirected} 

\end{tikzpicture}
\end{center}
\caption{}
\end{figure}
%------------------------\-graph defn

\begin{definition}[Neighbor, adjacent] Two vertices $X, Y \in V$ in a graph $G = (V,E)$ are called \textbf{adjacent} if there is an edge between them, either $(X,Y)$, $(Y,X)$, or ${X,Y}$. $X$ and $Y$ are also called \textbf{neighbors} in this scenario. 
\end{definition}

\begin{definition}[Route]
A \textbf{route} in a graph $G = (V,E)$ is a sequence $X_{1}, X_{2}, ..., X_{k}$ of vertices in $V$ such that there is an edge connecting $X_{i}$ to $X_{i+1}$ (independent of the direction of the edge), for $i = 1, ..., k-1, k \geq 1$. The integer $k$ is the length of the route.
\end{definition}

\begin{definition}[Path, directed cycle]\label{def:directedcycle}
A (directed) \textbf{path} in a directed graph $G=(V,E)$ is a route where vertices $X_{i}$ and $X_{i+1}$ are connected by a directed edge $(X_{i},X_{i+1})$. A directed path which begins and ends at the same vertex is called a \textbf{directed cycle}. 
\end{definition}

\begin{remark}
The distinction between a path and a route will be significant in later chapters. One should retain that a path is a directed sequence while a route is undirected. 
\end{remark}

\begin{definition}[Parent, ancestor]
Given two vertices $X,Y \in V$ in a directed graph $G = (V,E)$, $Y$ is called a \textbf{parent} of $X$ if there is an edge $(Y,X) \in E$. The set of parents of a vertex $X$ in a graph is denoted $\prod\nolimits_{X}^{G}$. Likewise, $X$ is called a \textbf{child} of $Y$. The set of \textbf{ancestors} of $X$, denoted $\alpha(X)$, is the set of all vertices $Y$ such that there exists a directed path from $Y$ to $X$, but no path from $X$ to $Y$. In the context of this thesis, this is the set of vertices upon which a vertex $X$ depends, either directly or indirectly. Likewise, if $Y$ is an ancestor of $X$, then $X$ is a \textbf{descendant} of $Y$.
\end{definition}

\begin{center}
\begin{figure}[h!]
\centering
\begin{tikzpicture}
\Vertex[x=0, y=2, size=.4, color=red, shape=rectangle]{x1} %parent higher
\Vertex[x = 3.5, y=0, size=.4, color=blue, shape=rectangle]{x2} %child lower
\Vertex[x=3.5, y=2 ,size=.4,color=blue, shape=rectangle]{x3}  %child higher 
\Vertex[x = 1.8, y=1, size=.5, color=gray]{x4} %starting node
\Vertex[x = 0, y=0, size=.4, color=red, shape=rectangle]{x5} %single branch
\Vertex[x = -1.8, y=1, size=.5, color=red]{x6} % lower granparent
\Vertex[x = -1.8, y=3, size=.5, color=red]{x7} %top left grandparent
\Vertex[x = 5.3, y=1, size=.5, color=blue]{x8} %grandchild lower
\Vertex[x = 5.3, y=-1, size=.5, color=blue]{x9} %grandchild higher
\Vertex[x = 1.8, y=-1, size=.5, color=white]{xa} %free child 

\Edge[Direct, distance=0.5](x1)(x4)
\Edge[Direct, distance=0.5](x4)(x2)
\Edge[Direct, distance=0.5](x4)(x3)
\Edge[Direct, distance=0.5](x2)(x3)

\Edge[Direct, distance=0.5](x5)(x4)
\Edge[Direct, distance=0.5](x6)(x1)
\Edge[Direct, distance=0.5](x7)(x1)
\Edge[Direct, distance=0.5](x7)(x6)
\Edge[Direct, distance=0.5](x5)(xa)
\Edge[Direct, distance=0.5](x2)(x8)
\Edge[Direct, distance=0.5](x2)(x9)

\end{tikzpicture}
\caption{Red vertices are ancestors of the gray vertex (squares denoting parents), and blue vertices are descendants of it (squares denoting children). The white vertex is neither an ancestor nor a descendant of the gray vertex.}
\end{figure}
\end{center}


\begin{remark}
$\prod\nolimits_{X}^{G} \subseteq \alpha(X)$. Likewise, the set of children of $X$ is a subset of the set of descendants of $X$. 
\end{remark}

%----------------------------------- chain graph defn
\begin{definition}[Chain graph, directed acyclic graph (DAG)]
A mixed graph $G = (V,E)$ is called a \textbf{chain graph} if it contains no directed cycles. A \textbf{directed acyclic graph} is a chain graph which is directed. This is abbreviated as \textit{DAG}.
\end{definition}

\begin{figure}[h!]
\begin{center}
\begin{tikzpicture}
\Vertex[x=0, y=1, size=.5, color=white]{x1}
\Vertex[x = 3.5, size=.5, color=white]{x2}
\Vertex[x=3.5, y = 2 ,size=.5,color=white]{x3}
\Vertex[x = 1.8, y=1, size=.5, color=white]{x4}
\Edge[Direct, distance=0.5](x1)(x4)
\Edge[Direct, distance=0.5,color=red](x3)(x2)
\Edge[Direct, distance=0.5,color=red](x2)(x4)
\Edge[Direct, distance=0.5,color=red](x4)(x3)
\Text[y=-.5, x=1.8]{not a DAG}


\Vertex[x=6.5, y=1, size=.5, color=white]{x1}
\Vertex[x = 10, size=.5, color=white]{x2}
\Vertex[x=10, y = 2 ,size=.5,color=white]{x3}
\Vertex[x = 8.3, y=1, size=.5, color=white]{x4}
\Edge[Direct, distance=0.5](x1)(x4)
\Edge[Direct, distance=0.5](x4)(x2)
\Edge[Direct, distance=0.5](x4)(x3)
\Edge[Direct, distance=0.5](x3)(x2)
\Text[y=-.5, x=8.3]{DAG}

\end{tikzpicture}
\end{center}
\caption{A directed cycle is shown in red.}
\end{figure}

\begin{remark}
Every undirected graph is a chain graph.
\end{remark}

\begin{definition}[Tree, Spanning tree]\cite{tree}
A \textbf{tree} is an unweighted, undirected connected acyclic graph. A \textbf{spanning tree} $T$ of a DAG $G=(V,E)$ is a subgraph of $G$ which is a tree and connects all vertices in $G$.
\end{definition}

%--------------------------------------------\chain graph defn



% bayesian statistics basics
\section{Foundations of Bayesian Statistics}

\begin{definition}[Bayesian Network ~\cite{stephenson}]
A \textbf{Bayesian network} is a pair $B = (G,P)$ where $G=(V,E)$ is a DAG and $P$ is a joint probability distribution defined on a set of random variables $\mathcal{X}$. That is, it is a DAG and an associated joint probability distribution. \newline
\null \quad \quad Vertices in $G$ have a one-to-one correspondence to the random variables in $\mathcal{X}$ associated with $P$, meaning the set of vertices V represents both the vertices of $G$ and the random variables of the joint probability distribution $P$. \newline
\null \quad \quad $P$ factorizes on the sets $\{X \cup \prod\nolimits_{X}^{G}|X\in G\}$. That is, for each $X \in G$ there is a factor in $P$ that depends on $X$ and $\prod\nolimits_{X}^{G}$, namely $P_{X}(X|\prod\nolimits_{X}^{G})$.
\end{definition}

\begin{example}[Simple Bayesian Network] The DAG in Figure ~\ref{fig:bayesiannet1} paired with the joint probability distribution $p(X_{1}=x_{1},X_{2}=x_{2}) = p(X_{1}=x_{1})\cdot p(X_{2}=x_{2}|X_{1}=x_{1})$ is a Bayesian network. 

\begin{figure}[h!]
\centering
\begin{tikzpicture}
\Vertex[size=.5, color=white, label=$X$]{x1}
\Vertex[x = 2, size=.5, color=white, label=$Y$]{x2}
\Edge[Direct, distance=0.5](x1)(x2)
\end{tikzpicture}
\caption{}
\label{fig:bayesiannet1}
\end{figure}
\end{example}

\begin{remark}
In the context of this thesis, we only consider graphs $G$ which are finite DAGs (meaning $|V| < \infty)$ unless explicitly stated otherwise. Further, for our purposes, all such DAGs are directed graphical models with associated probability distributions. Therefore we will refer to DAGs $G=(V,E)$ and Bayesian networks interchangeably, with necessary details clarified in context. 
\end{remark}

\begin{definition}[Sample space, event~\cite{snell}]
Given a random experiment (an experiment whose outcome depends on chance), the \textbf{sample space} of the experiment is the set of all possible outcomes. Each subset of a sample space is called an \textbf{event}, that is, an outcome or a collection of outcomes of the random experiment. Therefore, event $\subseteq$ sample space. 
\end{definition}



\begin{example}
 In the context of Bayesian networks, we are interested in possible values of random variables which are represented by vertices. As a simple example, suppose we have the random binary variables $X$ and $Y$. Then the sample space of $X$ is $\{0,1\}$, and likewise for $Y$.  Generally, we will be interested in answering questions such as the probability that the random variable $X$ takes on a value $x \in \{0,1\}$; these queries are written $p(X=0)$, $p(X=1)$, or more generally, $p(X=x)$. Some possible queries one might encounter include the following:
\begin{itemize}
\item $p(X=x,Y=y)$, the probability that $X=x$ and $Y=y$ simultaneously, computed $p(X=x)\cdot p(Y=y)$,
\item $p(X=x|Y=y)$, the probability that $X=x$ given that we have observed the condition $Y=y$.
\end{itemize}

Let the marginal probabilities of these random vectors be given as follows:
\begin{align*}
&p(X=1) = \frac{3}{10} 	& p(X=0) = \frac{7}{10} \\
&p(Y = 0) = \frac{6}{10} 	&p(Y=1) = \frac{4}{10}
\end{align*}


\begin{table}[h!]
  \begin{center}
    \begin{tabular}{  c | c c c c}
	$X$ 		& 1 & 1 & 0 & 0 \\
	$Y$		& 1 & 0 & 1 & 0 \\
	\hline \\[-1em]
	$p(X,Y)$	&$ \frac{3}{25}$ & $\frac{21}{100}$ & $\frac{7}{25}$ & $\frac{21}{50}$
    \end{tabular}
  \end{center}
  \caption{The joint probability function $p(X,Y)$.}
  \label{jprobtable}
  \end{table}

where for $X$ and $Y$ binary variables, $p(X=1) = 1-p(X=0)$ since $\sum_{x\in\{0,1\}}p(X=x) = 1$. This joint probability function $p:\{0,1\} \times \{0,1\} \rightarrow [0,1]$ can also be encoded as a table, as in Table \ref{jprobtable}.\newline
\null \quad \quad This means that the joint probability function $p(X,Y)$ is not a value, but a function of the likelihood of potential values from the sample space of $Z$ and $Y$. Then, we can answer the query $p(X=1,Y=0)$ by reading off the table: $\frac{3}{25}$. \newline
\null \quad \quad Next, suppose we have observed the condition $Y=0$. Then, we can \textit{condition} the distribution, meaning we update the probability function such that it reflects our observation. We do so by looking at the scenarios in which $Y=0$:
\begin{align*}
	&p(X=0|Y=0) = \frac{21}{50}	& \text{and} &p(X=1|Y=0) = \frac{21}{100}.
\end{align*}
Then, using the fact that the probabilities should sum up to $1$,  we normalize:
\begin{align*}
p(X=0|Y=0) & \ \ = \ \ \frac{p(X=0|Y=0)}{p(X=0,Y=0) + p(X=1,Y=0)} \\[2em]
& \ \ = \ \ \frac{p(X=1,Y=0)}{p(Y=0)},
\end{align*}
and
\begin{align*}
p(X=1|Y=0) & \ \ = \ \ \frac{p(X=1,Y=0)}{p(X=0,Y=0) + p(X=1,Y=0)} \\[2em]
& \ \ = \ \ \frac{p(X=1,Y=0)}{p(Y=0)}.
\end{align*}
Thus, with $p(Y=0) = \frac{63}{100}$, these give us

\begin{table}[h!]
  \begin{center}
    \begin{tabular}{  c | c c c c }
	$X|Y=0$ 		& 1 & 0  \\
		\hline \\[-1em]
	$p(X|Y=0)$		& $\frac{2}{3}$  & $\frac{1}{3}$  \\
    \end{tabular}
  \end{center}
  \end{table}
\end{example}

\begin{definition}[(Probabilistic) query]
A probabilistic query $q$ on a graph $G=(V,E)$ is an inference question of the form $p(X_{1}, ..., X_{m}|Y_{1}, ..., Y_{n})$ where $X_{i}, i \in [0,m] \in V$ and $Y_{j}, j \in [0,n] \in V.$ $X_{i}$ are the \textbf{targets} of $q$ while $Y_{j}$ are the \textbf{conditions} or \textbf{observations}. 
\end{definition}

\begin{theorem}[Bayes' Theorem~\cite{snell}] Given two random variables $X$ and $Y$ with $p(Y) \neq 0$
$$p(X|Y) = \frac{p(Y|X) \cdot p(X)}{p(Y)}.$$
\end{theorem}

\begin{remark}
Bayes' Theorem will serve as a significant basis for many of the calculations necessary for transforming graphs in later chapters. 
\end{remark}

\begin{definition}[(Statistical) independence~\cite{snell}]\label{statisticalindependence}
Let $X$ and $Y$ be random variables. Then $X$ and $Y$ are called \textbf{independent} whenever 
$$P(X,Y) = p(X)p(Y)$$ 
or equivalently if
$$ p(X) = p(X|Y)$$
and vice versa.
\end{definition}

\begin{remark}
~\ref{statisticalindependence} can be intuitively understood as follows: learning about $Y$ has no effect on our knowledge concerning $X$ and vice versa.
\end{remark}

\begin{definition}[Conditional independence~\cite{snell}]\label{conditionalindependence} 
Let $X$, $Y$, and $Z$ be random variables. Then $X$ is said to be \textbf{conditionally independent} of $Y$ given $Z$ iff $p(X|Y,Z) =p(X|Z)$.
Otherwise, they are said to be \textbf{conditionally dependent}. 
\end{definition}

\begin{remark}
Definition~\ref{conditionalindependence} can be intuitively understood as follows: if we know about $Z$, then learning about $Y$ has no effect on our expectation concerning $X$. Likewise, learning about $X$ has no effect on our expectation concerning $Y$ if we know about $Z$. 
\end{remark} 

\begin{remark} Definition ~\ref{statisticalindependence} and Definition ~\ref{conditionalindependence} also hold for disjoint sets $A,B$, and $C \subset V$ with $p$ a joint probability distribution defined on $V$. 
\end{remark}

\begin{example} Suppose Evin and Lisa take turns flipping a single coin which either results in heads ($H$) or tails ($T$). Naturally, one would assume that the probability of Lisa flipping heads is independent of the probability that Evin flips heads, that is, $$p(Lisa = H|Evin = H) = p(Lisa = H),$$ since flipping a fair coin once does not dictate the outcome of a second flip. This means the two experiments are independent.\newline
\null \quad \quad Now suppose we learn that the the coin is biased, calling the presence of the bias event $Z$. If the first coin flip results in heads, we would guess that the coin is more likely biased toward heads, and expect that the second coin flip to be heads. That is, the event of Evin flipping heads gives us information about the likelihood that Lisa will flip heads, given that we have observed $Z$. Then, whereas $p(Evin=H)$ and $p(Lisa=H)$ were previously independent from one another, observing event $Z$ makes them depend on one another. Therefore the two events are conditionally dependent given $Z$:
$$p(Lisa=H|Z,Evin=H) \neq p(Lisa=H|Evin=H).$$ 

\begin{center}
\begin{figure}[h!]
\begin{center}
\centering
\includegraphics[width=0.8\textwidth]{cointoss.jpeg}
\caption{Evin and Lisa engaging in a game of chance.}
\end{center}
\end{figure}
\end{center}

\end{example} 
\null \quad \quad Dependence, independence, and conditional independence have an intuitive manifestation within graph structure, which follows from Definition ~\ref{statisticalindependence}: \newline
\null \quad \quad Given a graph $G=(V,E)$ such as any of the graphs in Figure~\ref{fig:ind}, if two vertices $X, Y \in V$ have an edge between them, either $(X,Y)$ or $(Y,X)$, then they are dependent variables by construction of our graphs. \newline
\null \quad \quad If there exists a route between two vertices $X$ and $Y$, but $X$ and $Y$ are not adjacent, then they are conditionally independent, where the conditions are the vertices on that route. This is because information about $X$ can give us information about an intermediary node on the route and allow us to gain information about $Y$ (or vice versa). \newline
\null \quad \quad Finally, if two vertices are not connected by a route (disconnected) then they are independent, since information about $X$ cannot give us information about $Y$ under any circumstances. \\[2em]

\begin{figure}[h!]
\centering

\begin{center}
\begin{tikzpicture}
\Vertex[x=-4.5, y=-.7, size=.5, color=white, label=$X$]{x1}
\Vertex[x = -2.5, y=-.7, size=.5, color=white, label=$Y$]{x2}
\Edge[Direct, distance=0.5](x1)(x2)

\Vertex[size=.5, color=white, label=$X$]{x1}
\Vertex[x = 2, size=.5, color=white, label=$Y$]{x2}
\Vertex[y = -1.5, x=1 ,size=.5,color=white]{x3}
\Edge[Direct, distance=0.5](x1)(x3)
\Edge[Direct, distance=0.5](x3)(x2)

\Vertex[x = 3,size=.5, color=white, label=$X$]{x1}
\Vertex[x = 5, size=.5, color=white, label=$Y$]{x2}
\Vertex[y = -1.5, x=3, size=.5,color=white]{x3}
\Vertex[y = -1.5, x=5, size=.5,color=white]{x4}
\Edge[Direct, distance=0.5](x1)(x3)
\Edge[Direct, distance=0.5](x3)(x4)
\Edge[Direct, distance=0.5](x2)(x4)

\Vertex[x = 7.3, y=-.7, size=.5, color=white, label=$X$]{x1}
\Vertex[x = 8.7, y =-.7, size=.5, color=white, label=$Y$]{x2}


\Text[x=-3.5 ,y=-3]{dependent}

\Text[x=2.5 ,y=-3]{$X$ and $Y$ conditionally independent}

\Text[x=8 ,y=-3]{independent}

\end{tikzpicture}
\end{center}

\caption{}
\label{fig:ind}
\end{figure}




%%%%%%%%%%%%%%% CHAPTER 3 QUERYING
% General querying
\newpage
\chapter{Querying}

%\begin{itemize}
%\item Details of the problem statement
%\item Querying method without exploiting MEC (Depth first or breadth first search of dependent variables, ie ancestors) 
%\item Speed of this querying method. 
%\end{itemize}

\section{Outline of the Task}

%\null \quad \quad Given a Bayesian network $B$ and a query, our goal is to find a MEC equivalent graph $B'$ such that the number of nodes which must be computed is minimized for the same query over $B'$. By minimizing the number of nodes which must be computed, we can speed up the querying process to achieve faster inference. \newline


\null \quad \quad The first step toward our goal is to determine the set of random variables which must be involved in the computation to answer a given query over $G$. This will serve as a quantity we minimize in later chapters; the minimization will be to reduce the size of the set, meaning to find a Markov equivalent graph $G'$ to $G$ such that this number of vertices involved is minimized. \newline
\null \quad \quad Naturally, before we consider the minimization itself, we begin by identifying such a set of vertices. This chapter will focus on identifying this set of vertices which must be computed to answer a query, and given this, quantifying the computational cost of answering a query in general. 

\section{Structure of a Query}

\null \quad \quad An inference query is a question which asks the probability of an event given some observations. Queries can take advantage of multiple observations and have multiple targets. For example, one can ask for the probability that the random variable $X$ has takes on the value $x$ given that we have observed $Y = y$. Let $x$ be a value attained by the random variable $X$, and let $y$ be a value attained by the variable $Y$. This is written $p(X=x|Y=y)$. The general form of a query is as follows:

\begin{definition}[Query] Given a graph $G = (V,E)$ with $X_{1}, ...X_{m}, Y_{1}, ...Y_{n} \in V$, the general form of a query $q$ is $p(X_{1}, ..., X_{m} | Y_{1}, ... Y_{n})$, meaning the probability of variables $X_{1}, ..., X_{m}$ given that we have conditioned on observed values attained by variables $Y_{1}, ..., Y_{n}$. 
\end{definition}

\begin{remark}
Note that while a query on $G=(V,E)$ must have at least one target (otherwise there is nothing to compute), it does not necessarily need to include a condition. For example, $p(X=x)$ is a valid query for $X \in V$. 
\end{remark}

\begin{table}[h!]

  \begin{center}
  \scalebox{0.7}{ %----
    \begin{tabular}{  l | l | l | l  }
      \thead{Scenario} & \thead{Query} & \thead{Interpretation} & \thead{Example} \\
      \hline
      
       	\newline
 	\newline
      \makecell[l]{Query with a single \\ target and no \\ observations} &  \makecell[l]{$p(X=x)$} & \makecell[l]{Probability that $X=x$} & \makecell[l]{Probability that a patient \\ has lung cancer.} \\
      
 	\newline
 	\newline
      \makecell[l]{Query with a single \\ target, single \\ observation} &  \makecell[l]{$p(X=x|Z=z)$} & \makecell[l]{Probability that $X=x$ \\ given that we have \\ observed $Z=z$.} & \makecell[l]{Probability that a patient \\ has lung cancer given that \\ they are a tobacco smoker.} \\
 
      \newline
      \newline
      \makecell[l]{Query with multiple \\ targets, single \\ observation} &  \makecell[l]{$p(X=x,Y=y|Z=z)$} & \makecell[l]{Probability that $X=x$ and \\ $Y=y$ given that we have \\ observed $Z=z$.} & \makecell[l]{Probability that a patient \\ has lung cancer and high blood \\ pressure given that they are a \\ tobacco smoker.} \\
      
      \newline
      \newline
            \makecell[l]{Query with a single \\ target, multiple \\ observations} & \makecell[l]{$p(X=x|W=w,Z=z)$} & \makecell[l]{ Probability that $X=x$ \\ given that we have \\ observed $W=w$ and $Z=z$. } & \makecell[l]{Probability that a patient has \\ lung cancer given that they \\ are over 50 years of age and \\ a tobacco smoker} \\
            
     \newline
            \makecell[l]{Query with multiple \\ targets, multiple \\ observations} & \makecell[l]{$p(X=x, Y=y|W=w,Z=z)$} & \makecell[l]{ Probability that $X=x$\\ and $Y=y$  given \\ that we have observed \\ $W=w$ and $Z=z$. } & \makecell[l]{Probability that a patient has \\ lung cancer and high blood\\ pressure given that they \\ are over 50 years of age and \\ a tobacco smoker.} \\

    \end{tabular}
      } % ----
	\caption{Examples of possible queries which may be asked on a DAG $G$.}
  \end{center}
  \end{table}
  
  
\null \quad \quad Let $G$ be a DAG and $q$ a query. Define $\Delta(G, q)$ to be the set of vertices involved in the computation of $q$ on $G$. The goal of this chapter is to find this set, and in doing so, determine its size. In Chapter $5$, we will to present a method for finding $\argmin_{G' \in [G]} (|\Delta(G,q)|)$. We call this minimized graph $\Delta^{*}$.

\section{Vertices Required to Compute a Query}

\null \quad \quad In the following section, we determine which set of vertices a query $q$ on a DAG $G$ depends on, notated $\Delta(G,q)$. First we consider queries with single targets and single conditions, then move on to scenarios with multiple conditions, before finally generalizing to scenarios with multiple targets and conditions. We will see that the structure of the DAG (namely, position of targets in $G$ in relation to the position of conditions) plays a significant role in whether certain vertices must be involved in the computation of a query. \newline
\null \quad \quad Determining $\Delta(G,q)$ (and therefore $|\Delta(G,q)|$) for arbitrary $G$ and $q$ will later allow us to quantify the speedups gained by transforming the graph:  given a DAG $G$ and a Markov equivalent transformed DAG $G'$, we will compare $|\Delta(G,q)|$ to $|\Delta(G',q)|$. If $|\Delta(G',q)|$ is smaller, then one can conclude that the result of the transformation reduces the cost of answering the query. Then the question becomes whether the overhead of transformation is worth the earned speedup. \newline 
\null \quad \quad Once again, we will take advantage of the definition of conditional independence: $$P(A|B) = \frac{P(A,B)}{P(B)}.$$ 

\begin{lemma}
Computing a conditionless query $q = p(X_{0}, X_{1}, ..., X_{n})$ over a DAG $G$ depends on vertices $X_{0}, X_{1}, ..., X_{n}$ and all of their ancestors: $\alpha(X_{0}), \alpha(X_{1}), ..., \alpha(X_{n})$. 
\end{lemma}

\begin{proof}
Let $q = p(X_{0}, X_{1}, ..., X_{n})$ on a DAG $G$. Using the definition of a Bayesian network, we have that the joint probability distribution 
$$p(X_{0}, X_{1}, ... X_{n}) = \prod_{i=0}^{n} p(X_{i}|\prod\nolimits_{X_{i}}^{G}).$$ 
Therefore $q$ indeed depends on all ancestors of the targets. 
\end{proof}


%--------------------------------------threenode query example
\begin{example}\label{qe1}


 Let $q = p(X_{1} = x_{1})$ be a query on $G = (V,E)$ in Figure~\ref{fig:qe1}. The computation of $q$ is as follows:
 \begin{figure}[h!]
\centering
\begin{tikzpicture}
\Vertex[x=0, y=0, size=.5, color=white, label=$X_{1}$]{x1}
\Vertex[x = 2, y=0, size=.5, color=white, label=$X_{2}$]{x2}
\Vertex[x=4, y=0 ,size=.5,color=white, label=$X_{3}$]{x3} 
\Edge[Direct, distance=0.5](x3)(x2)
\Edge[Direct, distance=0.5](x2)(x1)
\Text[x = 2, y = -1]{$G$}
\end{tikzpicture}
\caption{}
\label{fig:qe1}
\end{figure}
 \begin{align*}
	p(X_{1} = x_{1}) & 	\ \ = \ \ \sum_{x_{2}} \sum_{x_{3}} p(X_{1}=x_{1}|X_{2}=x_{2}) \cdot p(X_{2}=x_{2}|X_{3}=x_{3}) \cdot p(X_{3}=x_{3}), \\
\end{align*}
and therefore requires considering the target vertex $X_{1}$ and its ancestors, namely $X_{2}$ and $X_{3}$.
\end{example}
%--------------------------------------\threenode query example


\begin{definition} \label{thm:querycostonetarget}
Let $q = p(X|Y)$ be a query on a DAG $G$. We refer to the cases that the computation of $q$ falls under as the following, depending on the position of $Y$ in $G$: 
\begin{itemize}
\item \textbf{Special case} If $X$ is conditionally independent of all ancestors of $Y$ given $Y$, then $q$ depends on $X$, $\alpha(X)\textbackslash \alpha(Y)$, and all vertices on a route from $X$ to $Y$ excluding $Y$ itself. 
\item \textbf{General case} Otherwise, in the case that $X$ is \textbf{not} independent of $\alpha(Y)$ given $Y$, $q$ depends on both $X$ and $Y$ in addition to the sets $\alpha(X)$ and $\alpha(Y)$. 
\end{itemize}
\end{definition}

\begin{remark}
The motivation for the contents of these sets is given via Theorem \ref{thm:querycost}.
\end{remark} 

\begin{remark}
Intuitively, one would assume that the General case also requires $\Delta(G,q)$ to contain all vertices on a route from $X$ to $Y$. This is only sometimes correct; we argue in Lemma~\ref{routeandancestors} that the inclusion of vertices along such a route is already covered by $\alpha(X) \cup \alpha(Y)$ when necessary, and is not included in $\alpha(X) \cup \alpha(Y)$ when $Z$ is not required in $\Delta(G,q)$.
\end{remark}


\begin{figure}[h!]
\begin{center}
\begin{tikzpicture}
\Vertex[x = 0, y=3, size=.5, color=white, label=$X$]{x}
\Vertex[x = 0, y=1.5, size=.5, color=white, label=$Z$]{y}
\Vertex[x=0, y = 0, size=.5,color=white, label=$Y$]{z}
\Edge[Direct, distance=0.5](x)(y)
\Edge[Direct, distance=0.5](y)(z)
\Text[x=-.8 ,y=3]{$G_{1}$}

\Vertex[x = 3, y=3, size=.5, color=white, label=$X$]{x}
\Vertex[x = 3, y=1.5, size=.5, color=white, label=$Z$]{y}
\Vertex[x=3, y = 0, size=.5,color=white, label=$Y$]{z}
\Edge[Direct, distance=0.5](y)(x)
\Edge[Direct, distance=0.5](y)(z)
\Text[x=2.2 ,y=3]{$G_{2}$}

\Vertex[x = 6, y=3, size=.5, color=white, label=$X$]{x}
\Vertex[x = 6, y=1.5, size=.5, color=white, label=$Z$]{y}
\Vertex[x=6, y = 0, size=.5,color=white, label=$Y$]{z}
\Edge[Direct, distance=0.5](z)(y)
\Edge[Direct, distance=0.5](y)(x)
\Text[x=5.2 ,y=3]{$G_{3}$}

\Vertex[x = 10, y=3, size=.5, color=white, label=$X$]{x}
\Vertex[x = 10, y=1.5, size=.5, color=white, label=$Z$]{y}
\Vertex[x=10, y = 0, size=.5,color=white, label=$Y$]{z}
\Edge[Direct, distance=0.5](z)(y)
\Edge[Direct, distance=0.5](x)(y)
\Text[x=9.2 ,y=3]{$G_{4}$}

\draw [decorate,decoration={brace,amplitude=8pt,mirror,raise=-5ex}]
  (-.8,-1.3) -- (6.8,-1.3) node[midway,yshift=.5em]{$Z$ in either $\alpha(X)$ or $\alpha(Y)$};
\draw [decorate,decoration={brace,amplitude=8pt,mirror,raise=-5ex}]
  (9,-1.3) -- (11,-1.3) node[midway,yshift=.5em]{$Z$ not in $\alpha(X)$ nor $\alpha(Y)$};
\end{tikzpicture}
\end{center}
\caption{}
\label{ancestorinclusion}
\end{figure}


\begin{lemma} \label{routeandancestors}
Let $X$ and $Y$ be two vertices in a DAG $G$. Let $Z$ be the vertex on a route between $X$ and $Y$. Then $Z$ is either included in the set $\alpha(X) \cup \alpha(Y)$, or is not required in $\Delta(G,q)$. 
\end{lemma}

\begin{proof}
To demonstrate this, consider all possible arrangements of three such connected vertices $X$, $Y$, and $Z$, as shown in Figure~\ref{ancestorinclusion}. In the first three graphs, $G_{1}, G_{2}$ and $G_{3}$, $Z$ is clearly an ancestor of either $X$, $Y$, or both simultaneously. Therefore $Z$ is already included through the inclusion of the sets $\alpha(X)$ and $\alpha(Y)$ in the cases of $G_{1}, G_{2}$, and $G_{3}$. Then, we consider the fourth arrangement, $G_{4}$. We will show that in this scenario, $Z$ is not required to compute either $p(X|Y)$ or $p(Y|X)$. Without loss of generality, we show this for $p(X|Y)$. We can express it as the following: \newline
$$p(X|Y) = \frac{\sum_{Z} p(Z|X,Y)\cdot p(X)\cdot p(Y)}{p(Y)}.$$ Here, both $p(X)$ and $p(Y)$ are conditionally independent of $Z$, since here, $Z$ is not a parent of either of them. Therefore, we can pull both $p(X)$ and $p(Y)$ out of the sum $\sum_{Z}p(Z|X,Y)\cdot p(X) \cdot p(Y)$. This gives us 
$$\sum_{Z} (p(Z|X,Y)\cdot p(X) \cdot p(Y)) = p(X) \cdot p(Y) \cdot \sum_{Z} p(Z|X,Y).$$
Noticing that $\sum_{Z}p(Z|X,Y)$ is a probability density, it must equal $1$. Therefore the above becomes 
$$p(X) \cdot p(Y) \cdot 1 = p(X) p(Y)$$
which does not depend on $Z$. We conclude that $Z$ is not necessary for the computation of $p(X|Y)$, and without loss of generality, $p(Y|X)$. \newline
\null \quad \quad Furthermore, we affirm that $Z$ is necessary in the cases of $G_{1}, G_{2},$ and $G_{3}$ by showing that the computations applied to $G_{4}$ cannot be applied to these cases. This can be seen by realizing that in the first three graphs, at least one of the vertices $X$ and $Y$ depends on $Z$, and therefore $p(X)$ and $p(Y)$ cannot be pulled out of the sum, since they are not conditionally independent of $Z$. The results is that $Z$ must remain in $\Delta(G,q)$ in these cases. 
\end{proof}

%\begin{corollary} 
%Lemma~\ref{routeandancestors} holds for a set $\mathcal{Z}$ such that each $Z_{i} \in \mathcal{Z}$ is on the route between $X$ and $Y$, but $X$ and $Y$ do not depend on any $Z_{i}$. 
%\end{corollary}

%\begin{proof}
%This can be seen using fact that $G$ is acyclic: if $G$ is acyclic, and $Z_{1}, Z_{2},...Z_{n}$ are on a route from $X$ to $Y$, then without loss of generality, each element $Z_{i} \in \mathcal{Z} =\{Z_{0},Z_{1},...,Z_{i} \}$ is either (1), in $\alpha(X) \cup \alpha(Y)$, or (2), $p(X)$ and $p(Y)$ are conditionally independent of from $Z_{i}$. If this were not the case, we would be able to find a cycle from some $Z_{i}$ back to $X$ or $Y$ which are themselves ancestors of $Z_{i}$.
%\end{proof}

% To see this, we derive a contradiction: \newline 
%\null \quad \quad Suppose $Z_{i} \in \mathcal{Z}$ is, without loss of generality, an ancestor of $X$. The
%\null \quad \quad Trivially, if we suppose that $Z_{i} \in \mathcal{Z}$ is \textbf{not} an ancestor of $X$ or $Y$, but $p(X)$ and $p(Y)$ are not conditionally independent of it, we derive a contradiction since $p(X)$ and $p(Y)$ must then dependent on it, and therefore be descendants of it. Thus, $Z_{i}$ must either be an ancestor, or $p(X)$ and $p(Y)$ msut be conditionally independent from it. 
%\end{proof}

\begin{lemma}\label{lemma:alternativecases}
 The conditions for the cases in Definition~\ref{thm:querycostonetarget} can be rephrased as follows: 
\begin{itemize}
\item The condition $Y$ is in the \textbf{special case} if, when $Y$ is removed from $G$, the resulting graph $G'$ consists of at least two disconnected subgraphs $G'_{1}$ and $G'_{2}$ such that $X \in G'_{1}$ and all vertices in $\alpha(Y) \in G'_{2}$. That is, removing $Y$ disconnects $\alpha(Y)$ from $X$.
\item The condition $Y$ is in the \textbf{general case} otherwise. That is, whenever the removal of $Y$ does not result in a disconnected graph such that $X$ and all vertices in $\alpha(Y)$ are in separate subgraphs. 
\end{itemize}
\end{lemma}

\begin{proof}
We prove this using the definition of conditional independence. Without loss of generality, assume that the graph contains exactly three vertices $X$, $Y$, and $Z$. If $Y$ is in the special case, where $X$ and $Z$ are conditionally independent given $Y$, then $p(X|Z,Y) = p(X|Y)$. Then, using this conditional independence the probability distribution can be dissected as follows:
\begin{align*}
 p(X,Y,Z)& \ \ = \ \ \prod_{V= X,Y,Z} p(V| \prod\nolimits_{V}^{G}) \\[1em]
& \ \ = \ \ p(X|\prod\nolimits_{X}^{G} \textbackslash Z) \cdot p(Z|\prod\nolimits_{Y}^{G} \textbackslash X) \cdot p(Y|\prod\nolimits_{Y}^{G}).
\end{align*}
Then, $X$ and $Z$ are in separate components after the removal of $Y$, and therefore are disconnected. \newline
\null \quad \quad In contrast, if $Y$ is in the general case, then $X$ and $Z$ are not conditionally independent given $Y$, and therefore the probability distribution cannot be separated into two disjoint components as above. This is because there is a path from $Z$ to $X$ after removing $Y$. We conclude that $X$ and $Z$ are still connected. \newline
\null \quad \quad Since this was done without loss of generality, the above proof holds for sets of vertices $\mathcal{X} = \{X_{1}, ... X_{m}\}$ (and likewise for $Y$ and $Z$.)
\end{proof}

%%%%%%%%%%%%%%%%%%%%%%%%%%%%%%%%%%%%

%----------------------Example of general VS special case
 \begin{figure}[h!]
\centering
\begin{center}
\begin{tikzpicture}
\Vertex[x=-1.3=,y=2, size=.5, color=white, label=$Z$]{Z}
\Vertex[x=1.3,y=2, size=.5, color=white, label=$W$]{W}
\Vertex[x=0,y=.7 ,size=.5, color=white, label=$Y$]{Y} 
\Vertex[x=0, y=-.7, size=.5, color=white, label=$X$]{X}

\Edge[Direct, distance=0.5](Z)(X)
\Edge[Direct, distance=0.5](Y)(X)
\Edge[Direct, distance=0.5](Z)(Y)
\Edge[Direct, distance=0.5](W)(Y)


\Vertex[x=4.7=,y=2, size=.5, color=white, label=$Z$]{Z}
\Vertex[x=7.3,y=2, size=.5, color=white, label=$W$]{W}
\Vertex[x=6,y=.7 ,size=.5, color=white, label=$Y$]{Y} 
\Vertex[x=6, y=-.7, size=.5, color=white, label=$X$]{X}

\Edge[Direct, distance=0.5](Y)(X)
\Edge[Direct, distance=0.5](Z)(Y)
\Edge[Direct, distance=0.5](W)(Y)

\Text[x =0, y= -1.5]{General case}
\Text[x=6,y=-1.5]{Special case}
\end{tikzpicture}
\end{center}
\caption{Example of one DAG in the general case and one in the special case.}
\label{fig:qe2}
\end{figure}

%----------------------\Example of general VS special case


\null \quad \quad When considering queries with more than one condition, one cannot classify the DAG $G$ as being either in the special case or general case, but rather, must classify each condition $Y_{j}$ into the cases. This is because a target $X$ might be independent of $\alpha(Y_{0})$ given a condition $Y_{0}$, but might still be dependent on $\alpha(Y_{1})$ given $Y_{1}$. Then, $Y_{0}$ would be in the special case, whereas $Y_{1}$ would be in the general case. As a result, the query would not depend on $Y_{0}$ and its ancestors, but would still depend on $Y_{1}$ and its ancestors. \newline
\null \quad \quad Furthermore, when considering multiple targets $X_{0}, X_{1}, ... X_{i}$, one must check whether a set of ancestors $\alpha(Y_{j})$ is independent of all vertices $X_{i}, i\in [0,m]$ given $Y_{j}$. Otherwise, if one of the targets was dependent on an ancestor of $Y_{j}$, the query $q$ would depend on that ancestor. From this, we arrive at Theorem ~\ref{thm:querycost}.

\begin{theorem} \label{thm:querycost}
Let $q = p(X_{0}, X_{1}, ..., X_{m} | Y_{0}, Y_{1}, ..., Y_{n})$ be a query on a DAG $G$. Then the computation of $q$ requires the following sets of vertices depending on the structure of $G$:
\begin{enumerate}
\item $q$ always depends on all $X_{i}$ and $\alpha(X_{i})\textbackslash \alpha(Y_{j})$ , 
\item For each $Y_{j}, j\in[0,n]$ in the special case, $q$ depends on all vertices along a route from each vertex $X_{i}, i\in [0,m]$ to $Y_{j}$, not including $Y_{j}$ itself, 
\item For each $Y_{j}, j\in[0,n]$ in the general case, $q$ depends on vertices $Y_{j}$ and $\alpha(Y_{j})$.
\end{enumerate}

Therefore, if $Y_{j}$ is in the special case, $q$ does not depend on $Y_{j}$ nor $\alpha(Y_{j})$. 
\end{theorem}

\begin{proof}
Without loss of generality, suppose we have three vertices $X$, $Y$, and $Z$. Then, let $q = p(X|Y)$ be a query asked on a DAG $G$. Using the definition of a Bayesian network, we have the following joint probability distribution in the general case:
$$p(X, Y) =  p(X|\prod\nolimits_{X}^{G}) \cdot p(Y|\prod\nolimits_{Y}^{G}).$$ \newline
\null \quad \quad Therefore $q$ indeed depends on all ancestors of target $X$ and the condition $Y$. As argued in~\cref{routeandancestors}, if a vertex $Z$ on the route between $X$ and each $Y$ is required to compute $q$, it is already included in the sets $\alpha(X)$ and $\alpha(Y)$: this is because a vertex $Z$ on such a route is either representable in terms of other vertices in the ancestor sets, or is already included in them. \newline
\null \quad \quad Next, we show that vertex $Y$ in the Special case can be excluded from $\Delta(G,q)$ in addition to the ancestors $\alpha(Y)$. Suppose all ancestors of $Y$ are conditionally independent of $X$ given $Y$. Then, when we condition on $Y=y$, we are setting it to a specific value and effectively removing from the model; ancestors of $Y$ no longer play a role since we are not interested in how they affect a fixed value, and they do not affect any variable past $Y$ in the model.\newline 
\null \quad \quad Likewise, all other variables are conditioned on $Y$ itself, and therefore we do not need to compute how $Y$ is affected. Therefore, we need only to consider vertices on a route from $X$ to $Y$ (in order to determine how $Y=y$ affects the targets), but not $Y$ or $\alpha(Y)$. \newline
\null \quad \quad The above reasoning holds for sets of vertices $\mathcal{X} = \{X_{1}, ... , X_{m}\}$, $\mathcal{Y}$ and $\mathcal{Z}$.
\end{proof}

\begin{center}
\begin{figure}[h!]
\centering
\scalebox{.8}{
   \begin{tikzpicture}
	        \node at (-2.25cm,2cm) [rectangle,draw, minimum size=.8cm, minimum width = 3cm] (ancestors) {$$};
	        	        \Text[x=-2.25,y=2]{$Z_{0}, Z_{1}, ..., Z_{k}$}
	        \node at (0cm,0cm) [circle,draw, minimum size=.8cm, thick] (Yjs) {$$};
        		        \Text[x=0,y=0]{$Y_{j}$}
	        \node at (-4.5cm,0cm) [rectangle,draw, minimum size=.8cm, minimum width=3cm] (Xjs) {$$};
	                  \Text[x=-4.5,y=0]{$X_{1}, X_{2}, ..., X_{m}$}
	       \draw (ancestors) -- (Yjs);
	       \draw (Xjs) -- (Yjs);
	       \Text[x=-2.25,y=-1]{Special case}
	       
	       
	        \node at (8.25cm,2cm) [rectangle,draw, minimum size=.8cm, minimum width = 3cm] (ancestors) {$$};
	        	        \Text[x=8.25,y=2]{$Z_{0}, Z_{1}, ..., Z_{k}$}
	        \node at (10.5cm,0cm) [circle,draw, minimum size=.8cm, thick] (Yjs) {$$};
        		        \Text[x=10.5,y=0]{$Y_{j}$}
	        \node at (6cm,0cm) [rectangle,draw, minimum size=.8cm, minimum width=3cm] (Xjs) {$$};
	                  \Text[x=6,y=0]{$X_{1}, X_{2}, ..., X_{m}$}
	       \draw (ancestors) -- (Yjs);
	       \draw (Xjs) -- (ancestors);
	       \draw (Xjs) -- (Yjs);
	       
	      \Text[x=8.25,y=-1]{General case}
	       
    \end{tikzpicture}
}
    \caption{Graphical structure of the two cases: in the general case, $X_{i}$s are not conditionally independent from all elements $Z_{k} \in \alpha(Y_{j})$ given $Y_{j}$. In the special case, they are.}
    \label{fig:querycases}
\end{figure}
\end{center}


\begin{example} Here, we apply Theorem~\ref{thm:querycost} to determine the set of vertices a query $q$ over the DAG $G$ in Figure~\ref{fig:querydependentex} depends on. An explicit calculation to verify this is not given here; instead, we simply identify the set. Such calculations will take place in later examples. \newline


 \begin{figure}[h!]
\centering
\begin{center}
\begin{tikzpicture}
\Vertex[x=1.5, y=0, size=.5, color=white, label=$X$]{X}
\Vertex[x=0, y=1.5, size=.5, color=white, label=$Y_{0}$]{Y0}
\Vertex[x=3, y=1.5, size=.5,color=white, label=$Y_{1}$]{Y1} 
\Vertex[x=0, y=3.5, size=.5, color=white, label=$Z_{0}$]{Z0}
\Vertex[x=1.5, y=3.5, size=.5,color=white, label=$W_{0}$]{W0} 
\Vertex[x=3, y=3.5, size=.5, color=white, label=$W_{1}$]{W1}


\Edge[Direct, distance=0.5](Y0)(X)
\Edge[Direct, distance=0.5](Y1)(X)
\Edge[Direct, distance=0.5](W0)(Y1)
\Edge[Direct, distance=0.5](W1)(Y1)
\Edge[Direct, distance=0.5](Z0)(Y0)
\Edge[Direct, distance=0.5](W0)(X)

\Text[x = 1.5, y = -1]{$G$}
\end{tikzpicture}
\end{center}
\caption{}
\label{fig:querydependentex}
\end{figure}

Suppose $q = p(X|Y_{0},Y_{1})$. Then, for each of the conditions $Y_{0}$ and $Y_{1}$, we determine whether it is in the general case of the special case. Immediately, we know that the query depends on at least the target $X$ and the set $\alpha(X)$. 
\begin{itemize}
\item $Y_{0}$ has only one ancestor, namely $Z_{0}$. $Z_{0}$ is conditionally independent of $X$ given $Y_{0}$. Therefore, the condition $Y_{0}$ is in the special case. As a result, $q$ does not depend on $Y_{0}$ nor $Z_{0}$. 
\item $Y_{1}$ has two ancestors, namely $W_{0}$ and $W_{1}$. $X$ is not conditionally independent from $W_{0}$ given $Y_{1}$, since we can find a route to $W_{0}$ which does not pass through $Y_{1}$. Therefore, $Y_{1}$ is in the general case, and we conclude that the query still additionally depends on $W_{0}$, $W_{1}$ and $Y_{1}$
\end{itemize}
We conclude that $\Delta(G,q) = \{X, Y_{1}, W_{0}, W_{1}\}$.  
\end{example}


%-------------------------------------fournode query example

\null \quad \quad The following examples verify which vertices are in the set $\Delta(G,q)$ outlined in Theorem ~\ref{thm:querycost}. First we present several computations in the general case, then computations in the special case. Finally, we show how two different queries over the same DAG $G$ can fall into each case. 

\begin{example}\label{ex:qe2}

\textbf{General Case.} Let $q = p(X_{3} = x_{3}, X_{4} = x_{4})$ be a query on $G = (V,E)$ in Figure~\ref{fig:qe2}. The computation of $q$ is as follows.
 
 \begin{figure}[h!]
\centering
\begin{center}
\begin{tikzpicture}
\Vertex[x=0, y=1, size=.5, color=white, label=$X_1$]{x1}
\Vertex[x = 3.5, y=0, size=.5, color=white, label=$X_{4}$]{x2}
\Vertex[x=3.5, y=2 ,size=.5,color=white, label=$X_{3}$]{x3} 
\Vertex[x = 1.8, y=1, size=.5, color=white, label=$X_{2}$]{x4}

\Edge[Direct, distance=0.5](x1)(x4)
\Edge[Direct, distance=0.5](x4)(x3)
\Edge[Direct, distance=0.5](x4)(x2)

\Text[x = 1.8, y = -1]{$G$}
\end{tikzpicture}
\end{center}
\caption{}
\label{fig:qe2}
\end{figure}
 
\begin{align*}
 q 	& \ \ = \ \  p(X_{3} = x_{3}, X_{4} = x_{4}) \\[1em]
	& \ \ = \ \ \sum_{x_{2}} \sum_{x_{1}} p(X_{3}=x_{3},X_{4}=x_{4}|X_{2}=x_{2}) \\[1em] 
		& \quad \quad \cdot p(X_{2}=x_{2}|X_{1}=x_{1}) \cdot p(X_{1}=x_{1}) & (1)\\[1em] 
	& \ \ = \ \ \sum_{x_{2}} \sum_{x_{1}} p(X_{3}=x_{3}|X_{2}=x_{2})p(X_{4}=x_{4}|X_{2}=x_{2}) \\[1em] 
		& \quad \quad  \cdot p(X_{2}=x_{2}|X_{1}=x_{1}) \cdot p(X_{1}=x_{1}) & (2)\\[1em] 
	& \ \ = \ \ \sum_{x_{2}} \sum_{x_{1}} p(X_{3}=x_{3}|X_{2}=x_{2}) \cdot p(X_{4}=x_{4}|X_{2}=x_{2}) \\[1em]  
		& \quad \quad \cdot p(X_{1}=x_{1}, X_{2}=x_{2}) & (3)
\end{align*}

where equality (1) is by construction of the Bayesian network and equality (2) comes from the definition of conditional probability since $X_{3}$ and $X_{4}$ are conditionally independent. Equality (3) comes from the following application of conditional probability:
\begin{align*}
p(X_{2}=x_{2}| X_{1}=x_{1}) \cdot p(X_{1}=x_{1})	& \ \ = \ \  p(X_{1}=x_{1}, X_{2} = x_{2}). 
\end{align*}

The query $q$ depends on the targets $X_{4}$ and $X_{3}$ and their parents, $X_{2}$ and $X_{1}$. 

\end{example}





%--------------------------------------\fournode query example



%-------------------------------------Querycost Example
\begin{example}

\textbf{General case.} The following example follows similar structure to Example~\ref{ex:qe2} and uses real values for the conditional probabilities. Consider the graph $G=(V,E)$ from ~\ref{fig:costofquery}, where a vertex $V$ takes a binary value $v \in [0,1]$, and suppose we are given a query $q = p(Y=1|Z=1)$. \newline
\begin{figure}[h!]
\centering
\begin{center}
\begin{tikzpicture}
\Vertex[x=0, y=1, size=.5, color=white, label=$X$]{x1}
\Vertex[x = 3.5, y=0, size=.5, color=white, label=$U$]{x2}
\Vertex[x=3.5, y=2 ,size=.5,color=white, label=$W$]{x3} 
\Vertex[x = 1.8, y=1, size=.5, color=white, label=$Y$]{x4}
\Vertex[x = 5.3, y=2, size=.5, color=white, label=$Z$]{x6}

\Edge[Direct, distance=0.5](x1)(x4)
\Edge[Direct, distance=0.5](x4)(x3)
\Edge[Direct, distance=0.5](x4)(x2)
\Edge[Direct, distance=0.5](x3)(x6)

\Text[x = 2.65, y = -1]{$G$}

\end{tikzpicture}
\end{center}
\caption{}
\label{fig:costofquery}
\end{figure}
\null \quad \quad Let the conditional probabilities be given by

\begin{align*}
	& p(X = 1) = 0.2 			& \\
	& p(Y = 1 | X = 0) = 0.6		& p(Y = 1 | X = 1) = 0.4 \\ 
	& p(W = 1 | Y = 0) = 0.4		& p(W = 1 | Y = 1) = 0.3 \\
	& p(Z = 1 | W = 0) = 0.7		& p(Z = 1 | W = 1) = 0.3 \\ 
	& p(U = 1 | Y = 0) = 0.8		& p(U = 1 | Y = 1) = 0.4 \\
\end{align*}


Then, the joint probability is given by $$p(X,Y,W,U,Z) = p(X) \cdot p(Y|X) \cdot p(U|Y) \cdot p(W|Y) \cdot p(Z|W).$$ 
\null \quad \quad The aim of this example is to demonstrate that our query depends on the set of ancestors $\alpha(Y)$ and the vertices on route from $Z$ to $Y$, in this case, the route $Y \rightarrow W \rightarrow Z$. For this computation, we set $Z=1$, and recall that since we are in a binary setting, for a vertex $V$ we have $p(V=0) = 1 - p(V=1)$. Then, to determine $p(Y=1|Z=1)$, we first compute $p(Y=1)$:

\begin{align*}
	p(Y = 1) 	& \ \ = \ \ \sum_{X} p(Y = 0|x) \cdot p(x) \\ 
			& \ \ = \ \ 0.6 \cdot 0.8 + 0.4 \cdot 0.2 \\
			& \ \ = \ \ 0.48 + 0.08 \\
			& \ \ = \ \ 0.56 \\
	P(Y = 0)        & \ \ = \ \  1 - p(Y=1) = 0.44. \\
\end{align*}

By similar calculation and plugging in the values $p(Y=1)$ and $p(Y=0)$, we obtain that $p(W = 0) = \sum_{Y} p(W=0|Y)p(Y) \approx 0.65$ and $P(W = 1) \approx 0.34$. Likewise, we compute $p( Z = 1) \approx 0.56$ and $p(Z = 0) \approx 0,43$. \newline
\null \quad \quad From here, we determine how the observation $Z=1$ affects the likelihood of values of $W$ and consequently $Y$. Bayes' theorem allows us to do the calculation using quantities specified in the conditional probabilities:

$$p(W=1 | Z = 1) \ \ = \ \ \frac{p(Z=1|W=1)\cdot p(W=1)}{p(Z=1)} \ \ \approx \ \ \frac{0.3 \cdot 0.34}{0.56} \ \ \approx \ \ 0.18.$$

Subsequently $p(W = 0 | Z = 1) \approx 0.81$. Now that we see how $W$ is affected by the observation that $Z = 1$, we can move upward in the set of ancestors to determine which vertices $Y$ is affected by. Continuing with our condition that $Z=1$ and applying Bayes' theorem again,

\begin{align*}
	p(Y = 1 | W = 1)  \ \ = \ \ \frac{ p(W = 1 | Y = 1) \cdot p(Y = 1) }{p(W=1)} \ \ \approx \ \ \frac{0.18 \cdot 0.44}{0.18} \ \ \approx \ \ 0.48.
\end{align*}

Finally,

\begin{align*}
	p(Y = 1 | Z = 1) 	& \ \ = p(Y = 1|W = 0)\cdot p(W = 0) + p(Y=1|W=1)\cdot p(W=1) \\ 
			& \ \ \approx \ \ (0.59 \cdot 0.81)+(0.48 \cdot 0.18) \\
			& \ \ \approx \ \ 0.17.
\end{align*}

Thus, by tracing the effects of the observation $Z=1$ on a route to $Y=1$ (contained within $\alpha(Z)$) and using the ancestors $\alpha(Y)$, we have answered the query $q = p(Y = 1|Z = 1)$. Notice that this computation did not involve the vertex $U$, which is neither an ancestor of $Y$ nor along the route between $Y$ and $Z$. 
\end{example}

%-------------------------------------\Querycost Example

%-------------------------------------Special case1 Example

 \begin{figure}[h!]
\centering
\begin{tikzpicture}
\Vertex[x=0, y=0, size=.5, color=white, label=$X_{1}$]{x1}
\Vertex[x = 2, y=0, size=.5, color=white, label=$X_{2}$]{x2}
\Vertex[x=4, y=0 ,size=.5,color=white, label=$X_{3}$]{x3} 
\Vertex[x=6, y=0 ,size=.5,color=white, label=$X_{4}$]{x4} 
\Edge[Direct, distance=0.5](x4)(x3)
\Edge[Direct, distance=0.5](x3)(x2)
\Edge[Direct, distance=0.5](x2)(x1)
\Text[x = 3, y = -1]{$G$}
\end{tikzpicture}
\caption{}
\label{fig:specialcase1}
\end{figure}

\begin{example} \textbf{Special Case.}  Let $q = p(X_{4}=x_{4}|X_{2}=x_{2})$ be a query on $G$ in Figure~\ref{fig:specialcase1}. Then the query only involves vertices $X_{4}, X_{3}$, and $X_{2}$, so there is no need to consider vertex $X_{1}$. This is because $X_{4}$ is conditionally dependent of $\alpha(X_{2}) = X_{1}$. 


$$ p(X_{4}=x_{4}|X_{2}=x_{2}) = \sum_{x_{3}} p(X_{4}=x_{4}|X_{3}=x_{3}) \cdot p(X_{3}=x_{3}|X_{2}=x_{2}).$$

\end{example}

%-------------------------------------\Special case 1 Example


%-------------------------------------Special case 2 Example



\begin{figure}[h!]
\centering
\begin{tikzpicture}
\Vertex[size=.5, color=white, label=$X_{1}$]{x1}
\Vertex[x = 2, size=.5, color=white, label=$X_{2}$]{x2}
\Vertex[y = -1.5, x=1 ,size=.5,color=white, label=$X_{3}$]{x3}
\Edge[Direct, distance=0.5](x1)(x3)
\Edge[Direct, distance=0.5](x2)(x3)
\end{tikzpicture}
\caption{}
\label{fig:comp1}
\end{figure}

\begin{example} This example compares two queries over the same graph $G$ in Figure~\ref{fig:comp1}. Let $q_{1} = p(X_{2} = x_{2}|X_{3}=x_{3})$ and let $q_{2} = p(X_{3}=x_{3}|X_{2}=x_{2})$. \newline
\null \quad \quad Note that $X_{2}$ is an ancestor of $X_{3}$, and trivially $X_{2}$ is not conditionally independent from $X_{2}$. Therefore $q_{1}$ is in the general case of Theorem~\ref{thm:querycost}, since the target of $q_{1}$ (namely $X_{2}$) is not disjoint from the ancestors of the condition, $\alpha(X_{3})$. The computation of $q_{1}$ therefore requires the target $X_{2}$, the ancestors $\alpha(X_{2})$, the condition $X_{3}$, and $\alpha(X_{3})$. These sets together involve $X_{1}, X_{2}$, and $X_{3}$. This dependence is verified:

\begin{align*}
q_{1} & \ \ = \ \ p(X_{2}=x_{2}|X_{3}=x_{3}) \\[1.2em]
& \ \ = \ \ \frac{p(X_{2}=x_{2},X_{3}=x_{3})}{p(X_{3}=x_{3})} \\[1em]
& \ \ = \ \ \frac{ \sum_{x_{1}} p(X_{1}=x_{1}) \cdot p(X_{2}=x_{2}) \cdot p(X_{3}=x_{3}|X_{1}=x_{1},X_{2}=x_{2})} { \sum_{x_{1}} \sum_{x_{2}} p(X_{1}=x_{1}) \cdot p(X_{2} = x_{2}) \cdot p(X_{3}=x_{3}|X_{1}=x_{1},X_{2}=x_{2})}.
\end{align*}

\null \quad \quad Now we consider $q_{2}$. In this case, the conditional variable $X_{2}$ has no ancestors, and therefore we do not need to consider its distribution, since we are in the special case of Theorem~\ref{thm:querycost}. We show below that $q_{2}$ only relies only on $X_{3}$ and $X_{1}$. \newline

\begin{figure}[h!]
\centering
\begin{tikzpicture}
\Vertex[size=.5, color=white, label=$X_{1}$]{x1}
\Vertex[x = 2, size=.5, color=white, label=$X_{2}$]{x2}
\Vertex[y = -1.5, x=1 ,size=.5,color=white, label=$X_{3}$]{x3}
\Edge[Direct, distance=0.5](x1)(x3)
\Edge[Direct, distance=0.5](x2)(x3)

\Vertex[x=5, size=.5, color=white, label=$X_{1}$]{x11}
\Vertex[x = 7, size=.5, color=white, opacity={0.3}, label=$x_{2}$]{x21}
\Vertex[y = -1.5, x=6 ,size=.5,color=white, label=$X_{3}$]{x31}
\Edge[Direct, distance=0.5](x11)(x31)
\Edge[Direct, distance=0.5, color=gray, style={dashed}, opacity = {0.4}](x21)(x31)
\end{tikzpicture}
\caption{$X_{2}$ effectively removed from the model by conditioning $X_{2}=x_{2}$.}
\label{fig:comp1fade}
\end{figure}


\null \quad \quad Intuitively, setting $X_{2}=x_{2}$ effectively removes the variable from the model, since we are conditioning other variables on this observation. Once the other variables are adjusted, $X_{2}$ no longer plays a role since it already has an established value and does not have parents that must be considered. That is, we are setting $p(X_{3}=x_{3}|X_{1}=x_{1}) = p(X_{3}=x_{3}'|X_{1}=x_{1}', X_{2}=x_{2})$ for all $x_{1}', x_{2}'$. Thus:

\begin{align*}
q_{2} & \ \ = \ \ p(X_{3}=x_{3}|X_{2}=x_{2}) & \\[1em]
& \ \ = \ \ \frac{p(X_{3}=x_{3}, X_{2}=x_{2})}{p(X_{2}=x_{2})} \\[1em]
& \ \ = \ \ p(X_{3}=x_{3})  \\[1em]
& \ \ = \ \ \sum_{x_{1}} p(X_{3}=x_{3}|X_{1}=x_{1},X_{2}=x_{2}) \cdot p(X_{1}=x_{1}) \\[1em]
& \ \ = \ \ \sum_{x_{1}} p(X_{3}=x_{3}|X_{1}=x_{1}) \cdot p(X_{1}=x_{1}). 
\end{align*}

Therefore, $q_{2}$ relies only on $X_{3}$ and $X_{1}$ as described in the special case, whereas $q_{2}$ relies on $X_{2}$ as well, as described in the general case. 


\end{example}

\section{Arithmetic Complexity of Computing a Query}\label{section:complexityofaquery}

\null \quad \quad Once the set of vertices a query depends on ($\Delta(G,q)$) has been established, we can consider the arithmetic cost of the computations involved. The goal of this section is to determine the number of operations (+, $\cdot$) necessary to compute a query. We will calculate this arithmetic cost in terms of how it scales with the inputs of the query. \newline
\null \quad \quad Let $G = (V,E)$ be a DAG and $q$ a query. Let $\mathcal{X}, \mathcal{Y}, \mathcal{Z}, \subseteq V$, where targets $X_{i} \in \mathcal{X}$ and conditions $Y_{j}$ are in $\mathcal{Y}$. Let $\mathcal{Z}$ be the set of vertices which $q$ depends on, which are neither targets nor conditions: $\mathcal{Z} = \Delta(G,q) \textbackslash (\mathcal{X} \cup \mathcal{Y})$. Then, the form of $q$ is $q = p(\mathcal{X}|\mathcal{Y}) = p(X_{0},...X_{m}|Y_{0},...,Y_{n})$. In order to compute $q$, we must make the following calculations:
\begin{enumerate}
\item Compute the joint distribution $p(\mathcal{X},\mathcal{Y},\mathcal{Z})$, that is, of all variables in $\Delta(G,q)$.
\item Marginalize all $Z_{k} \in \mathcal{Z}$ out of $p(\mathcal{X},\mathcal{Y},\mathcal{Z})$. 
\item Compute the joint marginal distribution of $\mathcal{Y}$.
\item Condition the joint distribution $p(\mathcal{X},\mathcal{Y})$ to get $p(\mathcal{X}|\mathcal{Y})$.
\end{enumerate}

We will break the above operations into separate pieces and quantify the arithmetic cost of each piece in order to compute the total cost of answering $q$. We will present a general argument for the cost of each, and then contextualize it for binary variables. In this section, many computations will make use of the fact that in a binary setting, the sum over the possible assignments of values in a set $\mathcal{A}$ requires adding $2^{|\mathcal{A}|}$-many values. \newline

\textbf{(1) Compute the joint distribution $p(\mathcal{X}, \mathcal{Y}, \mathcal{Z})$}
\vspace{-4mm}

\begin{myindentpar}{1.5em}
\null \quad \quad We begin by determining the cost of calculating the full joint distribution for all vertices in $\Delta(G,q)$, namely $p(\mathcal{X},\mathcal{Y},\mathcal{Z})$. Recall that the joint distribution factorizes as 
$$p(V_{0},V_{1},...V_{|V|-1}) = \prod_{i \in |V|-1} p(V_{i}|\prod\nolimits_{G}^{V_{i}})$$
 where again, the set $\prod\nolimits_{G}^{V_{i}}$ is the set of parents of vertex $V_{i}$. In a binary setting, the number of specifications used to describe the distribution of vertex $V_{i}$ is therefore $2^{|\prod\nolimits_{V_{i}}^{G}|}$. For example, a vertex $X$ with one parent $Y$ is specified by $p(X|Y=1)$ and $p(X|Y=0)$, two distributions.\newline
 \null \quad \quad Notice that the number of parents of a given vertex is not necessarily restricted. We will address this in Chapter 5 when we calculate the complexity of various algorithms. For now, however, we recognize that the number of parents of a vertex is bounded by the number of vertices in $G$, and therefore replace $|\prod\nolimits_{V_{i}}^{G}|$ with $|V|$. \newline
\null \quad \quad The product is over $(|\mathcal{X}| + |\mathcal{Y}| + |\mathcal{Z}|)$-many variables, or equivalently, $|\Delta(G,q)|$-many variables. Then, using the fact that multiplying $n$ values together requires $(n-1)$ operations, the result is a cost of $\mathcal{O}(|\mathcal{X}|+|\mathcal{Y}|+|\mathcal{Z}|)\cdot 2^{|V|}$. We can also bound this as $\mathcal{O}(|\mathcal{X}|+|\mathcal{Y}|+|\mathcal{Z}|)\cdot 2^{|\mathcal{X}|+|\mathcal{Y}|+|\mathcal{Z}|}$ or, to be more asymptotically descriptive, $\mathcal{O}(|V|)\cdot 2^{|V|}$.
\end{myindentpar} 

\textbf{(2) Marginalize $\mathcal{Z}$ out of the distribution to determine $p(\mathcal{X}, \mathcal{Y})$}
\vspace{-4mm}

\begin{myindentpar}{1.5em}
\null \quad \quad Next, we marginalize the elements $Z_{k} \in \mathcal{Z}$ out of the joint distribution $p(\mathcal{X},\mathcal{Y},\mathcal{Z})$ to determine $p(\mathcal{X},\mathcal{Y})$. The joint marginal distribution is given by 
$$p(\mathcal{X},\mathcal{Y}) = \sum_{\mathcal{Z}} p(\mathcal{X},\mathcal{Y}, \mathcal{Z}).$$

Then, since we must compute the marginal for every possible assignment of $\mathcal{X}$ and $\mathcal{Y}$, this requires $2^{|\mathcal{X}|+|\mathcal{Y}|}$-many sums. Each sum has $2^{|\mathcal{Z}|}$ summands, making the total number of operations in $\mathcal{O}(2^{|\mathcal{X}| + |\mathcal{Y}| + |\mathcal{Z}|})$.
\end{myindentpar}

\textbf{(3) Compute the joint marginal distribution of the conditionals $\mathcal{Y}$}
\vspace{-4mm}

\begin{myindentpar}{1.5em}
\null \quad \quad Next, we must compute the joint marginal distribution $p(\mathcal{Y}) = p(Y_{0},...,Y_{n})$ from $p(\mathcal{X},\mathcal{Y})$. To do this, we determine the cost of computing the marginal for a single condition $Y_{j}$ and then multiply it by the number of marginal distributions which must be computed -- one for each condition -- which is $|\mathcal{Y}|$. The joint marginal distribution $p(\mathcal{Y})$ is given by 
$$p(\mathcal{Y}) = \sum_{\mathcal{X}} p(\mathcal{X}, \mathcal{Y}).$$
Likewise, the marginal distribution for a single value $Y_{j} \in \mathcal{Y}$ is given by
$$p(Y_{j}) = \sum_{\mathcal{X}} \sum_{Y_{i} \neq Y_{j}} p(\mathcal{X}, Y_{i})$$
 Again, computing the sum of $n$ values requires $(n-1)$ operations. Then, the sum over $\mathcal{X}$ requires $(2^{|\mathcal{X}|}-1)$ operations. For the marginals of each condition $Y_{j}$ to be computed, we need to sum over the $2^{|\mathcal{Y}|}$-many possible conditions. Therefore, there is a total cost is $\mathcal{O}(2^{|\mathcal{X}| + |\mathcal{Y}|})$.
\end{myindentpar}

\newpage

\textbf{(4) Condition the joint distribution}
\vspace{-4mm}

\begin{myindentpar}{1.5em}
\null \quad \quad Finally, we must condition on elements in $\mathcal{Y}$. That is, we must find the specialized distribution $p(\mathcal{X}|\mathcal{Y})$ by conditioning $p(\mathcal{X}, \mathcal{Y})$ on $\mathcal{Y}$. Using the definition of the conditional, we have
$$p(\mathcal{X}|\mathcal{Y}) = \frac{p(\mathcal{X},\mathcal{Y})}{p(\mathcal{Y})},$$
where again, $p(\mathcal{Y})$ is the joint marginal of all $Y_{0}, Y_{1},...Y_{j} \in \mathcal{Y}$. We already have the numerator and the denominator from Steps $(2)$ and $(3)$ above respectively, meaning we simply have to perform the relevant divisions in this step. In the binary setting this amounts to $2^{(|\mathcal{X}| + |\mathcal{Y}|)}$ divisions. Then Step $(4)$ is also $\mathcal{O}(2^{|\mathcal{X}|+|\mathcal{Y}|})$.
\end{myindentpar}

\textbf{Total}
\vspace{-4mm}

\begin{myindentpar}{1.5em}
 \null \quad \quad In composite, the above pieces allow us to calculate $q$. The complexities of each component add up as shown in Figure \ref{fig:costtable}.
 \begin{center}
 \begin{table}[h!]
 \centering
 \begin{tabular}{r r}
 \   & $\mathcal{O}((|\mathcal{X}|+|\mathcal{Y}|+ |\mathcal{Z}|)\cdot 2^{|\mathcal{X}|+|\mathcal{Y}|+|\mathcal{Z}|})$ \\[1em]
$+$ & $\mathcal{O}(2^{(|\mathcal{X}|+|\mathcal{Y}| + |\mathcal{Z}|)})$ \\[1em]
$+$ & $\mathcal{O}(2^{(|\mathcal{X}| + |\mathcal{Y}|)})$ \\[1em]
$+$ & $\mathcal{O}(2^{|\mathcal{X}| + |\mathcal{Y}|})$ \\[1em]
 \hline \\[-1em]
$=$ & $\mathcal{O}(2^{|\mathcal{X}|+|\mathcal{Y}|+|\mathcal{Z}|})$ \\[1em]
 \end{tabular}
 \caption{}
 \label{fig:costtable}
 \end{table}
 \end{center}
\end{myindentpar}
We therefore see that reducing the size of $\mathcal{Z}$ has an exponential effect on the number or computations required to compute a query. 

\begin{theorem}
Let $G$ be a sparse DAG and let $q = p(\mathcal{X}|\mathcal{Y})$ with $\mathcal{X} = (X_{0},X_{1},...,X_{i})$, $\mathcal{Y} = (Y_{0},Y_{1},...Y_{j})$ and $(Z_{0},Z_{1},...,Z_{k}) = \mathcal{Z} = \Delta(G,q) \textbackslash (\mathcal{X} \cup \mathcal{Y})$. Then, if the size of $\mathcal{Z}$ is reduced by $n$, then the query cost is reduced by $2^{n}$.
\end{theorem}

\begin{proof}
This follows from the costs computed in Table$~\ref{fig:costtable}$. The complexity of answering $q$ is $\mathcal{O}(2^{|\mathcal{X}|+|\mathcal{Y}|+|\mathcal{Z}|})$. If we reduce $\mathcal{Z}$ by 1, we reduce the entire exponent by 1, and therefore, the entire cost is lowered by $2^{|\mathcal{X}|+|\mathcal{Y}|+|\mathcal{Z}|-1}$, meaning the final cost is half of the original cost. 
\end{proof}


\begin{remark}
Notice that reducing the size of $\mathcal{Z}$ (which is the desired result of edge reversals) only affects the complexity of the first two steps in the above querying process. 
\end{remark}

\begin{remark}
There are other ways to compute the marginals using the above steps. For example, if we had chosen to do the computations in the order (1), (3), (4), (2), would would arrive at the same marginal distribution. However, other orderings are more computationally expensive. Notice that if we did the steps in this order, the complexity of conditioning the joint distribution and computing the joint marginal distribution of the conditionals would each be increased from $\mathcal{O}(2^{|\mathcal{X}|+|\mathcal{Y}|})$ to $\mathcal{O}(2^{|\mathcal{X}|+|\mathcal{Y}|+|\mathcal{Z}|})$. It is therefore beneficial to begin the computation by computing the joint distribution, then marginalizing $\mathcal{Z}$ out before conditioning. 
\end{remark}




%%%%%%%%%%%%%%% CHAPTER 4 MEC
% Markov equivalence
\newpage
\chapter{Markov Equivalence}


%----------------------------------------------------------------------------------------
%	Markov Equivalence
%----------------------------------------------------------------------------------------

\section{Goal}

\null \quad \quad In order to reduce the cost of a query on a graph $G$, we will exploit the observation that two Bayesian networks with distinguishable structures can model the same probability distribution. In this case, the corresponding graphs are called \textit{Markov equivalent}. Given a query on a graph $G$, we search for a Markov equivalent graph $G'$ such that the cost of the query is minimized when considered on $G'$ instead of $G$.  \newline
\null \quad \quad To do so, we must develop a robust understanding of Markov equivalence. This includes: how to identify whether two graphs are Markov equivalent, how a graph can be altered while remaining in the set of Markov equivalent graphs (called the \textit{Markov equivalence class}), and how computations of queries on a graph are altered when the graph structure is altered. \newline
\null \quad \quad In this chapter, we present definitions, theorems, and examples to establish a sufficient understanding of Markov equivalence for this purpose. 

\section{Basic Properties of Markov Equivalence}

\begin{definition}[Markov equivalence~\cite{verma}]
Two directed acyclic graphs $G$ and $G'$ are \textbf{Markov equivalent} if for every Bayesian network $B = (G, \theta_{G})$, there exists a Bayesian network $B' = (G', \theta_{G'})$ such that $B$ and $B'$ describe the same probability distribution, and vice versa. This is denoted $G \ME G'$.
\end{definition}
\newpage

\begin{definition}[Markov equivalence class~\cite{verma}]
The set of all DAGs which are Markov equivalent to a DAG $G$ is called the \textbf{Markov equivalence class}(MEC) of $G$, denoted $[G]$.
\end{definition}

\null \quad \quad Significant research has been done to characterize Markov equivalence classes (~\cite{chickering, verma, andersson}). This research has identified several key features of a graphs which allow us to more easily identify reversible edges and construct MECs. The following definitions outline several properties of graphs (and edges within them) which we will use to determine whether two graphs lie in the same MEC. Furthermore, it will help us build an understanding of how a graph can be manipulated without altering which MEC it belongs to. 


%----------------------covered defn
\begin{definition}[Covered Edge~\cite{chickering}]\label{defcoverededge} 
Given a DAG $G=(V,E)$, an edge $e = (X, Y) \in E$ is called \textbf{covered} in $G$ iff $\prod\nolimits_{Y}^{G} = \prod\nolimits_{X}^{G} \cup \ X$. That is, $(X, Y)$ is covered in $G$ iff $X$ and $Y$ have identical parents in $G$ with the exception that $X$ is not a parent of itself.
\end{definition}

\begin{center}
\begin{figure}[h!]
\centering
\begin{tikzpicture}
\Vertex[size=.5, color=gray]{x1}
\Vertex[x = 2, size=.5, color=white]{x2}
\Vertex[y = -2, ,size=.5,color=white]{x3}
\Vertex[y=-2, x=2 ,size=.5,color=white]{x4}
\Edge[Direct, distance=0.5, color=red](x1)(x2)
\Edge[Direct, distance=0.5, color=red](x2)(x4)
\Edge[Direct, distance=0.5](x4)(x3)
\Edge[Direct, distance=0.5](x1)(x4)
\end{tikzpicture}
\caption{ }
\label{fig:covedge}
\end{figure}
\end{center}

In Figure~\ref{fig:covedge}, for example, the red edges are the only covered edges; the upper edge is covered because its vertices both have no parents; the righthand edge is covered because its vertices share the gray vertex as their single parent. 

%----------------------\covered defn


%----------------------skeleton defn

\begin{definition} [Skeleton]
Let $G=(V,E)$ be a directed graph. The undirected graph $G' = (V, E')$ which is constructed by removing the orientation of all edges in $E$ is called the \textbf{skeleton} of $G$. 
\end{definition}
 
\begin{center}
\begin{tikzpicture}
\Vertex[size=.5, color=white]{x1}
\Vertex[x = 2, size=.5, color=white]{x2}
\Vertex[y = -2, ,size=.5,color=white]{x3}
\Vertex[y=-2, x=2 ,size=.5,color=white]{x4}
\Edge[Direct, distance=0.5](x1)(x2)
\Edge[Direct, distance=0.5](x1)(x2)
\Edge[Direct, distance=0.5](x1)(x4)
\Edge[Direct, distance=0.5](x2)(x4)
\Edge[Direct, distance=0.5](x4)(x3)
\Text[x=1, y= -3]{$G$}

\scalebox{1.5}{
\Text[x=2.5 ,y=-.6667]{$\implies$} %because of scalebox: y = 1/1.5, x=3.75/1.5
}

\Vertex[x=5.5, size=.5, color=white]{x1b}
\Vertex[x=7.5, size=.5, color=white]{x2b}
\Vertex[x=5.5, y = -2, ,size=.5,color=white]{x3b}
\Vertex[x=7.5, y=-2,size=.5,color=white]{x4b}
\Edge[distance=0.5](x1b)(x2b)
\Edge[distance=0.5](x1b)(x2b)
\Edge[distance=0.5](x1b)(x4b)
\Edge[distance=0.5](x2b)(x4b)
\Edge[distance=0.5](x3b)(x4b)
\Text[x=6.5, y= -3]{$G' =\ $ skeleton$(G)$}
\end{tikzpicture}
\end{center}


%----------------------\skeletion defn 

%----------------------v-structures defn
\begin{definition} [v-structure]
A set of three vertices $X, Y, Z$ of a graph $G = (V,E)$ is called a \textbf{v-structure} (or immorality) iff  $(X, Y) \in E$ and $ (Z, Y) \in E$ but $X$ and $Z$ are not adjacent.
 
\begin{center}
\begin{tikzpicture}
\Vertex[x=-4.5, size=.5, color=white]{x1}
\Vertex[x = -2.5, size=.5, color=white]{x2}
\Vertex[y = -1.5, x=-3.5 ,size=.5,color=white]{x3}
\Edge[Direct, distance=0.5](x1)(x3)
\Edge[Direct, distance=0.5](x2)(x3)

\Vertex[size=.5, color=white]{x1}
\Vertex[x = 2, size=.5, color=white]{x2}
\Vertex[y = -1.5, x=1 ,size=.5,color=white]{x3}
\Edge[Direct, distance=0.5](x1)(x3)
\Edge[Direct, distance=0.5](x3)(x2)

\Vertex[x = 3,size=.5, color=white]{x1}
\Vertex[x = 5, size=.5, color=white]{x2}
\Vertex[y = -1.5, x=4, size=.5,color=white]{x3}
\Edge[Direct, distance=0.5](x3)(x1)
\Edge[Direct, distance=0.5](x3)(x2)

\Vertex[x = 6,size=.5, color=white]{x1}
\Vertex[x = 8, size=.5, color=white]{x2}
\Vertex[y = -1.5, x=7, size=.5,color=white]{x3}
\Edge[Direct, distance=0.5](x1)(x2)
\Edge[Direct, distance=0.5](x2)(x3)
\Edge[Direct, distance=0.5](x1)(x3)


\draw [decorate,decoration={brace,amplitude=8pt,mirror,raise=-5ex}]
  (-4.5,-3) -- (-2.5,-3) node[midway,yshift=.5em]{v-structure};
\draw [decorate,decoration={brace,amplitude=8pt,mirror,raise=-5ex}]
  (0,-3) -- (8,-3) node[midway,yshift=.5em]{not v-structures};
\end{tikzpicture}
\end{center}

\end{definition}
%----------------------\v-structures defn


\begin{lemma}[Criterion for Markov equivalence~\cite{verma}]\label{meccriterion} %verma theorem 1, pearl 1989 paper. 
Two graphs are equivalent iff they have the same skeleton and v-structures. 
\end{lemma}

\begin{proof}
A sketch of the proof is provided, as the complete proof involves many definitions and details which are not presented in this paper. First, we use the definition of an \textbf{active route} (sometimes called active path) with respect to a set $\mathbb{Z}$: a route $[X_{0}, X_{1}, ... X_{n}]$ in which all nodes $X_{i}$ are themselves \textbf{active}, meaning that (1) every middle node of a v-structure with respect to $\mathcal{Z}$ either is or has a descendant in $\mathcal{Z}$, and (2) every other node in the route is outside $\mathcal{Z}$. \newline
\null \quad \quad The proof depends on two lemmas presented in the same paper: \textbf{Lemma 1} has the consequence that adjacency of nodes is invariant among equivalent DAGs. \textbf{Lemma 2} has the consequence that one can determine the presence or absence of v-structures by observing certain separation properties of the graph alone, and vice versa. 
The two lemmas are unified with an inductive step to show that active paths in one DAG correspond to active paths in a Markov equivalent DAG. Furthermore, properties of active paths allow us to conclude that the DAGs must have the same skeleton and v-structures. 
\end{proof}

%----------------------\reversible edge defn
\begin{definition}[Irreversible edge~\cite{andersson}]
Let $G$ be a DAG. A directed edge $(X,Y) \in G$ is called \textbf{irreversible} in $G$ if changing $(X,Y)$ to $(Y,X)$ either creates or destroys a v-structure, or creates a directed cycle. An edge which is not irreversible is called \textbf{reversible}. 
\end{definition}
%----------------------\reversible edge defn

\begin{lemma}[Covered edges are exactly reversible edges~\cite{chickering}]\label{coveredrev} %lemma 1 in chickering
Let $G = (V,E)$ be a DAG, let $X,Y \in V$ and let $e = (X, Y) \in E$. Let $G' = (V,E')$ be the DAG constructed from $G$ by reversing the edge $(X, Y)$ to $(Y, X)$. Then $G$ and $G'$ are equivalent iff $e$ is a covered edge in $G$.
\end{lemma}

\begin{proof}
Let $G = (V,E)$ be a DAG. A sketch of the proof is as follows (see ~\cite{chickering} for the full proof):
	\begin{myindentpar}{1em}
	(\textbf{if}) Suppose $e = (X,Y)$ is a covered edge in $G$. Let $G'$ be equivalent to $G$, except that $e$ is reveresed to $e' = (Y,X)$. Then:
		\begin{itemize}
		\item $G'$ is also a DAG. To show this, suppose that $G'$ is not a DAG. Since $G$ and $G'$ only differ by $e' = (Y,X)$, there must be a directed cycle including $e'$ in $G'$. Then there is a path in $G$ from $Y$ to $X$. By assumption, $Y$ and $X$ have a shared set of parents. However, one can conclude from this that $G$ also contains a cycle because every cycle must include at least one node from $\prod\nolimits_{X}^{G}$, and therefore is not a DAG, a contradiction. By this contradiction, we conclude that $G'$ must be a DAG. 
		\item $G' \ME G$. Firstly, $G$ and $G'$ trivially have the same skeleton, since they only differ by the orientation of edge $e$. Then, if they were \textit{not} Markov equivalent graphs, they would differ by a v-structure. If $G'$ has a v-structure not present in $G$, it must include the edge $(Y,X)$. However, this would imply that $X$ has a parent which is not a neighbor to $Y$, contradicting the assumption that $e$ is covered. A similar argument holds for the scenario where $G$ has a v-structure not present in $G'$. 
		\end{itemize}
	
	(\textbf{only if}) We now show that if $e=(X,Y)$ is not a covered edge in $G$, we are in one of two cases: either $G'$ contains a directed cycle, or $G'$ is a DAG which is not equivalent to $G$. If $(X,Y)$ is not a covered edge in $G$, then at least one of the following hold:
	\begin{enumerate}
		\item There is a node $Z \neq X$ which is a parent of $Y$ but not of $X$. 
		\item There s a node $W$ which is a parent of $X$ but not of $Y$. 
	\end{enumerate}
	Let $Z \neq X$ be a parent of $Y$ in $G$ but not a parent of $X$ in $G$. If $Z$ and $X$ are not neighbors, then there is a v-structure consisting of edges $(X,Y)$ and $(Y,Z)$ in $G$ which does not exist in $G'$. If $X$ is a parent of $Z$ in $G$, then it must also be a parents of $Z$ in $G'$, which would imply that $G'$ contains a directed cycle and is therefore not a DAG. \newline
	\null \quad \quad The argument for the second case is equivalent, assuming $W$ is a parent of $X$ in $G$ and deriving a v-structure with $(W,X)$ and $(X,Y)$ which exists in $G'$ but not $G$. We conclude that $G$ would have to contain a directed cycle. In both scenarios, we have derived a contradiction which shows that $(X,Y)$ must be covered in $G$. \qedhere


\end{myindentpar}
\end{proof}


%------------------------------------------MEC threenode example
\begin{example}[Members of Markov equivalence class~\cite{verma}]\label{mecexample}
Consider the four DAGs in Figure~\ref{fig:sameMECseparateMEC}.

\begin{figure}[h!]
\begin{center}
\begin{tikzpicture}
\Vertex[x = 0, y=3, size=.5, color=white, label=$X$]{x}
\Vertex[x = 0, y=1.5, size=.5, color=white, label=$Y$]{y}
\Vertex[x=0, y = 0, size=.5,color=white, label=$Z$]{z}
\Edge[Direct, distance=0.5](x)(y)
\Edge[Direct, distance=0.5](y)(z)
\Text[x=-.8 ,y=3]{$G_{1}$}

\Vertex[x = 6, y=3, size=.5, color=white, label=$X$]{x}
\Vertex[x = 6, y=1.5, size=.5, color=white, label=$Y$]{y}
\Vertex[x=6, y = 0, size=.5,color=white, label=$Z$]{z}
\Edge[Direct, distance=0.5](z)(y)
\Edge[Direct, distance=0.5](y)(x)
\Text[x=5.2 ,y=3]{$G_{3}$}

\Vertex[x = 2, y=.8, size=.5, color=white, label=$X$]{x}
\Vertex[x = 4, y=.8, size=.5, color=white, label=$Z$]{z}
\Vertex[x=3, y = 2.2, size=.5,color=white, label=$Y$]{y}
\Edge[Direct, distance=0.5](y)(x)
\Edge[Direct, distance=0.5](y)(z)
\Text[x=2.1 ,y=1.8]{$G_{2}$}

\Vertex[x = 9, y=2.2, size=.5, color=white, label=$X$]{x}
\Vertex[x = 11, y=2.2, size=.5, color=white, label=$Z$]{z}
\Vertex[x=10, y = .8, size=.5,color=white, label=$Y$]{y}
\Edge[Direct, distance=0.5](x)(y)
\Edge[Direct, distance=0.5](z)(y)
\Text[x=9.1 ,y=1.25]{$G_{4}$}

\draw [decorate,decoration={brace,amplitude=8pt,mirror,raise=-5ex}]
  (-.8,-1.3) -- (6.8,-1.3) node[midway,yshift=.5em]{same MEC};
\draw [decorate,decoration={brace,amplitude=8pt,mirror,raise=-5ex}]
  (9,-1.3) -- (11,-1.3) node[midway,yshift=.5em]{separate MEC};
\end{tikzpicture}
\end{center}
\caption{}
\label{fig:sameMECseparateMEC}
\end{figure}

\null \quad \quad The fact that $G_{1}, G_{2}$, and $G_{3}$ are in the same MEC while $G_{4}$ is not can be seen by considering the joint probability distribution $p(X,Y,Z)$ of each of the graphs. 

\begin{myindentpar}{3em}
\begin{spacing}{1.25}
	$G_{1}$ : \quad $p(X,Y,Z) = p(X) \cdot p(Y|X) \cdot p(Z|Y)$ \newline
	$G_{2}$ : \quad $p(X,Y,Z) = p(Y) \cdot p(X|Y) \cdot p(Z|Y)$ \newline
	$G_{3}$ : \quad $p(X,Y,Z) = p(Z) \cdot p(Y|Z) \cdot p(X|Y)$
	\end{spacing}
\end{myindentpar}
By Bayes' theorem, all of the values of $G_{2}$ can be completely determined from values from $G_{1}$:
	$$
	p(X)p(Y|X) = p(X,Y) = p(Y)p(X|Y)
	$$
and $p(Z|Y)$ is unchanged. A similar application of Bayes' theorem shows that $G_{3}$ is completely determined by $G_{1}$ and $G_{2}$:
	$$
	p(z) \cdot p(Y|Z) = p(Z,Y) = p(Y) \cdot p(Z|Y)
	$$
where $p(Y)$ can be attained directly from $G_{2}$ and $p(Z|Y)$ can be attained from $G_{1}$. Therefore, we can conclude that since all three describe the same distribution, they lie in the same equivalence class. In contrast, the joint probability for $G_{4}$ is given by
	$$
	p(X,Y,Z) = p(X) \cdot p(Z) \cdot p(Y|X,Z)
	$$
which can not be determined from the values of $G_{1}, G_{2}$ and $G_{3}$. This is because in [$G_{1}$] %(but not in $G_{2}$ and $G_{3}$), 
$X$ is conditionally independent from $Y$ and $Z$, that is, $X$ is only independent in the circumstance that we are given values for $Y$ and $Z$. Meanwhile, in $G_{4}$, $X$ is marginally independent of $Z$, that is, independent in the circumstance that we ignore $Y$. Conditional independence does not provide information about marginal independence, and conversely, marginal independence does not imply conditional independence. Therefore, $G_{4}$ describes a different distribution, and does not lie in [$G_{1}$]. 

\end{example}
%------------------------------------------MEC threenode example


\begin{corollary}\label{coveredMEC}~\cref{meccriterion} and ~\cref{coveredrev} together imply that the graph $G'$ which results from only reversing the orientation of a covered edge $(X, Y) \in G$ is in the MEC of $[G]$. 
\begin{proof}
This follows from the fact that edge reversal does not change the skeleton of the graph $G$ (since no edges are added nor removed) and reversing covered edges does not alter the v-structures of $G$ (\cref{coveredrev}), thus the v-structures and skeleton are unchanged. 
\end{proof}
\end{corollary}



%----------------------------------------------------------------------------------------
%	Effects of Edge Reversal
%----------------------------------------------------------------------------------------
\section{Effects of Edge Reversal}


%---------------------------------------------- two node example

\begin{lemma}
Given a single directed edge between two parentless nodes in which the joint probability $p(X_{1},X_{2}) = p(X_{1}) \cdot p(X_{2}|X_{1})$, the new factorization of the joint probability distribution $p(X_{1},X_{2})$ can be computed as follows using only the known quantities $p(X_{1})$ and $p(X_{2}|X_{1})$:
$$
p(X_{1}, X_{2})\ \ = \ \ \sum_{X_{1}} p(X_{2}|X_{1}) \cdot p(X_{1}) \frac{ p(X_{2}|X_{1}) \cdot p(X_{1}) }{ \sum_{X_{1}} p(X_{2}|X_{1}) \cdot p(X_{1})}.
$$
\end{lemma}

\begin{figure}[h!]
\begin{center}
\begin{tikzpicture}
\Vertex[size=1, color=white, label=$X_{1}$]{x1}
\Vertex[x=4,size=1,color=white, label=$X_{2}$]{x2}
\Edge[Direct](x1)(x2)
\end{tikzpicture}

\scalebox{1.2}{$\Big\Downarrow$} 

\begin{tikzpicture}
\Vertex[size=1, color=white, label=$X_{1}$]{x1}
\Vertex[x=4,size=1,color=white, label=$X_{2}$]{x2}
\Edge[Direct](x2)(x1)
\end{tikzpicture}
\end{center}
\caption{}
\end{figure}

\begin{proof}
This follows directly from Bayes' theorem, which gives us
\begin{align*} 
	p(X_{1}| X_{2})	& \ \ = \ \ \frac{p(X_{2}|X_{1}) \cdot p(X_{1})}{p(X_{2})}	\\[2em]
				& \ \ = \ \ \frac{p(X_{2}|X_{1}) \cdot p(X_{1})}{\sum_{X_{1}} p(X_{2}|X_{1}) \cdot p(X_{1})}. 
\end{align*}
\end{proof}
%---------------------------------------------- \two node example


%---------------------------------------------- shared parent lemma

\begin{lemma}\label{sharedparent}
Given a directed edge between two vertices $X_{1}$ and $X_{2}$ with a shared parent vertex $X_{P}$, the joint probability $p(X_{1}, X_{2}, X_{p})$ after switching edge $(X_{1}, X_{2})$ to $(X_{2},X_{1})$ can be computed as follows using only known quantities $p(X_{p})$,  $p(X_{1}|X_{p})$, and \newline $p(X_{2}|X_{1},X_{p})$:
\begin{align*}
	p(X_{1},X_{2},X_{P}) & \ \ = \ \ 	[\frac{   p(X_{2}|X_{1},X_{p}) \cdot p(X_{1}|X_{p})   }{  \sum_{X_{1}} p(X_{1}|X_{p})\cdot p(X_{2}|X_{1},X_{p})  }  ] \\[1em] 
				& \quad \quad  \cdot [\sum_{X_{1}} p(X_{1}|X_{p})\cdot p(X_{2}|X_{1},X_{p})] \cdot p(X_{p}).
\end{align*}
\end{lemma}


\begin{figure}[h!]
\begin{center}
\begin{minipage}[t]{0.35\textwidth}
\begin{center}
\begin{tikzpicture}
\Vertex[x = 2, size=1, color=white, label=$X_{P}$]{xP}
\Vertex[x=4, y=-3,size=1,color=white, label=$X_{1}$]{x1}
\Vertex[x=0, y=-3,size=1,color=white, label=$X_{2}$]{x2}
\Edge[Direct, distance=0.5](x1)(x2)
\Edge[Direct, distance=0.5](xP)(x2)
\Edge[Direct, distance=0.5](xP)(x1)
\end{tikzpicture}
\end{center}
\end{minipage}
~
\begin{minipage}[t]{0.1\textwidth}
\begin{center}
\vspace{-7em}
\scalebox{1.5}{$\implies$} 
\end{center}
\end{minipage}
~
\begin{minipage}[t]{0.35\textwidth}
\begin{center}
\begin{tikzpicture}
\Vertex[x = 2, size=1, color=white, label=$X_{P}$]{xP}
\Vertex[x=4, y=-3,size=1,color=white, label=$X_{1}$]{x1}
\Vertex[x=0, y=-3,size=1,color=white, label=$X_{2}$]{x2}
\Edge[Direct, distance=0.5](x2)(x1)
\Edge[Direct, distance=0.5](xP)(x2)
\Edge[Direct, distance=0.5](xP)(x1)
\end{tikzpicture}
\end{center}
\end{minipage}
\end{center}
\caption{}
\end{figure}

	%---------------------------------------------- generalization of shared parent lemma
	
\begin{remark}
Furthermore, because this result does not depend on $p(X_{p})$, we can conclude that for a set of shared parents $P = \{X_{P_{1}},X_{P_{2}},\ ...\ X_{P_{i}}\}$ of two nodes $X_{1}$ and $X_{2}$, the same equality holds. Additionally, note that any reversal of an edge with a set of shared parents is MEC-invariant, since all such edges are covered (~\cref{coveredMEC}).
\end{remark}

\begin{figure}[h!]
\begin{center}
\begin{tikzpicture}
\Vertex[x = -1, size=1, color=white, label=$X_{P_{1}}$]{xP1}
\Vertex[x = .5, size=1, color=white, label=$X_{P_{2}}$]{xP2}
\Vertex[x = 5, size=1, color=white, label=$X_{P_{i}}$]{xPi}

\Text[y=.15, x=2.75]{$\textbf{   .\ \    .\ \    .  }$}

\Vertex[x=4, y=-3,size=1,color=white, label=$X_{1}$]{x1}
\Vertex[x=0, y=-3,size=1,color=white, label=$X_{2}$]{x2}

\Edge[Direct, distance=0.5](x2)(x1)

\Edge[Direct, distance=0.5](xP1)(x2)
\Edge[Direct, distance=0.5](xP1)(x1)

\Edge[Direct, distance=0.5](xP2)(x2)
\Edge[Direct, distance=0.5](xP2)(x1)

\Edge[Direct, distance=0.5](xPi)(x2)
\Edge[Direct, distance=0.5](xPi)(x1)
\end{tikzpicture}
\end{center}
\caption{}
\end{figure}
	
	%----------------------------------------------\generalization of shared parent lemma


\begin{proof}
From before the edge reversal, we have access to the values $p(X_{p}), p(X_{1}|X_{p})$, and $p(X_{2}|X_{1},X_{p})$. Our goal is to determine the new joint probability distribution $p(X_{1}, X_{2}, X_{p})$ using this information. Therefore, we must find $p(X_{p}), p(X_{2}|X_{p})$, and $p(X_{1}|X_{2},X_{p})$. First, $p(X_{p})$ is already trivially available. Then, to compute $p(X_{1}|X_{2},X_{p})$,

\begin{align*}
p(X_{1}|X_{2},X_{p}) & \ \ = \ \  \frac{ p(X_{1}, X_{2}, X_{p})  }{ p(X_{2},X_{p})  } \\[1em]
			    & \ \ = \ \  \frac{ p(X_{2}|X_{1},X_{p}) \cdot p(X_{1},X_{p}) }{ p(X_{2},X_{p}) } \\[1em]
			    & \ \ = \ \  \frac{ p(X_{2}|X_{1},X_{p}) \cdot p(X_{1}|X_{p}) \cdot p(X_{p}) }{ p(X_{2}|X_{1},X_{p}) } \\[1em]
			    & \ \ = \ \  \frac{ p(X_{2}|X_{1},X_{p}) \cdot p(X_{1}|X_{p}) }{ p(X_{2}|X_{p}) }
\end{align*}

which depends only on known quantities once we compute

\begin{align*}
p(X_{2}|X_{p}) 	& \ \ = \ \ \frac{p(X_{2},X_{p})}{p(X_{p})} \\[1em]
			& \ \ = \ \ \frac{1}{p(X_{p})} \sum_{X_{1}} p(X_{1},X_{2},X_{p}) \\[1em]
			& \ \ = \ \ \frac{1}{p(X_{p})} \sum_{X_{1}} p(X_{1},X_{p}) \cdot \frac{p(X_{2},X_{1}, X_{p})}{p(X_{1},X_{p})} \\[1em]
			& \ \ = \ \ \sum_{X_{1}} p(X_{1}|X_{p}) \cdot p(X_{2}|X_{1},X_{p}).
\end{align*}

Together, these give us 
\begin{align*}
p(X_{1},X_{2},X_{p})	& \ \ = \ \ p(X_{1}|X_{2},X_{p}) \cdot p(X_{2}|X_{p}) \cdot  p(X_{p}) \\[1em]
				& \ \ = \ \    
				[\frac{   p(X_{2}|X_{1},X_{p}) \cdot p(X_{1}|X_{p})   }{  \sum_{X_{1}} p(X_{1}|X_{p}) \cdot p(X_{2}|X_{1},X_{p})  }  ] \\[1em] 
				&  \quad \quad \cdot [\sum_{X_{1}} p(X_{1}|X_{p})\cdot p(X_{2}|X_{1},X_{p})] \cdot p(X_{p}).
\end{align*}
\end{proof}

\begin{remark}
Notice that the above cancels to show that the two graphs are Markov equivalent, since $p(X_{1}|X_{2},X_{p}) \cdot p(X_{2}|X_{p}) \cdot p(X_{p})=  p(X_{2}|X_{1},X_{p}) \cdot p(X_{1}|X_{p}) \cdot p(X_{p}).$
\end{remark}

%---------------------------------------------- \shared parent lemma



%----------------------------------------------------------------------------------------
%	Find Reversible Edges of a DAG Model
%----------------------------------------------------------------------------------------
\section{Identifying Reversible Edges}


\begin{definition}[Essential edge~\cite{andersson},~\cite{flesch}] %definition 10 flesch, definition 2.1 Andersson
An \textbf{essential edge} (also called a compelled edge~\cite{chickering}) of a graph $G = (V,E)$ is an edge $(X,Y) = e \in E$ such that for every graph $G' = (V,E')$ in $[G]$, the orientation of $e \in E$ is unchanged. That is, there is no graph $G' = (V,E')$ in $[G]$ such that $(Y,X) \in E'.$
\end{definition}

% ---------------------essential graph defn
\begin{definition} [Essential graph~\cite{andersson}]
The \textbf{essential graph} of a Markov equivalence class $[G]$ is a mixed graph $G_{*}$ such that the only directed edges in $G_{*}$ are essential edges of $[G]$. That is,
\begin{itemize}
	\item The directed edge $ e = (X, Y) \in G_{*}$ iff  $e \in G$ for every graph $G \in [G]$,
	\item The undirected edge ${X,Y} \in G_{*}$ iff there are graphs $G_{1}$ and $G_{2} \in [G]$ such that $(X, Y)$ in $G_{1}$ and $(Y, X)$ in $G_{2}$. 
\end{itemize}
\end{definition}
% --------------------\-essential graph defn


% ---------------------essential graph example
\begin{example}\label{essentialgraphexample} In this example, we identify the essential edges of a Markov equivalence class and construct the essential graph. Consider the following DAG $G = (V,E)$:

\begin{figure}[h!]
\begin{center}
\begin{tikzpicture}
\Vertex[x = 2, y=0, size=.5, color=white, label=$Y$]{y}
\Vertex[x = 4, y=0, size=.5, color=white, label=$Z$]{z}
\Vertex[x=3, y = 1.4, size=.5,color=white, label=$X$]{x}
\Edge[Direct, distance=0.5](x)(y)
\Edge[Direct, distance=0.5](y)(z)
\Edge[Direct, distance=0.5](x)(z)
\Text[x=3 ,y=-.5]{$G$}
\end{tikzpicture}
\end{center}
\caption{}
\end{figure}

We use the following realizations to show that the graph has no essential edges. If we switch any edge in $G$, the resulting graph $G'$ becomes one of two cases:
\begin{itemize}
	\item $G'$ is cyclic, and therefore no longer a DAG, meaning this edge was irreversible. 
		\newline \null \quad Example: reversing edge $(X,Z)$ as in \textbf{(A)} creates a cycle.

		\begin{center}
		\begin{tikzpicture}
		
			\Vertex[x = 1, y=0, size=.5, color=white, label=$Y$]{y}
			\Vertex[x = 3, y=0, size=.5, color=white, label=$Z$]{z}
			\Vertex[x=2, y = 1.4, size=.5,color=white, label=$X$]{x}
			\Edge[Direct, distance=0.5](x)(y)
			\Edge[Direct, distance=0.5](y)(z)
			\Edge[Direct, distance=0.5](z)(x)
			\Text[x=2 ,y=-1]{\textbf{(A)}}

		\end{tikzpicture}
		\end{center} 
			
	\item $G'$ is equivalent to $G$ up to relabeling, and therefore in the same MEC.
		\newline \null \quad Example: reversing $(X,Y)$ and changing labels according to $X \rightarrow Y',\  Y\rightarrow Z',$  and $Z \rightarrow X'$, as in \textbf{(B)}.
\end{itemize} 

\begin{center}
\begin{tikzpicture}

\Vertex[x = 6, y=0, size=.5, color=white, label=$Y$]{y}
\Vertex[x = 8, y=0, size=.5, color=white, label=$Z$]{z}
\Vertex[x=7, y = 1.4, size=.5,color=white, label=$X$]{x}
\Edge[Direct, distance=0.5](y)(x)
\Edge[Direct, distance=0.5](y)(z)
\Edge[Direct, distance=0.5](x)(z)

\Text[x=9 ,y=-1]{\textbf{(B)}}
\Text[x=9 ,y=.7]{\large $\implies$}
\Text[x=9 ,y=1.2]{relabeling}

\Vertex[x = 10, y=0, size=.5, color=white, label=$Z'$]{z'}
\Vertex[x = 12, y=0, size=.5, color=white, label=$X'$]{x'}
\Vertex[x=11, y = 1.4, size=.5,color=white, label=$Y'$]{y'}
\Edge[Direct, distance=0.5](y')(z')
\Edge[Direct, distance=0.5](z')(x')
\Edge[Direct, distance=0.5](y')(x')
\end{tikzpicture}
\end{center}

Thus the following graphs are included in [$G$], though they are not the entire MEC. The red edges in each are those with orientation which differs from edges in $G$. Notice, however, that some relabeling was required, meaning these edges do not necessarily correspond (under this new labeling) to covered edges.

\begin{center}
\begin{tikzpicture}
\Vertex[x = 2, y=0, size=.5, color=white, label=$Y'$]{y}
\Vertex[x = 4, y=0, size=.5, color=white, label=$Z'$]{z}
\Vertex[x=3, y = 1.4, size=.5,color=white, label=$X'$]{x}
\Edge[Direct, distance=0.5, color=red](y)(x)
\Edge[Direct, distance=0.5](y)(z)
\Edge[Direct, distance=0.5](x)(z)
\Text[x=3 ,y=-.5]{$G'_{1}$}

\Vertex[x = 6, y=0, size=.5, color=white, label=$Y'$]{y}
\Vertex[x = 8, y=0, size=.5, color=white, label=$Z'$]{z}
\Vertex[x=7, y = 1.4, size=.5,color=white, label=$X'$]{x}
\Edge[Direct, distance=0.5, color=red](z)(x)
\Edge[Direct, distance=0.5](y)(z)
\Edge[Direct, distance=0.5, color=red](y)(x)
\Text[x=7 ,y=-.5]{$G'_{2}$}

\Vertex[x = 10, y=0, size=.5, color=white, label=$Y'$]{y}
\Vertex[x = 12, y=0, size=.5, color=white, label=$Z'$]{z}
\Vertex[x=11, y = 1.4, size=.5,color=white, label=$X'$]{x}
\Edge[Direct, distance=0.5](x)(z)
\Edge[Direct, distance=0.5, color=red](z)(y)
\Edge[Direct, distance=0.5](x)(y)
\Text[x=11 ,y=-.5]{$G'_{3}$}
\end{tikzpicture}
\end{center}

Therefore, under some circumstance, for every edge $(U,V) = e \in G$, there is a DAG $G' \in [G]$ such that $(V,U) \in G'$. This means that $G$ has no essential edges. We construct the essential graph $G_{*}$ of $G$ by removing the orientation of all non-essential edges in $G$: 

\begin{center}
\begin{tikzpicture}
\Vertex[x = 6, y=0, size=.5, color=white, label=$Y$]{y}
\Vertex[x = 8, y=0, size=.5, color=white, label=$Z$]{z}
\Vertex[x=7, y = 1.4, size=.5,color=white, label=$X$]{x}
\Edge[Direct, distance=0.5](x)(y)
\Edge[Direct, distance=0.5](y)(z)
\Edge[Direct, distance=0.5](x)(z)
\Text[x=7 ,y=-.5]{$G$}
\Text[x=9 ,y=.7]{\large $\implies$}


\Vertex[x = 10, y=0, size=.5, color=white, label=$Z'$]{y}
\Vertex[x = 12, y=0, size=.5, color=white, label=$X'$]{z}
\Vertex[x=11, y = 1.4, size=.5,color=white, label=$Y'$]{x}
\Edge[distance=0.5](x)(y)
\Edge[distance=0.5](y)(z)
\Edge[distance=0.5](x)(z)
\Text[x=11 ,y=-.5]{$G_{*}$}
\end{tikzpicture}
\end{center}

\end{example}

% ---------------------\essential graph example


\begin{remark}
Andersson et. al.~\cite{andersson} demonstrate that essential graphs are consistent across a MEC and unique to it, meaning an essential graph can be used as a representative for a given MEC. Furthermore, essential graphs are not necessarily DAGs, and therefore do not generally belong to the MEC that they represent. \newline
\null \quad \quad This can be seen by the fact that the distinction between two MECs lies entirely in either differences in their skeletons or in the orientations of their compelled edges. \newline

\textit{``It is only in the heart that one can see rightly; what is essential is invisible to the eye.'' -Antoine de Saint-Exupéry, The Little Prince.}
\end{remark}


%--------------------\essential vs covered edge disctinction example
\begin{remark}
One should take care to note the distinction between implications about non-essential edges versus covered edges: 
\begin{itemize}
	\item For $G = (V,E)$, an edge $e \in E$ is non-essential iff there exists a graph $G'$ in $[G]$ such that $e$ is reversed,
	\item An edge $e$ in $G$ is covered iff the graph $G'$ derived from \textit{only} reversing $e$, is in the MEC of $G$. (Follows from ~\cref{coveredMEC}.)
\end{itemize}
That is, reversing $e$ \textit{exclusively} while remaining in the MEC of $G$ requires $e$ to be covered. Meanwhile, a non-essential edge $e \in G$ may require other edges in $G$ to be reversed as well to remain in the same MEC. This means that not all non-essential edges can be reversed at a given time. The following example demonstrates that some configurations of non-essential edge reversals are not possible if we wish to remain in the MEC of $G$.
\end{remark}

\begin{example}
Again consider members of the Markov equivalence class seen in ~\cref{mecexample}. The positions of the nodes are visually rearranged for clarity, but no properties are changed. \newline
\begin{figure}[h!]
\begin{center}
\begin{tikzpicture}
\Vertex[x = 0, y=3, size=.5, color=white, label=$X$]{x}
\Vertex[x = 0, y=1.5, size=.5, color=white, label=$Y$]{y}
\Vertex[x=0, y = 0, size=.5,color=white, label=$Z$]{z}
\Edge[Direct, distance=0.5](x)(y)
\Edge[Direct, distance=0.5](y)(z)
\Text[x=-.8 ,y=3]{$G_{1}$}

\Vertex[x = 3, y=3, size=.5, color=white, label=$X$]{x}
\Vertex[x = 3, y=1.5, size=.5, color=white, label=$Y$]{y}
\Vertex[x=3, y = 0, size=.5,color=white, label=$Z$]{z}
\Edge[Direct, distance=0.5](y)(x)
\Edge[Direct, distance=0.5](y)(z)
\Text[x=2.2 ,y=3]{$G_{2}$}

\Vertex[x = 6, y=3, size=.5, color=white, label=$X$]{x}
\Vertex[x = 6, y=1.5, size=.5, color=white, label=$Y$]{y}
\Vertex[x=6, y = 0, size=.5,color=white, label=$Z$]{z}
\Edge[Direct, distance=0.5](z)(y)
\Edge[Direct, distance=0.5](y)(x)
\Text[x=5.2 ,y=3]{$G_{3}$}

\Vertex[x = 10, y=3, size=.5, color=white, label=$X$]{x}
\Vertex[x = 10, y=1.5, size=.5, color=white, label=$Y$]{y}
\Vertex[x=10, y = 0, size=.5,color=white, label=$Z$]{z}
\Edge[Direct, distance=0.5](z)(y)
\Edge[Direct, distance=0.5](x)(y)
\Text[x=9.2 ,y=3]{$G_{4}$}

\draw [decorate,decoration={brace,amplitude=8pt,mirror,raise=-5ex}]
  (-.8,-1.3) -- (6.8,-1.3) node[midway,yshift=.5em]{same MEC};
\draw [decorate,decoration={brace,amplitude=8pt,mirror,raise=-5ex}]
  (9,-1.3) -- (11,-1.3) node[midway,yshift=.5em]{separate MEC};
\end{tikzpicture}
\end{center}
\caption{}
\end{figure}
\null \quad \quad Notice that edge $(Y,Z) \in$  $G_{1}$ is non-essential, since there exists another graph (namely $G_{3}$) in  [$G_{1}$] such that $(Z,Y) \in$ $G_{3}$. Therefore, we know that there exists some combination of edge reversals such that $(Y,Z)$ can be reversed without leaving [$G_{1}$]. However, if we choose to switch only $(Y,Z)$ to $(Z,Y)$, we have exactly graph $G_{4}$, which is no longer in the same MEC. Therefore, the fact that an edge is non-essential does not provide information about the circumstances in which the edge can be reversed without altering the MEC; only that in some context, it can be. This realization leads us to ~\cref{essentialsubset}.
\end{example}
%--------------------\essential vs covered edge distinction example

%--------------------covered is subset of essential
\begin{corollary}\label{essentialsubset}
Let $E_{c}$ be the set of covered edges in $G$, and let $E_{e}$ be the set of essential edges in $G$. Then $E_{c} \subseteq \bar{E_{e}}$, the set of non-essential edges in $G$.
\begin{proof}
This follows directly from the realization that non-essential edges can be switched while remaining in the same MEC under the correct conditions, while covered edges can be switched immediately, without regard to other edge conditions (the conditions for their reversal are always met). 
\end{proof}
\end{corollary}

\begin{example}
We use the graph $G$ from Example \ref{essentialgraphexample} to demonstrate Corollary \ref{essentialsubset}. Of course, this is not sufficient for a proof, but rather helps to contextualize the lemma. 

\begin{figure}[h!]
\begin{center}
\begin{tikzpicture}
\Vertex[x = 2, y=0, size=.5, color=white, label=$Y$]{y}
\Vertex[x = 4, y=0, size=.5, color=white, label=$Z$]{z}
\Vertex[x=3, y = 1.4, size=.5,color=white, label=$X$]{x}
\Edge[Direct, distance=0.5](x)(y)
\Edge[Direct, distance=0.5](y)(z)
\Edge[Direct, distance=0.5](x)(z)
\Text[x=3 ,y=-.5]{$G$}
\end{tikzpicture}
\end{center}
\caption{}
\end{figure}

 In~\cref{essentialgraphexample} we determined that all edges $e_{i} \in E$ of $G=(V,E)$ are non-essential, so the set of non-essential edges $E_{G}^{N} = \{(X,Y), (Y,Z), (X,Z) \}$. We can quickly see that the only covered edge of $G$ is $(Y,Z)$ since $Y$ and $Z$ share a single parent $X$, not including $Z$ itself. Then, indeed, $\{(Y,Z) \} \subset E_{G}^{N}$.

\end{example}
%--------------------\covered is subset of essential


% ---------------------reversible edge conceptions
\begin{remark} There are two scenarios we consider for finding reversible edges, that is, edges which can be reversed without altering the MEC of the graph:
\begin{itemize} 
\item Firstly, we may wish to find the set of edges which can be immediately reversed in one step, without regard to other edges in the graph. This coincides with the set of covered edges. 
\item Secondly, we may wish to find the set of edges which can ever be reversed within a graph, under the correct conditions of other edges being reversed beforehand/simultaneously. This coincides with the set of non-essential edges.
 \end{itemize}
\end{remark}
% --------------------\-reversible edge conceptions

%---------------------essential vs covered edge diagram
\begin{figure}[h!]
\begin{center}
\begin{tikzpicture}[scale=0.7]
	\draw[thick] (1cm,0) circle (36pt);
	\draw[thick] (4,0) circle (36pt);
	\draw[thick] (4.15,-.2) circle (18pt);
	
	\draw[red, thick] (1cm+-30.4pt,-19pt)--(1cm+-64pt,-39pt);
	\draw[red, thick] (4cm + 32.4pt,-16pt)--(4cm+64pt,-39pt);
	\draw[red, thick] (4.15cm + 15pt, 5pt)--(4.15cm+65, 33pt);
	
	\node [left] at (1cm+-64pt, -39pt) {\begin{tabular}{r} essential edges \\ (cannot be reversed) \end{tabular}};
	\node [right] at (4cm+64pt, -39pt) {\begin{tabular}{l} non-essential edges \\ (conditionally reversible) \end{tabular}};
	\node [right] at (4.15cm+65, 33pt) {\begin{tabular}{l} covered edges \\ (immediately reversible) \end{tabular}};
\end{tikzpicture}
\end{center}
\caption{}
\end{figure}

%---------------------\essential vs covered edge diagram






\begin{theorem}[Edge reversal sequence.~\cite{chickering}]\label{reversalsequence}
Given two Markov equivalent DAGs $G$ and $G'$, there exists a sequence of distinct edge reversals in $G$ with the following properties:
\begin{itemize}
	\item Each edge reversed in $G$ is a covered edge,
	\item After each reversal, $G$ is a DAG and $G \ME G'$
	\item After all reversals, $G = G'$. 
\end{itemize}
That is, we can transform $G$ into $G'$ via a finite sequence of covered edge reversals, such that all of the intermittently resulting graphs $G_{1}, G_{2},...G_{n} \ME G, G'$. 
\end{theorem}

\begin{proof}
Chickering's ~\cite{chickering} proof defines the procedure \textbf{Procedure Find-Edge} which takes a DAG $G$ as input. At each step, the procedure identifies a next edge to reverse. To be reversible, such an edge must be covered, by \cref{coveredrev}. The proof demonstrates that such an edge is identifiable when $G$ and $G'$ remain unequal, and each reversal reduces the number of edges in total which must be reversed, eventually transforming $G$ into $G'$. 
\end{proof}

%---------------------\Algorithm for finding covered edges






%%%%%%%%%POSSIBLE THINGS TO ADD BELOW%%%%%%%%%%%%%%%%%%%



%----------------------------------------------------------------------------------------
%	Constructing the Essential Graph of a DAG Model
%----------------------------------------------------------------------------------------
%\section{Essential Graph of a DAG Model}
%
%We quickly see that construction of the essential graph $G_{*}$ of a DAG model $G$ is quite complicated, as it involves identifying all essential edges of $G$, which in turn involves having access to or a characterization of all DAGs in the MEC $[G]$. Andersson et. al.~\cite{andersson} have done significant work to characterize all DAGs which lie within a given MEC, and have presented a polynomial time algorithm (The Construction Algorithm) to construct the unique essential graph associated with a MEC. The algorithm does not require and exhaustive search over the entire MEC. \newline
%\null \quad \quad A quick overview of the arguments made by Andersson et al. is as follows: 
%
%%ANDERSSON DESCRIPTION
%
%
%\begin{algorithm}[h!]\label{alg:constructionalgorithm}
%\DontPrintSemicolon
%\KwIn{A directed acyclic graph $G = (V,E)$}
%\KwOut{The essential graph $G_{*}$ of $G$} 
%\begin{spacing}{1.25}
%
%Define $G_{0} \coloneqq G$.\;
%For $i \geq 1$, convert every directed edge $(x,y) \in G_{i-1}$ that is not strongly protected in $G_{i-1}$ into an undirected edge $x \textemdash y$, obtaining a graph $G_{i}$. Stop after $k$ steps, where $k \geq 0$ is the smallest nonnegative integer such that $G_{k} = G_{k+1}$. Necessarily, $k \leq |E|$ \;
%This algorithm produces a sequence $G \ME G_{0} \subset ... \subset G_{k} = G_{k+1}$.
%\end{spacing}
%
%\caption{{\sc The Construction Algorithm} returns the essential graph of a DAG}
%\label{alg:constructionalgorithm}
%\end{algorithm}


%----------------------------------------------------------------------------------------
%	Random interesting things
%----------------------------------------------------------------------------------------

%\section{Random interesting things}
%\begin{theorem}[Parents of equivalent graphs~\cite{chickering}] %theorem 9 in chickering
%Let $G$ and $G'$ be any pair of Markov equivalent DAGs. If $G$ has a node with exactly $k$ parents, then $G'$ has a node with exactly $k$ parents. 
%\end{theorem}



%%%%%%%%%%%%%%% CHAPTER 5 EXPLOITING MEC
%Greedy strategy
\newpage
\chapter{Exploiting Markov Equivalence using a Greedy Strategy}

%%%%%%%%%%%%%%%%%%%%%%%
\section{Motivation}
%%%%%%%%%%%%%%%%%%%%%%%

\null \quad \quad One complication of minimizing the cost of a query over a Markov equivalence class is that Markov equivalence classes can be quite large~\cite{MECsize1, MECsize2}. Constructing all members of the class then searching for a minimal graph among those members may be very costly. \newline
\null \quad \quad Additionally, searching for covered edges (edges reversible at a given moment) is a significantly simpler task than searching for essential edges (edges which can be reversed under the correct circumstances, but not necessarily at a given moment). This is because looking for covered edges only requires considering one vertex at a time, whereas searching for essential edges requires looking at sets of edges which may have to be reversed simultaneously. One can take advantage of these circumstances by searching for an optimal single edge reversal over the set of reversible edges in a graph $G$. \newline
\null \quad \quad We therefore develop the greedy strategy, an algorithm in which each step entails searching for a single edge that can be reversed without leaving the original Markov equivalence class, while simultaneously reducing the number of vertices which must be computed to answer our query $q$ on $G$, written $|\Delta(G,q)|$. \newline
\null \quad \quad There are two aspects we wish to consider while searching for a reversible edge which reduce the cost of answering a query:
	\begin{itemize}
	\item Does reversing the edge alter existing v-structures in the graph, either by creating a new v-structure or destroying an existing one? 
	\item Does switching the edge indeed benefit our query, i.e., reduce the number of variables that must be considered while answering the query?
	\end{itemize}
We therefore present relevant background, create a procedure to identify reversible edges, develop a framework for when edge switching is beneficial, and analyze the efficacy of the procedure for minimization. 

%%%%%%%%%%%%%%%%%%%
\section{Background and Construction}\label{section:bgandconstruction}
%%%%%%%%%%%%%%%%%%%
\null \quad \quad In this section, we present the algorithms involved in the greedy strategy; each individual component is described and shown, however the complexities will not be discussed until later. 

\subsection{Finding Reversible Edges}
\null \quad \quad The greedy algorithm takes advantage of~\cref{coveredrev}, which states that reversible edges are exactly covered edges. Therefore, to search for reversible edges, we simply search for covered edges. We define the following algorithm to search for covered edges using~\cref{defcoverededge} directly, which tells us that covered edges $(X_{1},X_{2})$ have the same set of parents excluding $X_{1}$ itself.

\begin{algorithm}[h!]\label{findcoverededges}
\DontPrintSemicolon
\KwIn{A directed acyclic graph $G$}
\KwOut{The set of covered edges in $G$} 
\begin{spacing}{1.25}

covered\_edges $= []$ \;
\For {$e = (X_{1}, X_{2}) \in Edges(G)$} {
	\uIf {$\prod_{X_{1}}^{G} = \prod_{X_{2}}^{G} \textbackslash X_{1} $}{
		add $e$ to covered\_edges
	}
  }
\Return{covered\_edges}\;
\end{spacing}

\caption{{\sc Covered edges} returns a set of covered edges in a DAG}
\label{algo:coverededges}
\end{algorithm}

\subsection{Determining the Benefit of a Reversal}
\null \quad \quad Once we have identified the set of reversible edges, we must determine whether a switch is advantageous. Let $G=(V,E)$ be a DAG with $(X_{1},X_{2}) \in E$, and let $G'$ be the DAG achieved by reversing $(X_{1},X_{2})$. Then, to determine whether reversing $(X_{1},X_{2})$ is advantageous given a query $q$, we compare $|\Delta(G,q)|$ to $|\Delta(G',q)|$. If the latter is smaller, we have reduced the number of vertices the computation of $q$ depends on, and therefore the reversal is advantageous. \newline
\null \quad \quad Then, we can determine which edge reversal is most advantageous by searching for $max_{G'}(|\Delta(G,q)| - |\Delta(G',q)|)$ where, without loss of generality, $G'$ is the DAG achieved by reversing a single covered edge in $G$. 
%%%%%%%%%%%%%%%%%
\begin{algorithm}[h!]
\DontPrintSemicolon
\KwIn{A directed acyclic graph $G$, a query $q$}
\KwOut{The most advantageous possible edge reversal in $G$} 
\begin{spacing}{1.25}

edge\_advantages $=\{\}$ \;
\For {$e$ $\in$ covered\_edges($G$) } {
	$G' = G$ with $e$ reversed 
	edge\_advantages[$e$] = $|\Delta(G,q)| - |\Delta(G',q)|$ \;
  }
best\_reversal = $e$ such that edge\_advantages[$e$] = max(values in edge\_advantages) \;
\Return{best\_reversal}\;
\end{spacing}

\caption{{\sc Is Advantageous} returns the most advantageous edge reversal over $G$ given a query}
\label{algo:isadv}
\end{algorithm}
%%%%%%%%%%%%%%%%%%%%%%

\subsection{Identifying the vertices a query depends on}
\null \quad \quad Notice that Algorithm~\ref{algo:isadv} requires algorithmically determining the set $\Delta(G,q)$. To do so involves two subtasks: first, we must determine which case (either special or general) each condition in the query is in. Then, once all conditional vertices have been classified, we must add vertices accordingly to the set $\Delta(G,q)$. \newline
\null \quad \quad There are at least two possible methods by which one can determine which case a vertex is in. These correspond to the two ways of framing the cases, namely~\cref{thm:querycostonetarget} and~\cref{lemma:alternativecases}: \newline 
\null \quad \quad \cref{thm:querycostonetarget} suggests that we approach the cases by considering paths. In this work, we will use this method exclusively. Our goal is to determine whether, for a given condition $Y$, a target $X$ is conditionally independent of $\alpha(Y)$ given $Y$. To do so, we check all route from $X$ to each element in $\alpha(Y)$, calling these $Y_{\alpha_{i}},\  i \in [1,|\alpha(Y)|]$. If there exists a route from $X$ to some $Y_{\alpha_{i}}$ which does not pass through $Y$, we know that $X$ is not independent of $\alpha(Y)$ given $Y$. This would mean that $Y$ is in the general case.  If all paths from $X$ to $Y_{\alpha_{i}}$ pass through $Y$, then $\alpha(Y)$ is independent of $X$ given $Y$, and therefore $Y$ is in the special case. \newline
\null \quad \quad The alternative method corresponding to \cref{lemma:alternativecases} involves creating a copy of the graph $G$ called $G'$ in which the vertex $Y$ is removed from $G'$, and checking whether 
\begin{enumerate}
\item $G'$ is made up of two disjoint subgraphs $G_{1}$ and $G_{2}$,
\item $X \in G_{1}$ and all elements in $\alpha(Y) \in G_{2}$. 
\end{enumerate}
If both of the above are true, then $X$ is independent from $\alpha(Y)$ given $Y$, and therefore $Y$ is in the special case. Otherwise, $Y$ is in the general case. \newline
\null \quad \quad The method outlined by \cref{thm:querycostonetarget} is presented in Algorithm~\ref{algo:findcase}:

%%%%%%%%%%%%%%%%%
\begin{algorithm}[h!]
\DontPrintSemicolon
\KwIn{A DAG $G$, a target $X$, a condition $Y$}
\KwOut{The case of condition $Y$ with respect to target $X$ in $G$} 
\begin{spacing}{1.25}

$G'$ = skeleton of $G$ \#to ensure routes, not directed paths \;
\textit{paths} = all simple paths from $X$ to $Y$ in $G'$ (no repeated vetices)\;
\uIf {$|\prod_{G}^{Y}| = 0$} {
	\textit{case} = `special' \;
	\Return{case}
}
\For {\textit{path} in \textit{paths}} {
	\For {ancestor in $\alpha(Y)$} {
		\uIf {ancestor $\in$ path} {
			\textit{case} = `general' \;
			\Return{case} \;
		}
		\uElse {} {
		continue \;
		}
	}
}
\#only reaches this block if a case has not already been established \;
\textit{case} = `special' \;
\Return{case}\;

\end{spacing}
\caption{{\sc Find case} returns the case of a single condition $Y$ with respect to a target $X$, either Special or General}
\label{algo:findcase}
\end{algorithm}

%%%%%%%%%%%%%%%%%%

\null \quad \quad The logic of the general case block is as follows: if any path from $X$ to $Y$ passes through an element in $\alpha(Y)$, then there is a back-route to $Y$ through its ancestors, meaning there is a path connecting $X$ to $\alpha(Y)$ which does not pass through $Y$.\newline
\null \quad \quad This algorithm is applied repeatedly until the cases of all conditions $Y$ have been determined, at which point the appropriate vertices can be added to $\Delta(G,q)$ accordingly. This repeated application and collection of relevant vertices is handled by Algorithm~\ref{algo:createdeltaset}. \newline 


%%%%%%%%%%%%%%%%%
\begin{algorithm}[h!]
\DontPrintSemicolon
\KwIn{A DAG $G$, a target $X$, a set of conditions $Ys$ which correspond to a query $q  =p(X|Ys)$}
\KwOut{$\Delta(G,q)$} 
\begin{spacing}{1.25}

$G'$ = skeleton of $G$ \#to ensure routes, not directed paths \;
$\Delta(G,q)$ = [] \;
\textit{cases} = \{\} \;
\uIf {len(Ys) = 0} {
	$\Delta(G,q)$ = $alpha(X)$ \;
	\Return{$\Delta(G,q)$}
}
Add $X$ and $\alpha(X)$ to $\Delta(G,q)$ \;
\For {$Y$ is $Ys$} {
	\textit{shortest\_path} = the shortest path from $X$ to $Y$ over $G'$ \;
	\textit{cases}[Y] = find\_case(G,X,Y) \;
	\uIf {$Y$ is in the general case} {
		\For {$Y_{\alpha}$ in $\alpha(Y)$} {
			add $Y_{\alpha}$ to $\Delta(G,q)$ \;
		}
	\uIf {$Y$ is in the special case} {
		\For {\textit{node} in \textit{shortest\_path}} {
			add \textit{node} to $\Delta(G,q)$ \;
			}
		Remove $Y$ from $\Delta(G,q)$ \;
		Remove $\alpha(Y)$ from $\Delta(G,q)$ \;
		}
	}
}

\Return {$\Delta(G,q)$ with no repeating values}


\end{spacing}
\caption{{\sc Create Delta Set} returns the set $\Delta(G,q)$.}
\label{algo:createdeltaset}
\end{algorithm}




%%%%%%%%%%%%%%%%%%






%%%%%%%%%%%%%%%%%%%%%%
\section{Complexity of the Greedy Strategy}
%%%%%%%%%%%%%%%%%%%%%%%

\null \quad \quad In this section, we will find the complexity of each of the algorithms involved in our greedy strategy. Ultimately, the goal is to compare the costs of identifying and enacting a beneficial transformation versus computing the query without transforming the graph. To do so, we will compare the complexity of the following: 
\begin{enumerate}
\item Computing a query $q$ on a DAG $G$,
\item Identifying a beneficial edge reversal $e \in E$ which can be done on $G$,
\item Computing the probabilities of the new graph $G'$ achieved by reversing $e$.
\item Computing a query $q$ on $G'$. 
\end{enumerate}
which ultimately allows us to determine whether the cost of reversal was worth the speedups of answering $q$. \newline

\subsection{Restrictions on Graph Structure}\label{restrictions}

\null \quad \quad In order to reasonably compute the complexity of these algorithms, we must establish some restrictions on the structure of our DAG $G$ for the following reasons:
\begin{itemize}
\item Even in sparse graphs -- graphs in which $|E| \in \mathcal{O}(|V|)$ -- the number of parents is not inherently bounded. For example, the graph could consist of a single child vertex of all other vertices in the graph. Therefore, even restricting ourselves to sparse graphs is not sufficient if we wish to keep our algorithms tractable. To combat this, we set a limitation of the maximum number of parents allowable for a given vertex. 
\item Likewise, the number of cycles (Definition~\ref{def:directedcycle}) in a graph are not inherently restricted. This means that the number of paths between two arbitrary vertices in $G$ are affected exponentially. We therefore also restrict the maximum number of cycles allowable in the graph. 
\end{itemize}

\null \quad \quad Luckily, in practice, these restrictions are reasonable for a graphical model; graphical models are meant to be sparse graphs by the nature of their usage, and are often explicity constructed this way~\cite{sparse1, sparse2}. \newline
\null \quad \quad The particular restrictions we enforce are: (1) The maximum degree of a given vertex $v \in V$ is logarithmic, that is, maxdeg($G$) $\in \mathcal{O}(log(|V|))$, and (2) the maximum number of cycles in $G$ is logarithmic; num\_cycles $\in \mathcal{O}(log(|V|))$. \newline
\begin{remark} One result of these restrictions is that a single conditional probability table of $G$ contains at most $\mathcal{O}(|V|)$-many values and $\mathcal{O}(|V|)$-many paths between two given vertices. 
\end{remark}

\subsection{Complexity of Components}

\null \quad \quad Let $G = (V,E)$ be a sparse DAG with the aforementioned restrictions that (1) maxdeg($G$) $\in \mathcal{O}(log(|V|))$ and num\_cycles $\in \mathcal{O}(log(|V|))$. Let the vertices in $G$ be binary. \newline

\textsc{Covered\_edges:}
\begin{myindentpar}{1.5em} Algorithm~\ref{algo:coverededges} has complexity $\mathcal{O}(|V| \cdot log(|V|))$. This is because the main component of the algorithm is an iteration through the set of edges $E$ ($|V|$-many steps), wherein each iteration, two sets of vertices (bounded by the total number of vertices $|V|$) are compared. Since the sets of vertices being compared are sets of parents, they are bounded by $log(|V|)$ by assumption. Therefore, the total complexity is $\mathcal{O}(|V| \cdot log(|V|))$. \newline
\end{myindentpar}

\textsc{Find\_Case:} 
\begin{myindentpar}{1.5em}Algorithm~\ref{algo:findcase} has complexity $\mathcal{O}(2^{c}|V|^{2})$ where $c$ is the number of cycles in $G$. This is because each cycle multiplies the number of possible paths between two vertices by $2$ in the worst case.  By the requirements that $c \in \mathcal{O}(log(|V|))$, this complexity becomes $\mathcal{O}(|V|^{3})$. \newline 
\end{myindentpar}

\textsc{Create\_Delta\_Set:} 
\begin{myindentpar}{1.5em} Algorithm~\ref{algo:createdeltaset} relies on a breadth-first search to identify a shortest path from a vertex $X$ to $Y$ in $G' = skeleton(G)$. Note that it only has to be called once, since a single breadth-first search can return all of the shortest paths requires from a vertex $X$ to each $Y_{i}$. A breadth first-search has complexity $\mathcal{O}(|V|+|E|)$. In the case if a sparse graph, meaning $|E| \in \mathcal{O}(|V|)$, this is $\mathcal{O}(|V|^{2})$. However, \textsc{Create\_Delta\_Set} should be considered in conjunction with \textsc{Find\_Case}, which it calls for each condition $Y_{j}$ in $q$. We use the fact that $Y \in \mathcal{O}(|V|)$. The dominating factor of calling \textsc{Create \_Delta\_Set} is calling \textsc{Find\_Case}. Therefore, \textsc{Create\_Delta\_Set} and \textsc{Find\_Case} together amount to a complexity of $\mathcal{O}(|V| \cdot |V|^{3}) = \mathcal{O}(|V|^{4})$ in the case of our restricted graphs. \newline
\end{myindentpar}

\textsc{Is\_Advantageous:} 
\begin{myindentpar}{1.5em}Algorithm~\ref{algo:isadv} called \textsc{Create\_Delta\_Set} twice for every covered edge $e\in E$. Naturally, the number of covered edges is bounded by $|E|$ which, in a sparse graph, is $\mathcal{O}(|V|)$. Therefore, considering the complexity of \textsc{Is\_Advantageous} called $\mathcal{O}(|V|)$-many times, we have $\mathcal{O}(|V|^{5})$. \newline
\end{myindentpar}

\textsc{Edge\_Switch:} 
\begin{myindentpar}{1.5em}Reversing an edge $(X,Y)$ to $(Y,X)$ requires us to compute the new conditional probabilities of $X$ and $Y$ as outlined in Lemma~\ref{sharedparent}. This reversal has complexity $\mathcal{O}(2^{|\prod_{G}^{X}|})$ for a target vertex $X$. That is, it scales exponentially with respect to the number of parents of the target node. Under our requirement that $\prod_{G}^{X} \in \mathcal{O}(log(|V|))$, the algorithm has complexity $\mathcal{O}(2^{log(|V|)}) = \mathcal{O}(|V|)$.  \newline
\end{myindentpar}

\subsection{Complexity of Querying under Restrictions}

\null \quad \quad  In Section~\ref{section:complexityofaquery}, we established the number of operations required to compute a query $q$ over a graph $G$. In this section, we find the complexity of each component of the computation under the restrictions placed in Section~\cref{restrictions}. The goal is to determine how the querying costs are affected by these restrictions, and specifically to demonstrate that under these reasonable assumptions on graph structure, the complexities of the computations presented in Section~\ref{section:complexityofaquery} become tractable. \newline
\null \quad \quad As before, we let $G = (V,E)$ be a DAG and $q$ a query. Let $\mathcal{X}, \mathcal{Y}, \mathcal{Z}, \subseteq V$, where targets $X_{i} \in \mathcal{X}$ and conditions $Y_{j}$ are in $\mathcal{Y}$. Let $\mathcal{Z}$ be the set of vertices which $q$ depends on, which are neither targets nor conditions: $\Delta(G,q) \textbackslash (\mathcal{X} \cup \mathcal{Y})$. Then, the form of $q$ is $q = p(\mathcal{X}|\mathcal{Y}) = p(X_{0},...X_{m}|Y_{0},...,Y_{n})$. \newline

Computing the joint distribution $p(\mathcal{X},\mathcal{Y},\mathcal{Z})$:
\begin{myindentpar}{1.5em}
We established earlier that the number of operations required to compute $p(\mathcal{X},\mathcal{Y},\mathcal{Z})$ is $(|\mathcal{X}|+|\mathcal{Y}|+ |\mathcal{Z}|-1)\cdot 2^{|\mathcal{X}|+|\mathcal{Y}|+|\mathcal{Z}|}$. By the assumption that $G$ is sparse, this becomes $\mathcal{O}(|V|) \cdot \mathcal{O}(2^{|V|}) = \mathcal{O}(|V|\cdot 2^{|V|})$.\newline
\end{myindentpar}

Marginalize $\mathcal{Z}$ out of the distribution:
\begin{myindentpar}{1.5em}
The arithmetic cost of marginalizing $\mathcal{Z}$ out of the distribution $p(\mathcal{X},\mathcal{Y},\mathcal{Z})$ is in $\mathcal{O}(2^{|\mathcal{X}|+|\mathcal{Y}|+|\mathcal{Z}|})$. Then, this is $\mathcal{O}(2^{|V|})$. \newline
\end{myindentpar}

Compute the marginal distributions of the conditionals $\mathcal{Y}$:
\begin{myindentpar}{1.5em}
The number of operations required to compute the joint marginal distribution $\mathcal{Y}$ is in $\mathcal{O}(2^{|\mathcal{X}|+|\mathcal{Y}|})$. Since $|\mathcal{X}| + |\mathcal{Y}| \leq |V|$, this is in $\mathcal{O}(2^{|V|})$. \newline
\end{myindentpar}

Condition the joint distribution on $\mathcal{Y}$:
\begin{myindentpar}{1.5em}
The cost of conditioning the joint distribution $p(\mathcal{X},\mathcal{Y})$ on the conditions $\mathcal{Y}$ to find $p(\mathcal{X}|\mathcal{Y})$ is $2^{(|\mathcal{X}| + |\mathcal{Y}|)}$. Again since $|\mathcal{X}| + |\mathcal{Y}| \leq |V|$, this becomes $\mathcal{O}(2^{|V|})$. \newline
\end{myindentpar}

Adding the complexities of each of the above subtasks yields the final complexity of $\mathcal{O}(2^{|V|})$, presented in \cref{thm:querycomplexitytotal}.

\begin{theorem}\label{thm:querycomplexitytotal}
 Let $G = (V,E)$ be a sparse binary DAG with the restrictions that $(1)$ the number of cycles in $G $ is in $\mathcal{O}(log(|V|)$ and $(2)$ the maximum degree of a vertex is in $\mathcal{O}(log(|V|))$. Let $q$ be a query. Then, the total cost of answering $q$ is $\mathcal{O}(2^{|V|})$.
\end{theorem}

\begin{remark} Since all instances of $2^{|V|}$ resulted from the binary setting, one can more generally conclude that querying in a categorical setting (in which each variable can take on $n$-many values), the complexity is $\mathcal{O}(n^{|V|})$
\end{remark}

\newpage
\subsection{Benefit of Edge Reversal on Querying}

\begin{table}[h!]
  \begin{center}
    \begin{tabular}{l  l  l }
	&Finding the joint distribution $p(\mathcal{X},\mathcal{Y},\mathcal{Z})$ 	& $(|\mathcal{X}|+|\mathcal{Y}|+ |\mathcal{Z}|-1)\cdot 2^{|E|}$ \\[1em]
	&Marginalizing out $\mathcal{Z}$	 						& $\mathcal{O}(2^{(|\mathcal{X}| + |\mathcal{Y}| + |\mathcal{Z}|)})$ \\[1em]
	&Finding the joint marginal $p(\mathcal{Y})$ 					& $\mathcal{O}(2^{(|\mathcal{X}| + |\mathcal{Y}|)}$) \\[1em]
	&Conditioning on all conditions $\mathcal{Y}$					& $\mathcal{O}(2^{(|\mathcal{X}| + |\mathcal{Y}|)}$) \\[1em]
    \end{tabular}
  \end{center}
  \caption{}
   \label{tab:tableofcosts}
  \end{table}
  
\null \quad \quad In this section, we determine the arithmetic speedup gained by reversing a single covered edge in a DAG $G$ with the restrictions outlined in~\cref{section:complexityofaquery}. In Chapter $3$ we determined that arithmetic cost of querying is the sum of the four components shown in Table~\ref{tab:tableofcosts}.\newline

\begin{lemma}\label{lemma:onevertex}
Let $G=(V,E)$ be a DAG. Reversing a covered edge $(X,Y) \in E$ to be $(Y,X)$ changes $|\Delta(G,q)|$ by at most one vertex. 
\end{lemma}

\begin{proof}
Recognizing that reversible edges are covered edges, reversing an edge $(X,Y)$ to $(Y,X)$ can only reduce $\Delta(G,q)$ by one vertex. This is because $X$ and $Y$ share ancestors (with the exception of $X$ itself being an ancestor of $Y$), so reversing $(X,Y)$ does not disconnect $\alpha(X)$ from $\Delta(G,q)$. 
\end{proof}

\null \quad \quad In response to~\cref{lemma:onevertex}, we determine the number of operations saved by removing a single vertex from $\Delta(G,q)$. Let $Z \in \mathcal{Z}$ be the vertex which is removed from $\Delta(G,q)$ by reversing an edge $(Z,\cdot)$ to $(\cdot, Z)$. Here, we define $\mathcal{Z}$ to still include $Z$ after reversal. Then the cost of finding the joint distribution $p(\mathcal{X}, \mathcal{Y}, \mathcal{Z})$, which was previously $(|\mathcal{X}|+|\mathcal{Y}|+|\mathcal{Z}|-1)\cdot 2^{|\mathcal{X}|+|\mathcal{Y}|+|\mathcal{Z}|}$, becomes $(|\mathcal{X}|+|\mathcal{Y}|+|\mathcal{Z}|-2)\cdot 2^{|\mathcal{Z}|+|\mathcal{Z}|+|\mathcal{Z}|-1}$. \newline
\null \quad \quad Next, we notice that only two of the above terms depended on $\mathcal{Z}$: finding the joint distribution $p(\mathcal{X},\mathcal{Y},\mathcal{Z})$, and marginalizing out $\mathcal{Z}$. Since we have already considered the effect on finding the joint distribution, we now only need to consider the speedup on the marginalization. The cost of marginalizing out all elements in $\mathcal{Z}$ is reduced from $2^{(|\mathcal{X}|+|\mathcal{Y}|+|\mathcal{Z}|)}$ to $2^{(|\mathcal{X}|+|\mathcal{Y}|+|\mathcal{Z}|-1)}$. \newline
\null \quad \quad The total number of operations saved is therefore $2^{|\mathcal{X}|+|\mathcal{Y}|+|\mathcal{Z}|} + 2^{|\mathcal{X}|+|\mathcal{Y}|+|\mathcal{Z}|-1}$. This is an exponential improvement; each time the size of $\mathcal{Z} \subset \Delta(G,q)$ is reduced by $1$, the cost of computing $q$ halves. That is, if we reduce the size of $\mathcal{Z}$ by $k$-many variables, the complexity of answering $q$ is improved by $\mathcal{O}(2^{k}).$

%%%%%%%%%%%%%%% CHAPTER 6 COMPARISON
% Comparison
\newpage
\chapter{Efficacy of Transformation}

\section{Comparison of methods}
\null \quad \quad In Chapter $5$ we determined the complexity of computing a query over a DAG $G$, the complexity of finding an edge reversal in $G$ such that $\Delta(G,q)$ is reduced via the \textit{greedy strategy}, and of recalculating the probabilities of graph after edge reversal. The result is a fully specified DAG $G'$ which is Markov equivalent to $G$ such that answering $q$ on $G'$ is faster than answering $q$ on $G$. However, we must now compare the speedups earned in answering $q$ on $G'$ instead of $G$ with the cost of transforming $G$ into $G'$. We will also consider these results in the context of sequences of queries, rather than individual queries. \newline
\null \quad \quad In this chapter, we continue enforce restrictions on the DAG $G$ outlined in~\cref{section:complexityofaquery}: that the maximum number of cycles in $G$ is in $\mathcal{O}(log(|V|))$, that the maximum degree of a vertex is in $\mathcal{O}(log(G))$, that $G$ is sparse, and that $G$ has binary variables. \newline
\begin{table}[h!]
  \begin{center}
    \begin{tabular}{ l  l }
	\quad The cost of answering a query $q$ on $G=(V,E)$ 		& $\mathcal{O}(2^{|V|})$ \\[1em]
	\quad The speedups earned by reversing $n$ edges in $G$		& $\mathcal{O}(2^{n})$ \\[1em]
	\hline \\
	\quad  The cost of finding all covered edges 				& $\mathcal{O}(|V| \cdot log(|V|))$ \\[1em]
	$+$ The cost of finding $\Delta(G',q)$ for all covered edges 	& $\mathcal{O}(|V|^{5})$\\[1em]
	$=$ The cost of finding an advantageous edge reversal 		& $\mathcal{O}(|V|^{5})$ \\[1em]
	\hline \\
	\quad The cost of calculating an edge reversal					& $\mathcal{O}(|V|)$ 
    \end{tabular}
      \caption{}
      \label{tab:complexitytable}
  \end{center}
  \end{table}
\null \quad \quad The calculations in Chapter $5$ revealed the complexities shown in Table \ref{tab:complexitytable}.
While calculating the edge reversal itself is less costly than answering a query, we immediately see that the most expensive component of the \textit{greedy strategy} is determining the optimal reversal to make. \newline
\null \quad \quad We also notice that finding the entire set of covered edges in $G$ only has to be done once, since reversing an edge only changes the set of covered edges locally. That is, only edges adjacent to the reversed edge can switch whether they are covered after the reversal. By the assumption that vertices in the graphs have a small number of parents (specifically $\mathcal{O}(log(|V|)))$, this is fast to check. After the set of covered edges is found, multiple reversible edges can be evaluated and reversed without calling \textsc{Find\_Covered\_Edges} again. Therefore, the overhead cost of $\mathcal{O}(|V|)$ only applies once for finding transformations of a given graph $G$. \newline

\begin{theorem}\label{thm:reversalbenefit}
Identifying and reversing beneficial edges is always worth the computational improvement earned from reversal for graphs with $|V|$ \textgreater \ $22$. That is, the complexity of identifying and reversing a beneficial edge is at most as much computational cost as the query saves.
\end{theorem}

\begin{proof}
The complexity of finding covered edges and reversing a beneficial one is $\mathcal{O}(|V|^{5})$. The computation of a query is halved whenever $|\Delta(G,q)|$ is reduced by $1$. Therefore, by finding approximate integer solutions to $2^{|V|} = |V|^{5}$, we see that $2^{|V|}$ is larger whenever $|V| \not\in [2,22]$.
\end{proof}

\begin{remark}
Note that Theorem~\ref{thm:reversalbenefit} does not imply that reversal is \textit{not} worth doing whenever $|V|$ \textless \  $22$, but rather, that it isn't necessarily, but still may be.
\end{remark}

\begin{remark}
While $2^{|V|}$ is a fixed quantity because it corresponds to a fixed number of possible events described by the distribution, $|V|^{5}$ is a worst-case complexity which could likely be reduced with more effective algorithm design. The bound of $V$ \textless \ $22$ is certainly not strict in practice, and it is likely possible that edge reversal could become advantageous for smaller graphs as well. One can reasonably adopt the heuristic that edge reversal is generally advantageous. A promising area for future research is to develop more effective algorithms to reduce the cost of finding beneficial edges to reverse. 
\end{remark}

\section{Reversal in the Context of Multiple Queries} 
\null \quad \quad In this section, we explore the benefits and limitations of edge reversal if we consider a sequence of queries $Q = \{q_{1}, q_{2}, ..., q_{n}\}$ instead of a single query $q$. For example, if we have observable medical data about patients and we wish to determine whether they have multiple diseases (but not the likelihood of them having several diseases simultaneously), we may wish to answer multiple queries consecutively. \newline
\null \quad \quad Answering a sequence of queries $\{p(X_{1}), p(X_{2}), ..., p(X_{n})\}$ is different than answering a single query with multiple targets, $p(X_{1}, X_{2}, ... X_{n})$ in several ways: the former quantifies each probability as individual events, whereas the latter considers the likelihood that both diseases are present simultaneously. Additionally, a sequence of queries allows us to look at different conditions for each query. For example, if $p(X_{1})$ is relevant to the condition $Y_{1}$ in a DAG $G$, but $X_{2}$ is not, then we can simply answer the queries separately ($p(X_{1}|Y_{1})$ and $p(X_{2}$)), rather than considering every condition in both queries. \newline
\null \quad \quad While answering a sequence of queries $Q=\{q_{1}, q_{2}, ..., q_{n}\}$ allows us to determine the probabilities of multiple variables individually, it is also an expensive process since each query must be computed independently. However, knowing the sequence of queries in advance gives us the advantage of being able to consider which graph transformations may be beneficial to the entire sequence, and further, how we might permute the sequence before transforming to reduce the costs of querying. 

\subsection{Universal Effects}

\begin{theorem}\label{thm:universalrev}
Given a DAG $G$ and a reversible edge $(X,Y) \in E$, there is no universally beneficial edge reversal. That is, we can always find a query $q$ such that computing $q$ is more costly after reversing $(X,Y)$ to $(Y,X)$. 
\end{theorem}

\begin{proof}
To demonstrate this, we show that we can always find a query $q$ such that $|\Delta(G,q)|$ is increased when $(X,Y)$ is reversed to $(Y,X)$. Let $G$ be a DAG, and let $(X,Y)$ be an arbitrary edge in $G$. Then $X \in \alpha(Y)$. Let $q = p(X)$. A query of this form depends on exactly $X \cup \alpha(X)$, since it has no conditions and is in the general case by default. \newline
\null \quad \quad Suppose $G'$ is the graph achieved by reversing $(X,Y)$ to $(Y,X)$ in $G$. Then, $Y$ is added to $\alpha(X)$ in $G'$. $X$ still depends on all other vertices in $\alpha(X)$ since we have only added a parent, not removed any. Therefore, $\Delta(G,q)$ now includes the vertex $X$, meaning that $|\Delta(G,q)|$ has grown by $1$. Recall that by Lemma~\ref{lemma:onevertex}, since $(X,Y)$ is a covered edge, $X$ and $Y$ share a set of parents, so the query's cost is only increased by $1$. Thus, since the result that $|\Delta(G',q)|$ \textgreater \ $|\Delta(G,q))|$ is independent of all other properties of $G$, and the chosen edge is arbitrary, we conclude that we can always find such a query $q$. 
\end{proof}

\begin{remark}
A rephrasing of~\ref{thm:universalrev} is that for a set of queries $Q = \{q_{1}, q_{2}, ..., q_{n}\}$, there is not necessarily a single edge reversal which benefits the entire set simultaneously. 
\end{remark}

\begin{definition}
Let $G$ be a DAG, and let $q_{1}$ and $q_{2}$ be queries. If an edge reversal of $(X,Y)$ in $G$ is detrimental to $q_{1}$ but beneficial to $q_{2}$, we call $q_{1}$ and $q_{2}$ \textbf{conflicting queries}. If the reversal is beneficial to both or detrimental to both, we call them \textbf{coinciding queries}. If $|\Delta(G,q_{1})|$ is unaffected by an edge reversal, we call $q_{1}$ a neutral query with respect to the edge. 
\end{definition}

\begin{center}
\begin{figure}[h!]
\centering
\begin{tikzpicture}
\Vertex[x=-4.5, size=.5, color=white, label=$X$]{x1}
\Vertex[x = -2.5, size=.5, color=white, label=$Y$]{x2}
\Vertex[y = 1.5, x=-3.5 ,size=.5,color=white, label=$Z$]{x3}
\Edge[Direct, distance=0.5](x3)(x1)
\Edge[Direct, distance=0.5](x3)(x2)
\Edge[Direct, distance=0.5](x1)(x2)

\Vertex[size=.5, color=white, label=$X$]{x1}
\Vertex[x = 2, size=.5, color=white, label=$Y$]{x2}
\Vertex[y = 1.5, x=1 ,size=.5,color=white, label=$Z$]{x3}
\Edge[Direct, distance=0.5](x3)(x1)
\Edge[Direct, distance=0.5](x3)(x2)
\Edge[Direct, distance=0.5](x2)(x1)

\Text[x=-3.5 ,y=-.5]{$G$}
\Text[x=1, y=-.5]{$G'$}
\end{tikzpicture}
\caption{}
\label{fig:querydetriment}
\end{figure}
\end{center}

\begin{example} Let $G$ be the DAG in Figure~\ref{fig:querydetriment} and let $G'$ be the DAG achieved by reversing the covered edge $(X,Y)$ in $G$. Let $q = p(X)$. In $G$, we see that $\Delta(G,q) = \{X,Z\}$ since the query depends on the target $X$ and its single ancestor $Z$. \newline
\null \quad \quad In contrast, $\Delta(G',q) = \{X,Y,Z\}$ since the query depends on the target $X$ and its ancestors, $\alpha(X) = \{Z,Y\}$. Therefore, reversing the covered edge $(X,Y)$ to be $(Y,X)$ has increased the number of vertices required for the computation by $1$. As a result, the complexity of computing the query has doubled.
\end{example}

\subsection{Permutations of Sequences of Queries}
\null \quad \quad Recognizing that there are conflicting, coinciding, and neutral queries with respect to a DAG $G$ provides a framework for finding a permutation of a sequence of queries $Q = \{q_{1}, q_{2}, ..., q_{n}\}$ such that the number of edge reversals is minimized when each query is answered in order on its minimal graph. \newline
\null \quad \quad For example, if $q_{1}$ and $q_{2}$ are coinciding queries while $q_{3}$ conflicts with both, we immediately recognize that $Q'=\{q_{1},q_{3},q_{2}\}$ is a less efficient sequence than $Q$. This is because after computing $q_{3}$, its conflicting query $q_{2}$ is either being answered on a less efficient graph than $G$, or we must again incur the costs of transforming the graph back to its original form. By grouping coinciding queries, we can reduce the costs of answering the sequence of queries; this is done by reducing the number of transformations that must be made while simultaneously lowering the cost of individual queries. We then follow the heuristic that coinciding queries with respect to an edge reversal should appear nearby one another in the final permutation. \newline
\null \quad \quad To group queries by this heuristic, we develop a measurement of distance between two queries on a DAG. Creating this distance entails creating an indexing system for oriented vertices in $G$: \newline
\null \quad \quad We must first create an ordering for the edges in $G$; the order itself is arbitrary. Assign each edge $e \in E$ to an integer label in $[1,|E|]$. Likewise, label each vertex $v \in V$ with a value in $[1,|V|]$. 
\begin{definition}
Let $G$ be a DAG and let each edge be assigned arbitrarily to an integer label in $[1,|V|]$. If an edge's orientation is from a higher indexed vertex to a lower indexed vertex, we refer to this edge as having \textbf{positive} orientation with respect to the labeling. Otherwise, we say the edge has \textbf{negative} orientation. 
\end{definition}
\null \quad \quad From here onward, we will refer to a single, consistent labeling of edges and vertices. We will then encode the orientations of a DAG $G$ by a binary sequence of length $|E|$, where each position of the sequence encodes the orientation of the edge. Then, all members of a Markov equivalence class can be encoded as binary sequences. \newline

\begin{lemma} Given a DAG $G$, we determine the \textbf{optimal graph} $G_{q} \in [G]$ for a query $q$ by iterating over the set of covered edges $E_{C}$ in $G$; if an edge reversal is advantageous for $q$ (meaning the reversal reduces $|\Delta(G,q)|$), it is reversed. If it is detrimental or neutral, it is not reversed. The resulting graph, which minimizes $|\Delta(G,q)|$ over $[G]$, is $G_{q}$.
\end{lemma}

\begin{remark}Since finding an optimal graph $Q_{q}$ in $[G]$ does not assign a specific orientation to neutral edges with respect to $q$, optimal graphs are not unique to a query. However, the set $|\Delta(G,q)|$ has fixed size across all possible optimal graphs. 
\end{remark}

\begin{example} Let $G_{q_{1}}$ be the optimal graph in the Markov equivalence class $[G]$ to answer $q_{1}$. Let $G_{q_{2}}$ be the optimal graph in $[G]$ to answer $q_{2}$, as shown in Figure \ref{fig:binaryencoding}. \newline
\null \quad \quad The binary sequence which encodes the orientations of these graphs will have lengths $|E| = 3$, where the first digit refers to the orientation of $e_{1}$ and the second digit refers to the orientation of $e_{2}$, and the third to $e_{3}$. Then each edge $e_{i}$ is encoded at a fixed position $i$, which is consistent across all graphs in $[G]$ since they share a skeleton. Assigning $1$ for positively oriented edges and $0$ for negatively oriented covered edges, we arrive at the following binary encoding of the graphs:
\begin{align*}
G_{q_{1}} : & \ [1,0,1]  \\
G_{q_{2}} : & \ [1,1,1]
\end{align*}

\begin{center}
\begin{figure}[h!]
\centering
\begin{tikzpicture}
\Vertex[x=-4.5, size=.5, color=white, label=$X_{1}$]{x1}
\Vertex[x = -2.5, size=.5, color=white, label=$X_{2}$]{x2}
\Vertex[y = 1.5, x=-3.5 ,size=.5,color=white, label=$X_{3}$]{x3}
\Edge[Direct, distance=0.5, label = $e_{1}$](x3)(x1)
\Edge[Direct, distance=0.5, label=$e_{3}$](x3)(x2)
\Edge[Direct, distance=0.5, label=$e_{2}$](x1)(x2)

\Vertex[size=.5, color=white, label=$X_{1}$]{x1}
\Vertex[x = 2, size=.5, color=white, label=$X_{2}$]{x2}
\Vertex[y = 1.5, x=1 ,size=.5,color=white, label=$X_{3}$]{x3}
\Edge[Direct, distance=0.5, label = $e_{1}$](x3)(x1)
\Edge[Direct, distance=0.5, label=$e_{3}$](x3)(x2)
\Edge[Direct, distance=0.5, label = $e_{2}$](x2)(x1)

\Text[x=-3.5 ,y=-.5]{$G_{q_{1}}$}
\Text[x=1, y=-.5]{$G_{q_{2}}$}
\end{tikzpicture}
\caption{}
\label{fig:binaryencoding}
\end{figure}
\end{center}
\end{example}
\null \quad \quad Then, iterated over the entire set of reversible edges in $G$, we define the \textit{query distance} between $q_{1}$ and $q_{2}$ to be the Hamming distance between the two integer sequences encoding the optimal graphs for $q_{1}$ and $q_{2}$. That is, for the two sequences $S_{1}$ and $S_{2}$, the distance is defined as the number of positions $i$ where $S_{1}[i] \neq S_{2}[i]$. \newline
\null \quad \quad For example, if we have the sequences $[1,1,0,1,0]$ and $[1,1,1,0,0]$, the distance (number of mismatches) is $2$, since the third and fourth entries do not match. Similarly, for the graphs in Figure \ref{fig:binaryencoding}, the distance is $1$. 

\begin{definition} Given a DAG $G$ and two queries $q_{1}$ and $q_{2}$, the \textbf{query distance} between $q_{1}$ and $q_{2}$ with respect to $G$, written $d_{q}^{G}(q_{1},q_{2})$ is defined as the Hamming distance between the binary encodings of the optimal graphs for $q_{1}$ and $q_{2}$ in $[G]$.
\end{definition}

\begin{lemma}\label{lemma:hammingdistance}
Let $S_{1}$ be the binary sequence encoding the optimal graph for a query $q_{1}$ in $[G]$. Likewise, let $S_{2}$ be the binary sequence encoding the optimal graph for $q_{2}$. The set of conflicting edges between $S_{1}$ and $S_{2}$ can be calculated via a bitwise \textsc{XOR} operator on $S_{1}$ and $S_{2}$.
\end{lemma}

\begin{proof} \textsc{XOR}$(S_{1}, S_{2})$ returns  binary sequence $S^{*}$ with the property that $S^{*}[i] = 1$ exactly when exactly one of the values $S_{1}[i]$ and $S_{2}[i]$ is $1$. 
\end{proof}

\begin{remark}
Query distance is symmetric. This is easy to verify, since the number of mismatches between two sequences is independent of the order in which the sequences are presented. 
\end{remark}

\begin{example}
Let $S_{1}$ be the encoding for the optimal graph for $q_{1}$ in $[G]$. Let $S_{2}$ be the encoding for $q_{2}$.
\begin{align*}
& S_{1} & \quad  [1,1,1,0,0,1]\\ 
& S_{2} & \quad  [1,1,0,0,0,0]\\ 
 \cline{1-3} \\[-2em]
& \textsc{XOR}(S_{1}, S_{2}) & \quad [0,0,1,0,0,1] 
\end{align*}
The result has $2$ ones (mismatches), meaning $d_{q}^{G}(q_{1},q_{2}) = 2$. 
\end{example}

\begin{lemma}
Let $G_{q_{1}}$ and $G_{q_{2}}$ be the optimal graphs for $q_{1}$ and $q_{2}$ in $[G]$ respectively. Then, the query distance between $q_{1}$ and $q_{2}$ is exactly the number of edge reversals requires to transform $Q_{q_{1}}$ to $G_{q_{2}}$.
\end{lemma}
\proof{This result follows from the definition of query distance, since each mismatched entry in the binary sequences encodes a disagreement in edge orientations between $G_{q_{1}}$ and $G_{q_{2}}$. Each disagreeing edge requires one edge reversal in order for the orientations to match. }

\begin{lemma} The complexity for determining the query distance between two queries over a DAG $G$ given their optimal graph encodings is $\mathcal{O}(|V|)$.
\end{lemma}

\begin{proof}
The complexity of a bitwise operation is $\mathcal{O}(|E|) = \mathcal{O}(|V|)$. This is because one must iterate through the result of \textsc{XOR}($S_{1},S_{2})$ to determine the total number of mismatched edge orientations (total number of $1$s) in the resulting sequence.
\end{proof}

\null \quad \quad Computing query distance over a Markov equivalence class $[G]$ relies on knowing an optimal graph $G_{q_{i}}$ for each query $q_{i}$ on $[G]$. Therefore, we outline the required steps to compute query distance from scratch, in order contextualize the process. Let $Q = \{q_{1}, q_{2},... q_{n}\}$.
\begin{enumerate}
\item For each query $q_{i} \in Q$, compute $\Delta(G,q_{i})$.
\item Initialize the data structures for reversible edges. That is, for each pair of vertices connected by an edge, we compare the set of parents of those vertices. This is to determine whether the edge is reversible, and if so, and mark it as such. 
\item For each query $q_{i}$, using the data structures initialized in step $(2)$, repeatedly find advantageous edges to reverse and reverse them in the data structures. They should \textit{only} be reversed in the graph, and no actual probabilities should be recomputed. Once all advantageous edges have been reversed for a given query $q_{i}$, the resulting graph serves as the optimal graph $G_{q_{i}}$.
\item Compute query distance via the method described in Lemma \ref{lemma:hammingdistance} using the graphs determined in step (3). 
\end{enumerate}

\subsection{Minimizing Permutations of Sequences of Queries}

\null \quad \quad Then, to find the permutation which minimizes the number of graph transformations necessary as well as the cost of computing each query, we aim to minimize the query distance between all pairs of succeeding queries in $Q$. That is, given a sequence of queries $Q = \{q_{1}, q_{2}, ..., q_{n}\}$, we wish to find a permutation of $Q$ such that the query distance between all consecutive pairs $(q_{i}, q_{i+1})$ is minimized. For $Q_{p_{k}},\ (k \in [0, n!])$ a permutation of $Q$, the minimization is:
$$\min_{Q_{p_{k}}}(\sum_{i=1}^{n-1}d_{q}^{G}(q_{i},q_{i+1})).$$

\null \quad \quad For $Q = \{q_{1}, q_{2}, ..., q_{3}\}$ a sequence of queries, we can work toward this minimization by constructing graphical data structure separate from the structure of the underlying Bayesian net $G$:

\begin{definition}\label{querydistancegraph}
Let $Q= \{q_{1}, q_{2}, ..., q_{n}\}$. Let $G=(V,E)$ be the DAG on which all queries in $Q$ are being asked. Then, the \textbf{query distance graph} $G_{Q}$ is a complete graph with one vertex for each query in $Q$, and edge weights between vertices equal to the pairwise query-distance between the corresponding queries. 
\end{definition}

\begin{remark}
The query distance graph $G_{Q}$ is constructed as follows: for each query $q_{i} \in Q$, create a vertex $v_{i} \in V$, meaning $|Q| = |V|$. Connect every pair of vertices in $G_{q}$ by an undirected edge; this means $G_{Q}$ is a complete graph. Weight each edge $(v_{i}, v_{j})$ with the query distance $d_{q}^{G}(q_{i}, q_{j})$. 
\end{remark}

\begin{remark}
If the original DAG $G$ is not an optimal graph of some query $q_{i} \in Q$ in Definition \ref{querydistancegraph}, then another vertex $v_{i+1}$ should be added to represent $G$. This is because we must also consider the distance from the specified starting graph, since that tells us the cost of transforming from the specified graph to another optimal Markov equivalent graph.  Therefore, in a query distance graph, $|V|$ is either $|Q|$ or $|Q|+1$. Adding this single vertex does not add to the asymptotic complexity of operations on the graph. 
\end{remark}

\begin{example} Let $Q = \{q_{1}, q_{2}, q_{3}\}$ be a sequence of queries on $G$. Then, if $d_{q}^{G}(q_{1}, q_{2}) \\ = 3$, $d_{q}^{G}(q_{1}, q_{3}) = 2$, and $d_{q}^{G}(q_{2}, q_{3}) = 1$, the query distance graph $G_{Q}$ will have the form shown in Figure \ref{fig:querydistancegraphex}.
\begin{figure}[h!]
\centering
\begin{tikzpicture}
\Vertex[x=-4.5, size=.5, color=white, label=$v_{1}$]{x1}
\Vertex[x = -2.5, size=.5, color=white, label=$v_{2}$]{x2}
\Vertex[y = 1.5, x=-3.5 ,size=.5,color=white, label=$v_{3}$]{x3}
\Edge[distance=0.5, label = $2$](x3)(x1)
\Edge[distance=0.5, label= $1$](x3)(x2)
\Edge[distance=0.5, label=$3$](x1)(x2)

\Text[x=-3.5 ,y=-.5]{$G_{Q}$}
\end{tikzpicture}
\caption{}
\label{fig:querydistancegraphex}
\end{figure}
\end{example}

\null \quad \quad Once we have constructed $G_{Q}$ for a sequence of queries $Q$, finding the optimal permutation of $Q$ such that the queries can be answered most efficiently via edge reversal can be rephrased as the following task: find a path which visits every vertex in $G_{Q}$ such that sum of the weights of the traversed edges is minimized.  \newline
\null \quad \quad If we do not wish to store information about the graphs determined at intermediate steps between two optimal graphs, then we must search for a path which visits each vertex once; this problem is equivalent to the Hamiltonian Path problem, which searches for a path which visits every vertex on a complete graph exactly once. In the case of a complete graph, this problem is trivial. However, we do not know of a polynomial-time algorithm to compute the \textit{shortest} Hamiltonian path in a complete graph~\cite{hamiltonianpath}. \newline
\null \quad \quad In our scenario, however, it may be advantageous to store intermediate representations of the graphs while we are doing inference on all the queries in $Q$. Here, we don't have to find a single path which traverses each vertex once, but rather, can find a path which cycles back upon itself when convenient.\newline
\begin{example}\label{ex:quicktree}
Suppose we have three queries, $q_{1}$, $q_{2}$, and $q_{3}$ in a sequence $Q$. Let $G_{q_{i}}$ be an optimal graph for each $q_{i} \in Q$. Then, we may come across circumstances in which it is advantageous to transform from $G_{q_{1}}$ to $G_{q_{2}}$, and then return to our stored graph $G_{q_{1}}$ before transforming to $G_{q_{3}}$ to answer $q_{3}$. 
\end{example}
\null \quad \quad In this event, it is in our interest to store data about $G_{q_{1}}$ rather than re-determining its optimal graph if we wish to return to it to construct other optimal graphs. Using this method, we now have the option to jump back to a previous graph representation if desired. As such, our traversal of the query distance graph may not be a single path, but rather, a tree. This reflects the opportunity to jump back to representation along previously determined branches, which has only the cost of storing graph data. 

\begin{example} Let $Q = \{q_{1}, q_{2}, q_{3}, q_{4}, q_{5}, q_{6}\}$ be a sequence of queries on a DAG $G$. Then, using the method of jumping outlined in Example \ref{ex:quicktree}, our traversal of the query distance graph $G_{Q}$ may follow a tree structure. Suppose we have the following traversal, which does not store intermediate graph representations and therefore must return to previous vertices: $$q_{1} \rightarrow q_{2} \rightarrow q_{3} \rightarrow q_{2} \rightarrow q_{1} \rightarrow q_{4} \rightarrow q_{5} \rightarrow q_{4} \rightarrow q_{6}.$$\newline
\null \quad \quad This traversal indeed covers all queries in $Q$, but is highly inefficient if one has to recompute all intermediate graphs $G_{q_{i}}$. Instead however, recognizing that jumping to previous graphs allows us to traverse $G_{Q}$ via a tree, we may see the traversal structure shown in Figure \ref{fig:querydistancegraphex}.
\begin{figure}[h!]
\centering
\begin{tikzpicture}
    \node (p1) at ( 0,4) {$q_1$}; 
    \node (p2) at ( 1,4) {$q_2$};
    \node (p3) at ( 2,4) {$q_3$};
    \node (p4) at ( 0,2) {$q_4$};
    \node (p5) at ( 1,2) {$q_5$};
    \node (p6) at ( 0,0) {$q_6$};

    \begin{scope}[every path/.style={->}]
       \draw (p1) -- (p2);
       \draw (p2) -- (p3); 
       \draw (p1) -- (p4);
       \draw (p4) -- (p5);
       \draw (p4) -- (p6);
    \end{scope}  

\end{tikzpicture}
\caption{Structuring our traversal of $G_{Q}$ as this tree allows us to jump back to $q_{1}$ and $q_{4}$ as appropriate by storing their data during intermediate transformation steps. }
\label{fig:querydistancegraphex}
\end{figure}
\end{example}

\begin{lemma}\label{lem:representspace}
For a sequence of queries $Q=\{q_{1}, q_{2}, ..., q_{n}\}$, storing a single graph representation requires at most $|E| + |V|^{2} = \mathcal{O}(n \cdot  |V|^{2})$ space.
\end{lemma}
\begin{proof}
 This is due to the fact that we must store the orientation of each edge and the conditional probability table for each vertex. Maintaining the assumption that each vertex has at most $log(|V|)$-many parents, with an upper bound of $|V|$, the worst case scenario may require us to store $n$ many representations, where $n$ is the number of queries in a sequence $Q$. In total, this becomes $\mathcal{O}(n \cdot |V|^{2})$ space. 
\end{proof}

\begin{remark}
The space requirement outlined in Lemma \ref{lem:representspace} is not a hard boundary, and can likely be improved. 
\end{remark}

\null \quad \quad Now that we have established the usefulness of tree-like traversals of query distance graphs, we must determine how to identify beneficial (and minimal) traversal trees. The requirements of the sought tree are the following: it must contain each vertex and all of the edges from the original graph $G$ (that is, it must be a spanning tree). The total sum of edge weights in the path must be minimized, meaning we must search for a \textit{minimum spanning tree}. \newline
\null \quad \quad Identifying minimum spanning trees of a DAG $G$ is a well established algorithmic problem. There are many algorithms which return minimum spanning trees, such as Kruskal's algorithm~\cite{spanning}. Kruskal's algorithm runs in $\mathcal{O}(|E| log(|V|))$ time. \newline
\null \quad \quad Therefore, once all pairwise distances are established between distances and $G_{Q}$ is constructed, one can run Kruskal's algorithm to find the minimum spanning tree. This tree is equivalent to the optimal permutation of queries in a sequence $Q$, assuming that one is storing intermittent graph data in order to return to previous states. Determining this optimal sequence of queries requires $\mathcal{O}(n \cdot |V|^{2})$ space, where $n$ is the length of $Q$, and $|V|$ is the number of vertices in $G_{Q}$. 


\begin{figure}[h!]
\begin{center}
\begin{tikzpicture}
\Vertex[x = 0, y=3, size=.5, color=white, label=$X_{1}$]{x}
\Vertex[x = 0, y=1.5, size=.5, color=white, label=$X_{2}$]{y}
\Vertex[x=0, y = 0, size=.5,color=white, label=$X_{3}$]{z}
\Edge[Direct, distance=0.5](x)(y)
\Edge[Direct, distance=0.5](y)(z)
\Text[x=-.8 ,y=3]{$G_{a}$}

\Vertex[x = 3, y=3, size=.5, color=white, label=$X_{1}$]{x}
\Vertex[x = 3, y=1.5, size=.5, color=white, label=$X_{2}$]{y}
\Vertex[x=3, y = 0, size=.5,color=white, label=$X_{3}$]{z}
\Edge[Direct, distance=0.5](y)(x)
\Edge[Direct, distance=0.5](y)(z)
\Text[x=2.2 ,y=3]{$G_{b}$}

\Vertex[x = 6, y=3, size=.5, color=white, label=$X_{1}$]{x}
\Vertex[x = 6, y=1.5, size=.5, color=white, label=$X_{2}$]{y}
\Vertex[x=6, y = 0, size=.5,color=white, label=$X_{3}$]{z}
\Edge[Direct, distance=0.5](z)(y)
\Edge[Direct, distance=0.5](y)(x)
\Text[x=5.2 ,y=3]{$G_{c}$}

\end{tikzpicture}
\end{center}
\caption{}
\label{permutationexample}
\end{figure}

\begin{example}\label{querydistanceex}
Let $Q = \{q_{1}. q_{2}, q_{3} \}$ be a sequence of queries asked on the DAG $G_{a}$ in Figure \ref{permutationexample}, where $q_{1} = p(X_{1}),\ q_{2} = p(X_{2})$, and $q_{3}=p(X_{3}).$ Then $G_{b}$ and $G_{c}$ are the two other members of the MEC $[G_{a}]$. \newline
\null \quad \quad By determining $|\Delta(G_{k},q_{i})|$ for each $G_{k} \ in [G_{a}]$ and each query $q_{i} \in Q$, we can determine the optimal graph for each query. For example, $|\Delta(G_{k},q_{1})| = 3, 1,$ and $2$ for $G_{a}$, $G_{b}$, and $G_{c}$ respectively. Therefore, $G_{b}$ is the optimal graph in $[G_{a}]$ for $q_{1}$. \newline
\null \quad \quad Likewise, $G_{c}$ is the optimal graph for $q_{2}$, while $G_{b}$ and $G_{c}$ are both optimal graphs for $q_{3}$. We assign $G_{c}$ to be the optimal graph for $q_{3}.$ $G_{b}$ and $G_{c}$ are relabeled as such in Figure \ref{optimalgraphex}.

\begin{figure}[h!]
\begin{center}
\begin{tikzpicture}

\Vertex[x = 3, y=3, size=.5, color=white, label=$X_{1}$]{x}
\Vertex[x = 3, y=1.5, size=.5, color=white, label=$X_{2}$]{y}
\Vertex[x=3, y = 0, size=.5,color=white, label=$X_{3}$]{z}
\Edge[Direct, distance=0.5](y)(x)
\Edge[Direct, distance=0.5](y)(z)
\Text[x=2.2 ,y=3]{$G_{b}$}

\Vertex[x = 6, y=3, size=.5, color=white, label=$X_{1}$]{x}
\Vertex[x = 6, y=1.5, size=.5, color=white, label=$X_{2}$]{y}
\Vertex[x=6, y = 0, size=.5,color=white, label=$X_{3}$]{z}
\Edge[Direct, distance=0.5](z)(y)
\Edge[Direct, distance=0.5](y)(x)
\Text[x=5.2 ,y=3]{$G_{c}$}

\draw [decorate,decoration={brace,amplitude=8pt,mirror,raise=-5ex}]
  (2,-1.3) -- (4,-1.3) node[midway,yshift=.5em]{$G_{q_{1}}$};
\draw [decorate,decoration={brace,amplitude=8pt,mirror,raise=-5ex}]
  (5,-1.3) -- (7,-1.3) node[midway,yshift=.5em]{$G_{q_{2}} = G_{q_{3}}$};

\end{tikzpicture}
\end{center}
\caption{}
\label{optimalgraphex}
\end{figure}

Since $G_{q_{2}} = G_{q_{3}}$, we immediately know that $d_{q}^{G_{a}}(q_{2},q_{3}) = 0.$ Then, since $G_{q_{1}}$ differs from $G_{q_{2}}$ by one edge, we determine that $d_{q}^{G_{a}}(q_{1}, q_{2}) = 1$. Trivially, $d_{q}^{G_{a}}(q_{1}, q_{3}) = 1$ as well. Therefore, the query distance graph $G_{Q}$ has the form shown in Figure \ref{querygraphex}.


\begin{figure}[h!]
\centering
\begin{tikzpicture}
\Vertex[x=-4.5, size=.5, color=white, label=$v_{1}$]{x1}
\Vertex[x = -2.5, size=.5, color=white, label=$v_{2}$]{x2}
\Vertex[y = 1.5, x=-3.5 ,size=.5,color=white, label=$v_{3}$]{x3}
\Edge[distance=0.5, label = $1$](x3)(x1)
\Edge[distance=0.5, label= $0$](x3)(x2)
\Edge[distance=0.5, label=$1$](x1)(x2)

\Text[x=-3.5 ,y=-.5]{$G_{Q}$}
\end{tikzpicture}
\caption{The query distance graph $G_{Q}$}
\label{querygraphex}
\end{figure}

\null \quad \quad Furthermore, $g_{c}$ only differs from $G_{a}$ by one edge, whereas $G_{b}$ differs from $G_{a}$ by two edges. Therefore, an optimal traversal (which is a minimal spanning tree over $G_{Q}$) should begin by computing $G_{c}$ from $G_{a}$, answering $q_{2}$ and $q_{3}$ in either order, then should compute $G_{b}$ to answer $q_{1}$. There is no need to save states in this traversal, since the tree is a single branch, namely
$$q_{2} \rightarrow q_{3} \rightarrow q_{1},$$
or equally effectively,
$$q_{3} \rightarrow q_{2} \rightarrow q_{1}.$$
\end{example}

\begin{remark}
In Example \ref{querydistanceex}, since both $G_{b}$ and $G_{c}$ can serve as the optimal graph $G_{q_{3}}$ for $q_{3}$, the query distance graph in Figure \ref{querygraphex2} will also provide an optimal ordering of queries.

\begin{figure}[h!]
\centering
\begin{tikzpicture}
\Vertex[x=-4.5, size=.5, color=white, label=$v_{1}$]{x1}
\Vertex[x = -2.5, size=.5, color=white, label=$v_{2}$]{x2}
\Vertex[y = 1.5, x=-3.5 ,size=.5,color=white, label=$v_{3}$]{x3}
\Edge[distance=0.5, label = $0$](x3)(x1)
\Edge[distance=0.5, label= $1$](x3)(x2)
\Edge[distance=0.5, label=$1$](x1)(x2)

\Text[x=-3.5 ,y=-.5]{$G_{Q}$}
\end{tikzpicture}
\caption{The query distance graph $G_{Q}$}
\label{querygraphex2}
\end{figure}

Therefore, the following traversals are also optimal when $G_{q_{3}} =  G_{b}$: 
$$q_{2} \rightarrow q_{3} \rightarrow q_{1}$$
$$q_{2} \rightarrow q_{1} \rightarrow q_{3}.$$
\end{remark}

\null \quad \quad Overall, the approach of optimally permuting a sequence of queries $Q = \{q_{1}, q_{2}, ..., q_{n}\}$ on a DAG $G$ entails determining optimal graphs $G_{q_{i}}$ in the $[G]$ for each query $q_{i}$, then structuring the optimal graphs as vertices $v_{i}$ of a query distance graph $G_{Q}$. That means, we must find the pairwise query distance between all queries in $Q$. Once we have done this, we must find a minimal spanning tree of $G_{Q}$. Using the fact that we can store intermediate graphs $G_{q_{i}}$ while transforming $G$, this minimal spanning tree provides us with an optimal permutation of the queries, a tree describing traversal of the queries, and a description of which edges must be reversed in order to move from one optimal graph to the next. \newline
\null \quad \quad As seen earlier, storing a graph representation is in $\mathcal{O}(n \cdot |V|^{2})$ where $n$ is the length of $Q$. Finding a minimum spanning tree of $G_{Q}$ (which has $\mathcal{O}(n)$-many vertices) is in $\mathcal{O}(|E|\cdot log(n)) = \mathcal{O}(n^{2} \cdot log(n))$ since $|E| = |n \times n|$ in a complete graph. This is $\mathcal{O}(n^{2}\cdot log(n))$. \newline
\null \quad \quad Next, computing the query distance between two queries is in $\mathcal{O}(|V|)$, where $|V|$ is the number of vertices in the original graph $G$. This must be done for each pair of queries, amounting to $\mathcal{O}(n^{2} \cdot |V|)$. In composite, this process can be quite expensive, but will certainly never return a worse permutation of $Q$ than the original one. \newline
\null \quad \quad In the worst case scenario, where the resulting permutation does not improve computation time, we earn no speedups. This method will not return a worse permutation that the original one, but the cost of searching may not be balanced by the benefits of the result. In general, however, the success of this method depends \textit{highly} upon the structures of the queries in $Q$. This is especially the case for queries which have significant  mutual variery in form (differing targets and conditions), since we can intuitively see that they are unlikely to share optimal graphs and therefore searching for beneficial permutations will more likely be advantageous. \newline












%%%%%%%%%%%%%%% CHAPTER 7 CONCLUSION 
% Conclusion
\newpage
\chapter{Conclusion}
\section{Results and Implications}
\null \quad \quad In this thesis, we have presented preliminaries from graph theory and Bayesian statistics to aid our understanding of Markov equivalence. We defined a query, identified the set $\Delta(G,q)$ upon which a query depends, quantified the cost of the computing an arbitrary query, and determined the speedups attainable for computing a query by reversing an edge in a Bayesian network. \newline
\null \quad \quad We explored major results in Markov equivalence, allowing us to understand the contexts in which edge reversals are possible. This understanding of Markov equivalence let us take advantage of the non-identifiability property of Bayesian networks by developing algorithms which find and reverse edges which reduce the complexity of inference for a specific query. Under reasonable restrictions on graph structure, we quantified the complexity of these algorithms and compared them to the speedups earned through using them. We concluded that the cost of identifying and reversing advantageous edges is almost always worth the speedups gained. \newline
\null \quad \quad Finally, we explored the implications of these results over sequences of queries, and developed tools to help us find optimized permutations of such sequences. We determined that one can develop a tree-structure to describe an optimal traversal of queries by developing a \textit{query distance graph}. \newline
\null \quad \quad \textbf{The major results of this thesis inform us that edge reversal \textit{is}, in most circumstances, an advantageous strategy for reducing the cost of computing a query.} They also give us a context for understanding of how to re-order sequences of queries to minimize cost. Furthermore, the frameworks behind many of these results may allow for others to easily and effectively explore the topics beyond the results presented here.  

\section{Future work} 
\null \quad \quad This work has relied on several restrictions in order to quantify the complexity of the algorithms used. Firstly, many of the scenarios explored were in the binary setting. With some patience, the logic of the binary setting can be extended to categorical variables. Further, in cases where quantifying the complexity of an algorithm is difficult-- say, if the graph has an unrestricted number of parents-- one could experimentally determine whether the greedy strategy is effective by timing it over a large number of randomly generated graphs. Then, even when the arithmetic complexity is unclear, one can gain an understanding of whether it is effective, and if so, under what circumstances. \newline
\null \quad \quad We also explored the greedy strategy, meaning we looked for a single effective edge reversal over the set of covered edges. In future work, one could explore other methods, such as generating the entire Markov equivalence class $[G]$ of a given input DAG $G$, and determining which graph $G' \in [G]$ minimizes the cost of the query. \newline
\null \quad \quad In more detailed consideration, one could compare the method presented in ~\cref{section:bgandconstruction} of determining which case (special or general) a query is in to the other proposed method; that is, compare the method of searching among all paths between target nodes and condition parents to the method of copying the graph, removing the conditional vertex, and checking the structure of the newly created graph.\newline
\null \quad \quad It is likely that the cost of finding beneficial edges to reverse can be reduced by more efficient algorithm design, as well as the space-requirement outlined in Lemma \ref{lem:representspace}.\newline
\null \quad \quad In general, the analysis in this thesis has been limited to only considering the worst-case complexities of the algorithms, as is typical for complexity analysis. However, on average the benefit of these algorithms will be much larger than the complexities presented suggest. Therefore, a valuable line of research would be to explore the more complicated problem of quantifying average-case complexities for each algorithm. Average-case analysis would provide much more insight into the actual benefit of these algorithms in practice. For example, in Theorem \ref{thm:reversalbenefit}, which states that reversing beneficial edges is worthwhile whenever $|V|$ \textgreater \  $22$, the lower bound is built on the assumption of a worst-case scenario. On average, the reversal may be beneficial in a wider set of scenarios. 


%%%%%%%%%%%%%%%FINALITIES
%references 
\newpage
\bibliography{bibliography}

\newpage
\chapter*{Appendix}

%%%%%%%%%%%%%%%%%%%%%%%%%%%%%%%%%%%%%
\section*{Python code for determining the set $\Delta(G,q)$}

For a DAG $G$, the following code determines which nodes are in the se $\Delta(G,q)$ for a query $q$. It uses two functions:
\begin{enumerate}
\item \texttt{find\_case}, a helper function which determines whether nodes are in the special case or the general case according to the logic from ~\cref{thm:querycost}.
\item \texttt{create\_delta\_set}, the main function which populates the set $\Delta(G,q)$ depending on the result of \texttt{find\_case}.
\end{enumerate}


\begin{python}

# Helper function for create_delta_set(); checks for paths between targets and parents of conditions.
# Inputs: a DAG $G$, a target vertex $X$, a single conditional vertex $Y \in \{Y_{0},Y_{1},...Y_{n}\}$ for the query $q = (X|Y_{0},Y_{1},....Y_{n})$.
#Returns: the case of individual vertex $Y$ in the query $q$, either special or general.

def find_case(G,X,Y,cases):
    skeleton = G.to_undirected()
    y_ancestors = list(nx.ancestors(G,Y))
    paths = list(nx.all_simple_paths(skeleton, X, Y, cutoff=None)) #use skeleton so they are undirected (routes not paths).

    if len(y_ancestors) == 0:
        cases[Y] = "special" #Yj in trivial special case
        return(cases)

    # The logic of the following is to find paths from X to Y: if any of them pass through an ancestor of Y then there is a "back route" to Y through its ancestors, meaning there is a path connecting X to a(Y)
    # which does not pass through Y. 
    for path in paths:
        for y_ancestor in y_ancestors:
            if y_ancestor in path: #There exists a path from X to a(Y) which does not pass through Y
                cases[Y] = "general" #Yj in general case 
                return(cases)
            else:
                continue
            
    #Only reaches this block of code if all paths to a(Y) passed through Y; in nontrivial special case.
    cases[Y] = "special" #Yj in the special case
    return(cases)	
    
\end{python}
    
    
\begin{python}
  
#Inputs: A DAG G, a target vertex X, and conditional vertices Ys = {Y_{0},Y_{1},...Y_{n}})$corresponding to a query q = p(X|Y_{0}, Y_{1},...Y_{n}). 
#Returns: the set Delta(G,q). 

def create_delta_set(G,X,Ys): #Graph G, one target X, set of multiple conditions Ys = {y0, y1, ...}.
    skeleton = G.to_undirected()
    
    dependent = []
    cases = {} #Dictionary with (key, value) pairs: (Y,case).
    if not nx.is_connected(skeleton): #Only consider connected graphs.
        return(0)
    if len(Ys) == 0:
        return(list(nx.ancestors(G,X)))
    
    for Y in Ys:
        cases = check_paths(G,X,Y,cases)
    
    shortest_path = nx.shortest_path(skeleton,X,Y) #use skeleton so we have shortest route (undirected)
    x_ancestors = list(nx.ancestors(G,X))
    y_ancestors = list(nx.ancestors(G,Y))
    
    dependent.append(X)
    for x_ancestor in x_ancestors:
        dependent.append(x_ancestor)

    for Y in Ys:
        if cases[Y] == "general":
            for y_ancestor in y_ancestors:
                dependent.append(y_ancestor)
                
        if cases[Y] == "special":
            for node in shortest_path:
                dependent.append(node)
            dependent = list(set(dependent))
            dependent.remove(Y)
            for y_ancestor in y_ancestors:
                if y_ancestor in dependent:
                    dependent.remove(y_ancestor)
    
    return(list(set(dependent)))
    
\end{python}


%%%%%%%%%%%%%%%%%%%%%%%%%%%%%%%%%%%%%
\section*{Python code for identifying advantageous reversals}

\null \quad \quad The following pieces of code determine which edge reversal is most advantageous at a single given time given a Dag $G$ and a query $q = (X|Y_{0},Y_{1},...,Y_{n})$. The first piece, \texttt{find\_covered\_edges} searches for covered edges since they are equivalent to immediately reversible edges. It does so by comparing the parents of each of the two vertices corresponding to an edge in $G$. 

\begin{python}
# Inputs: A DAG G
# Return the set of covered edges in a graph. Independent of query. 

def find_covered_edges(G):
    covered_edges = []
    for edge in G.edges: #for edge (X,Y)
        source_parents = list(G.predecessors(edge[0])) #parents of X
        target_parents = list(G.predecessors(edge[1])) #parents of Y
        source_parents.append(edge[0]) #parents of X union X
        if set(source_parents) == set(target_parents):
            covered_edges.append(edge)
    return(covered_edges)

\end{python}

\null \quad \quad Next, \texttt{is\_advantageous} takes in a DAG $G$, a covered edge $e$, and the parameters of the query $q = p(X|Y_{0},Y_{1},...,Y_{n})$. It creates a graph $Grev$, the graph which results if $e$ is reversed, and calculates $\Delta(Grev,q)$. Then, the difference between $\Delta(G,q)$ and $\Delta(Grev,q)$ is returned, indicating how advantageous the reversal is. 

\begin{python}
# Inputs: A DAG G, a covered edge, the parameters of the query q = p(X|Ys).
# Returns: The number of vertices Delta(G,q) is reduced by if the edge is reversed, that is, how advantageous the reversal is. 

def is_advantageous(G,edge, X, Ys):
    vertex_cost_difference = 0
    cost_before = len(find_case(G, X, Ys))
    
    x = edge[0]
    y = edge[1]
    Grev = G.copy()
    Grev = structure_reverse_edge(Grev,x,y)
    cost_after = len(find_case(Grev,X,Ys))
    if cost_after < cost_before:
        vertex_cost_difference = cost_before - cost_after
    return(vertex_cost_difference)

\end{python}

\null \quad \quad Finally, \texttt{best\_reversal} iterates through the set of covered edges in $G$ to determine which covered edge is most advantageous when reversed, and returns this edge. 

\begin{python}
# Input: A DAG G, target vertex X, the set of conditional vertices Ys which correspond to the query q = p(X|Ys). 
# Returns: the most advantageous edge reversal possible at a given time. 
def best_reversal(G, X, Ys):
    covered_edges = find_covered_edges(G) 
    edge_advantages = {}
    for edge in covered_edges:
         edge_advantages[edge] = is_advantageous(G,edge,X,Ys)
    max_adv_edge = max(edge_advantages, key=edge_advantages.get)
    return(max_adv_edge)

\end{python}
\newpage

%%%%%%%%%%%%%%%%%%%%%%%%%%%%%%%%%%%%%
\section*{Python code for encoding conditional node data}

\null \quad \quad The following code assume binary variables, that is, for each vertex $X$ in a DAG $G$ corresponding to a random variable, $X = x \in \{0,1\}.$ Encoding even binary variables is expensive and nontrivial, but the same arguments can be extended to categoricals for those patient enough to implement it. \newline
\null \quad \quad Each vertex in $G$ is specified by a set of conditionals corresponding to the values its parents can take. Therefore, the number of conditionals corresponds to the number of parents it has. For example, if $X$ is a parentless vertex, it will be specified by $p(X=1)$. Since we are in the binary case, this is sufficient, since $p(X=0) = 1-p(X=1)$.\newline
\null \quad \quad If the vertex $Y$ has a single parent $X$, it will be specified by $p(Y=1|X=0)$ and $p(Y=1|X=1)$ from which one can again determine $p(Y=0|X=0)$ and $p(Y=0|X=1)$. A node $Z$ with parents $X$ and $Y$ will be specified by $p(Z=1|X=0,Y=0)$, $p(Z=1|X=0,Y=1)$, $p(Z=1|X=1,Y=0)$ and $p(Z=1|X=1,Y=1)$. Generally, a vertex with $P$ parents will be specified by $2^{P}$ conditionals. Therefore, one must construct a dynamic data structure such as the following, which accounts for the variety in number of parents (and therefore conditionals) a vertex can have. 

\begin{python}
import itertools
class NodeData:
    '''
    Attributes:
        parents: a list of strings that contains the names of the parent nodes
        conditionals: a list of size 2^p (where p is the number of parents) 
            that contains the conditional probabilities of the RV of this node given all possible values for the parents
    '''
    def __init__(self, name, parents=None, conditionals=None):
        self.name = name
        self.parents = parents

        # initializing the data structures,
        # mostly stuff that ensures that everything also works for nodes without parents
        if parents == None:
            self.num_parents = 0
        else:
            self.num_parents = len(self.parents)

        if conditionals == None:
            self.conditionals = [0 for _ in range(2**self.num_parents)]
        else:
            self.conditionals = conditionals

            
    def set_conditional(self, p, parent_assignment=None):
        '''
        parent_assignment: a dict that maps values to names of parents, e.g.
        {'x1' : 0, 'x2' : 1, 'x3' : 0, ...}
        this function computes the index of self.conditionals where the required value is stored and sets the conditional probability
        '''
        # handle special case of node without parents:
        if self.parents == None:
            self.conditionals[0] = p
            return
        # general case:
        index = 0
        for p_i in range(len(self.parents)):
            index += parent_assignment[self.parents[p_i]] * 2**p_i

        self.conditionals[index] = p

        
    def get_conditional(self, parent_assignment):
        '''
        parent_assignment: a dict that maps values to names of parents, e.g.
        {'x1' : 0, 'x2' : 1, 'x3' : 0, ...}
        this function computes the index of self.conditionals where the required value is stored and returns the conditional
        '''
        
        ##input of get conditionals is parent_assignment, dict which grows linearly with number of parents. 
        ##So we can calculate this in terms of numparsx. 
        
        # handle special case of node without parents:
        if self.parents == None:
            return self.conditionals[0]
        # general case:
        index = 0
        for p_i in range(len(self.parents)):
            index += parent_assignment[self.parents[p_i]] * 2**p_i

        return self.conditionals[index]

    
    def print_conditionals(self):
        '''
        prints the full conditional distribution
        iterates through all possible assignments and prints the conditional distribution for each assignment in a seperate line
        '''
        all_possible_parent_assignments = [x for x in itertools.product([0,1], repeat=self.num_parents)]

        for a_i in range(2**self.num_parents):
            assignment = {self.parents[i] : all_possible_parent_assignments[a_i][i] for i in range(self.num_parents)}
            # some list-magic that produces a sufficiently nice string:
            assignment_string = "P("+self.name+" = 1 | " + ', '.join([str(x) 
            		+ " = " + str(assignment[x]) for x in assignment]) + " ) 
            		= " + str(self.get_conditional(assignment))
            print (assignment_string)
\end{python}

%%%%%%%%%%%%%%%%%%%%%%%%%%%%%%%%%%%%%
\section*{Python code for calculating edge reversals in a binary setting}

Likewise to the data structures above, the following code assumes binary variables. It must also dynamically access the conditional probabilities specifying each node, since the length of specifiers of a vertex with $P$ parents is $2^{P}$. 

\begin{python}
def switch_edge(x, y):
    '''
    switches the edge (x,y) to (y,x)
    x, y: NodeData
    '''
    # preparation:
    # use itertools.product to get all possible assignments of values for the parent nodes:
    
    all_possible_parent_assignments = list(itertools.product([0,1], repeat=x.num_parents))
    num_parents_assigments = len(all_possible_parent_assignments)

    # get conditional joint distribution p(x,y | parents):
    # initialization:
    pxe0ye0 = [0 for _ in range(2**x.num_parents)]  
    pxe0ye1 = [0 for _ in range(2**x.num_parents)]
    pxe1ye0 = [0 for _ in range(2**x.num_parents)]
    pxe1ye1 = [0 for _ in range(2**x.num_parents)]
    
    
    # iterate through all possible assignments for the common parents of x and y:
    for a_i in range(num_parents_assigments):
        
        assignemnt_dict_x = {x.parents[i] : all_possible_parent_assignments[a_i][i] for i in range(x.num_parents)}
        assignemnt_dict_y = {x.parents[i] : all_possible_parent_assignments[a_i][i] for i in range(x.num_parents)}
        assignemnt_dict_y[x.name] = 0
        
        # case x=0, y=1:
        pxe0ye1[a_i] = (1-x.get_conditional(assignemnt_dict_x)) 
        		* y.get_conditional(assignemnt_dict_y)
        # case x=0, y=0:
        pxe0ye0[a_i] = (1-x.get_conditional(assignemnt_dict_x)) 
        		* (1-y.get_conditional(assignemnt_dict_y))

        assignemnt_dict_y[x.name] = 1
        # case x=1, y=1:
        pxe1ye1[a_i] = x.get_conditional(assignemnt_dict_x) 
        		* y.get_conditional(assignemnt_dict_y)
        # case x=1, y=0:
        pxe1ye0[a_i] = x.get_conditional(assignemnt_dict_x) 
        		* (1-y.get_conditional(assignemnt_dict_y))

    
    # marginalize x from the conditional distribution of y:
    pye0 = [pxe0ye0[i] + pxe1ye0[i] for i in range(num_parents_assigments)]
    pye1 = [pxe0ye1[i] + pxe1ye1[i] for i in range(num_parents_assigments)]

    # condition x on the possible values of y:
    pxe0gye0 = [pxe0ye0[i]/pye0[i] for i in range(num_parents_assigments)]
    pxe0gye1 = [pxe0ye1[i]/pye1[i] for i in range(num_parents_assigments)]
    pxe1gye0 = [pxe1ye0[i]/pye0[i] for i in range(num_parents_assigments)]
    pxe1gye1 = [pxe1ye1[i]/pye1[i] for i in range(num_parents_assigments)]

    # construct new nodedata objects:
    x_new_parents = [y.name]
    if not x.parents == None:
        x_new_parents += x.parents
    x_new = NodeData(x.name, parents=x_new_parents)

    for a_i in range(num_parents_assigments):
        
        # construct the assignment-dictionary:
        assignment_dict = {x.parents[i] : all_possible_parent_assignments[a_i][i] for i in range(x.num_parents)}

        # case y=0:
        # add y to assignment_dict:
        assignment_dict[y.name] = 0
        # set conditional for x_new:
        x_new.set_conditional(p=pxe1gye0[a_i], parent_assignment=assignment_dict)

        # case y=1
        # add y to assignment_dict:
        assignment_dict[y.name] = 1
        # set conditional for x_new:
        x_new.set_conditional(p=pxe1gye1[a_i], parent_assignment=assignment_dict)

    y_new_parents = []
    
    if not x.parents == None:
        y_new_parents += x.parents
        
    y_new = NodeData(y.name, parents=y_new_parents)
    
    for a_i in range(num_parents_assigments):
        # construct the assignment-dictionary:
        assignment_dict = {x.parents[i] : all_possible_parent_assignments[a_i][i] for i in range(x.num_parents)}
        y_new.set_conditional(p=pye1[a_i], parent_assignment=assignment_dict)

    return x_new, y_new
\end{python}

% declaration of authorship
\newpage
{\center{\section*{Declaration of Authorship}}}
\thispagestyle{empty}

\null \vskip 1em

I hereby declare that I have completed this work independently and using only the sources and tools specified. The author has no objection to making the present Master's thesis available for public use in the university archive.

\null \vskip 1em

Ich erkläre, dass ich die vorliegende Arbeit selbstständig und nur unter Verwendung der angegebenen Quellen und Hilfsmittel angefertigt habe. Seitens des Verfassers bestehen keine Einwände die vorliegende Masterarbeit für die öffentliche Benutzung im Universitätsarchiv zur Verfügung zu stellen.

\vskip 6em
 
 % author
\begin{minipage}[t]{0.5\textwidth}
\begin{flushleft} 
Ort, Abgabedatum \newline
(Place, date)
\end{flushleft}
\end{minipage}
~
\begin{minipage}[t]{0.5\textwidth}
\begin{flushright} 
Unterschrift der Verfasserin \\
(Signature of the Author) \ \ \ \ \ \ 
\end{flushright}
\end{minipage}




\end{document}
